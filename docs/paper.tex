\documentclass[11pt]{article}
\usepackage[utf8]{inputenc}
\usepackage[T1]{fontenc}
\usepackage{geometry}
\geometry{letterpaper,margin=1in}
\usepackage{amsmath,amssymb,amsthm,mathtools,bm}
\usepackage{bbm}
\usepackage{booktabs}
\usepackage{graphicx}
\usepackage{hyperref}
\usepackage[square]{natbib}
\usepackage{enumitem}
\usepackage{caption}
\usepackage{xcolor}
\usepackage{mdframed}
\hypersetup{
    colorlinks=true,
    linkcolor=blue!60!black,
    citecolor=blue!60!black,
    urlcolor=blue!60!black
}

\setlist[itemize]{leftmargin=1.5em}
\setlist[enumerate]{leftmargin=1.5em}

\newcommand{\E}{\mathbb{E}}
\newcommand{\Var}{\mathbb{V}\mathrm{ar}}
\newcommand{\Cov}{\mathbb{C}\mathrm{ov}}
\newcommand{\R}{\mathbb{R}}

\title{A Research Program for State-Dependent DTS Scaling in Corporate Credit Spreads:\\
Integrating Structural Model Priors with Empirical Estimation}

\author{%
Bernd J. Wuebben \\
AllianceBernstein, New York\\
\texttt{bernd.wuebben@alliancebernstein.com}
}

\date{\today}

\begin{document}

\maketitle

\begin{abstract}
\noindent This document presents a comprehensive research program for enhancing the Duration-Times-Spread (DTS) framework by integrating theoretical predictions from structural credit models with empirical estimation. Recent theoretical work \cite{Wuebben2025} demonstrates that proportional spread movements fail systematically across maturities even for investment-grade bonds, with 1-year bonds exhibiting 4--6$\times$ higher percentage spread sensitivity than 10-year bonds from the same issuer. These structural model predictions provide strong priors for the state-dependent elasticity $\lambda_{i,t}$ that we seek to estimate empirically.

The research program proceeds sequentially: (1) \textbf{Raw validation} testing structural model predictions through a three-pronged approach---bucket-level analysis, within-issuer tests, and sector interaction analysis---before any complex regression modeling; (2) \textbf{Establishing variation} documenting that DTS betas differ systematically across bonds; (3) \textbf{Theory testing} evaluating whether Merton predictions explain observed variation; (4) \textbf{Time-variation analysis} assessing stability of relationships; (5) \textbf{Robustness checks} examining tail behavior, shock decomposition, and liquidity effects; (6) \textbf{Production specification} selecting parsimonious model using hierarchical testing framework.

Key improvements over standard approaches: (i) separation of ``is there variation?'' (Stage A) from ``does theory explain it?'' (Stage B); (ii) three complementary Stage 0 methodologies---bucket-level for aggregate patterns, within-issuer for pure maturity effects holding credit quality constant, and sector interactions for formal inference about industry-specific deviations; (iii) hierarchical model selection guided by theory rather than atheoretical horse-racing; (iv) explicit treatment of unbalanced panels as methodological choice, not separate research stage.

The research delivers both academic and practical value: academically, we provide the first comprehensive empirical test of structural model spread dynamics using market-wide bond data with proper identification of within-issuer effects; practically, we determine whether simpler Merton-based adjustment factors suffice or whether complex empirical estimation with sector-specific refinements is necessary for production risk systems.

\vspace{0.3cm}

\noindent \textbf{Structure:} Sections 1--2 cover objectives and data construction (including issuer identification methodology critical for within-issuer analysis). Section 3 introduces structural model priors. Section 4 (Stage 0) conducts raw validation using three complementary approaches: bucket-level analysis, within-issuer tests, and sector interaction analysis. Sections 5--6 (Stages A--B) establish and explain cross-sectional variation. Sections 7--11 cover time-variation, robustness, production specification selection, and research task summary.
\end{abstract}

\newpage
\tableofcontents

\newpage

\section{Objectives and High-Level Framework}

\subsection{Motivation and constraints}

In practice, DTS is already embedded in risk systems, attribution frameworks, and portfolio construction for credit portfolios. Any enhanced modeling of spread dynamics must therefore:

\begin{itemize}
\item remain \emph{DTS-centric}: spreads and returns should be modeled as functions of $DTS_{i,t}$ and index-level DTS factors;
\item produce enhancements in the form of a \emph{multiplicative adjustment} $\lambda_{i,t}$, rather than replacing DTS altogether;
\item be implementable and interpretable for risk management, performance attribution, and relative-value work.
\end{itemize}

The empirical question is: \medskip

\emph{For a given macro environment, sector, rating, maturity, liquidity state, and idiosyncratic issuer profile, how do exogenous shocks---macro, sectoral, and issuer-specific---translate into spread movements, and how should these sensitivities be scaled relative to standard DTS?}

\subsection{Core modeling equation and DTS anchor}

We work in discrete time at observation dates $t \in \{1,\dots,T\}$. For each bond $i$ in universe $U$ (IG or HY), denote:

\begin{itemize}
\item $OAS_{i,t}$: option-adjusted spread;
\item $OASD_{i,t}$: OAS-based spread duration;
\item $DTS_{i,t} = OAS_{i,t} \cdot OASD_{i,t}$: standard DTS;
\item $OAS^{(U)}_{t}$: index-level OAS for universe $U$ (IG or HY);
\item $f^{(U)}_{DTS,t} = \frac{\Delta OAS^{(U)}_{t}}{OAS^{(U)}_{t-1}}$: index-level relative spread change, our DTS factor return.
\end{itemize}

We define bond-level relative spread changes as
\begin{equation}
y_{i,t} \equiv \frac{\Delta OAS_{i,t}}{OAS_{i,t-1}} = \frac{OAS_{i,t} - OAS_{i,t-1}}{OAS_{i,t-1}}.
\end{equation}

The canonical DTS assumption for bond $i$ in universe $U$ is
\begin{equation}
y_{i,t} \approx f^{(U)}_{DTS,t} + \varepsilon_{i,t},
\end{equation}
up to idiosyncratic error terms. We \emph{retain} the DTS factor-centric structure, but we generalize it to permit a state-dependent elasticity $\lambda_{i,t}$:
\begin{equation}
y_{i,t} \approx \lambda_{i,t} \, f^{(U)}_{DTS,t} + \varepsilon_{i,t}.
\label{eq:core-lambda}
\end{equation}

In excess-return space, letting $r^{e}_{i,t}$ denote the excess return of bond $i$ over a duration-matched Treasury or swap hedge, we want to write
\begin{equation}
r^{e}_{i,t} \approx OAS_{i,t-1}\Delta t + DTS_{i,t}\, \lambda_{i,t}\, f^{(U)}_{DTS,t} + \text{idiosyncratic term}.
\end{equation}

The research program is thus centered on specifying, estimating, and validating $\lambda_{i,t}$ as a function of observable bond characteristics and state variables.

\subsection{Universe split: IG and HY}

We can define the following two universes:

\begin{enumerate}
\item $U = \text{IG}$: the Bloomberg Barclays U.S.\ Corporate Investment Grade Index; and
\item $U = \text{HY}$: the Bloomberg Barclays U.S.\ High Yield 2\% Issuer Cap Index.
\end{enumerate}

It is not clear as to whether to pool IG and HY in estimation. To start with we pool IG and HY bond observations. The aim is to produce a unified $\lambda$ specification for IG and HY, with potentially different parameterizations and state dependencies. Depending on findings we may have to split the universe into IG and HY bonds.

\subsection{Integration with structural model theory}

This research program explicitly incorporates predictions from structural credit models (specifically the Merton framework) as theoretical priors for $\lambda_{i,t}$. Following \cite{Wuebben2025}, structural credit theory establishes that: 

\begin{enumerate}
\item \textbf{Cross-maturity proportionality fails severely in IG:} At spread levels below 300 bps, bonds with 1-year maturity have 4--6$\times$ higher percentage spread sensitivity than 10-year bonds from the same issuer.

\item \textbf{Maturity effects dominate credit quality effects:} In investment-grade markets, cross-maturity elasticity ratios of 400--500\% dwarf same-maturity credit quality variations of 20--35\%.

\item \textbf{Functional form predictions:} The Merton model implies specific functional forms for $\lambda$ as functions of spread level $s$, maturity $T$, and the ratio $R/(sT)$ where $R$ captures the fraction of debt value at risk.

\item \textbf{Regime-dependent patterns:} Proportionality failures are most severe in investment-grade (spreads $<$300 bps, maturity dispersion $>$3 years), moderate in high-yield (300--1000 bps), and paradoxically improve in distressed markets ($>$1000 bps).
\end{enumerate}

These theoretical predictions provide powerful benchmarks for empirical estimation. Rather than searching blindly for $\lambda$ patterns in data, we can:

\begin{itemize}
\item Test whether observed elasticity ratios match Merton predictions
\item Use structural adjustment factors as starting values for iterative estimation
\item Decompose empirical $\lambda$ into theory-consistent and residual components
\item Identify systematic deviations suggesting additional mechanisms (liquidity, sentiment, technical factors)
\end{itemize}

The research program therefore operates on two parallel tracks throughout:

\medskip
\begin{mdframed}[backgroundcolor=blue!5!white,roundcorner=4pt]
\textbf{Track 1: Theory-Guided Estimation}

Start from Merton-implied $\lambda^{\text{Merton}}(s_i, T_i)$ and test whether empirical data support these predictions. Estimate deviations and assess their economic significance.

\noindent \textbf{Track 2: Unrestricted Estimation}

Allow data to determine $\lambda$ coefficients without structural constraints. Compare to Track 1 to assess incremental explanatory power of unconstrained specifications.
\end{mdframed}

\subsection{Sequential research design}

The stages proceed hierarchically with explicit decision points:

\begin{mdframed}[backgroundcolor=green!5!white,roundcorner=4pt]
\textbf{Stage 0: Raw Validation (Three-Pronged Approach)}

Test Merton predictions directly from spread changes before any regression analysis using three complementary methodologies:
\begin{itemize}
\item \textbf{Bucket-level analysis:} Aggregate patterns across rating $\times$ maturity $\times$ sector buckets
\item \textbf{Within-issuer analysis:} Pure maturity effects holding credit quality constant
\item \textbf{Sector interaction analysis:} Formal inference about industry-specific deviations
\end{itemize}
\emph{Decision: Does theory provide adequate baseline? Where does it need augmentation?}

\noindent \textbf{Stage A: Establish Variation}

Document that DTS betas differ across bonds. \emph{Decision: If no variation, stop—standard DTS sufficient.}

\noindent \textbf{Stage B: Explain Variation}

Test whether Merton predictions match observed beta patterns. \emph{Decision: Pure Merton / Calibrated / Need flexibility?}

\noindent \textbf{Stage C: Test Stability}

Assess whether static $\lambda$ suffices or time-variation required. \emph{Decision: Add macro state or keep static?}

\noindent \textbf{Stage D: Robustness}

Only after Stages A--C conclusive: test tail behavior, shock decomposition, liquidity. \emph{Diagnose where/why theory fails.}

\noindent \textbf{Stage E: Production Specification}

Hierarchical selection: Standard DTS $\to$ Pure Merton $\to$ Calibrated $\to$ Empirical (with sector adjustments where warranted). \emph{Stop at simplest adequate model.}
\end{mdframed}

\textbf{Key principle:} Each stage conditional on previous results. Don't add complexity until simple models demonstrably fail.


\section{Data Construction and Sample Filters}

\subsection{Sample period}

\textbf{Primary sample}: January 2013 -- October 2025

\textbf{Rationale}:
\begin{itemize}
\item Excludes 2008--2009 crisis (extreme outlier, liquidity breakdown when Merton least reliable)
\item 12+ years provides multiple credit cycles (2013 taper tantrum, 2015--16 energy crisis, 2020 COVID, 2022 rate shock)
\end{itemize}

\textbf{Robustness sample}: Use ICE data from January 2002 -- December 2024

Test whether results hold when including crisis period (separate analysis, not primary).

\subsection{Universe definition}

For each universe $U \in \{\text{IG}, \text{HY}\}$:

\begin{itemize}
\item At each date $t$, the cross-section $\mathcal{I}_{U,t}$ consists of all bonds that are constituents of the relevant Bloomberg Barclays index on that date.
\item We accept the full index methodology: new issues enter when eligible; bonds exit upon maturity, call, downgrade, or when they fail inclusion criteria.
\item There is \emph{no artificial restriction} to fixed-coupon non-call bullet bonds beyond what the index methodology already imposes.
\end{itemize}

Consequently, the panel $(i,t)$ is naturally \emph{unbalanced}: many bonds appear only for part of the sample; new issues are frequent, particularly in IG. We \emph{embrace} this unbalanced nature and design the econometric work to handle it explicitly (see Section~\ref{sec:panel-structure}).

\subsection{Expected sample size}

Based on Bloomberg Barclays index historical composition:

\begin{table}[h]
\centering
\begin{tabular}{lrr}
\toprule
& \textbf{IG} & \textbf{HY} \\
\midrule
Avg bonds per month & 5,500 & 1,800 \\
Issuers with 2+ bonds outstanding & 800 & 250 \\
Issuer-weeks for within-issuer tests & 180,000 & 45,000 \\
\bottomrule
\end{tabular}
\caption{Expected sample size for primary sample (2013--2025)}
\end{table}

This provides adequate power for all Stage 0 tests:
\begin{itemize}
\item Bucket-level analysis benefits from large samples per bucket (5,000--20,000 bond-weeks for major buckets)
\item Within-issuer analysis has 225,000+ issuer-weeks for pooled estimation
\item Sector interaction analysis pools millions of bond-week observations across sectors
\end{itemize}

\subsection{Core variables}

For each bond $i \in \mathcal{I}_{U,t}$ and date $t$:

\begin{itemize}
\item $OAS_{i,t}$ (in basis points);
\item $OASD_{i,t}$ (in years);
\item $DTS_{i,t} = OAS_{i,t}\cdot OASD_{i,t}$;
\item clean price, accrued interest, yield-to-worst;
\item time to maturity $M_{i,t}$ (years);
\item index sector classification (e.g.\ Industrials, Financials, Utilities, etc.);
\item credit rating, mapped to coarse buckets (e.g.\ AAA/AA, A, BBB for IG; BB, B, CCC for HY);
\item issue size (amount outstanding), floating vs fixed, seniority;
\item liquidity proxies where available: TRACE trading volume, bid-ask spread, number of trades, turnover.
\end{itemize}

For the index level (per universe $U$):

\begin{itemize}
\item $OAS^{(U)}_t$ and $OASD^{(U)}_t$, to define $DTS^{(U)}_t$ and $f^{(U)}_{DTS,t}$;
\item sector- and rating-specific index OAS series $OAS^{(U)}_{s,t}$, $OAS^{(U)}_{r,t}$ for sector $s$ and rating $r$ (for Stage D decomposition).
\end{itemize}

\subsection{Time frequency and return definitions}

We start with \emph{weekly} frequency to mitigate microstructure noise:

\begin{itemize}
\item Define a weekly observation grid $t_1, t_2, \dots, t_T$ as e.g.\ Fridays or the last trading day of each week.
\item Construct $OAS_{i,t}$ and $OAS^{(U)}_t$ on that grid by taking end-of-day values.
\end{itemize}

Define:
\begin{align}
y_{i,t} &\equiv \frac{\Delta OAS_{i,t}}{OAS_{i,t-1}} = \frac{OAS_{i,t} - OAS_{i,t-1}}{OAS_{i,t-1}}, \\
f^{(U)}_{DTS,t} &\equiv \frac{\Delta OAS^{(U)}_{t}}{OAS^{(U)}_{t-1}} = \frac{OAS^{(U)}_{t} - OAS^{(U)}_{t-1}}{OAS^{(U)}_{t-1}}.
\end{align}

\textbf{Robustness}: Re-run key specifications (Stages A--B) at daily and monthly frequencies to assess sensitivity to horizon.

\subsection{Panel structure and econometric approach}
\label{sec:panel-structure}

\subsubsection{Unbalanced panel by design}

The Bloomberg Barclays indices have substantial turnover:
\begin{itemize}
\item New issues enter monthly (IG: $\sim$50--100/month, HY: $\sim$20--50/month)
\item Bonds mature, are called, or exit via downgrade/upgrade
\item Average bond stays in IG index 6.5 years, HY index 4.2 years
\end{itemize}

We \textbf{embrace} this unbalanced structure rather than restricting to balanced sub-panels, which would:
\begin{itemize}
\item Eliminate 60\%+ of observations
\item Introduce survivorship bias (old bonds differ systematically from new)
\item Miss new issue dynamics (relevant for practitioners)
\end{itemize}

\subsection{Issuer identification methodology}
\label{sec:issuer-identification}

Proper issuer identification is \textbf{critical} for the within-issuer analysis in Stage 0. We must correctly aggregate bonds to their Ultimate Parent to isolate pure maturity effects from cross-issuer heterogeneity.

\subsubsection{Primary identification approach}

\begin{itemize}
\item Use \textbf{Ultimate Parent ID} from Bloomberg or other data provider
\item Match on \textbf{Seniority} (only compare senior unsecured to senior unsecured, subordinated to subordinated, etc.)
\item Exclude subsidiaries with materially different credit profiles
\item Manual validation for large issuers with complex corporate structures
\end{itemize}

\subsubsection{Edge cases requiring special treatment}

\begin{enumerate}
\item \textbf{Holding company vs operating subsidiary bonds:} Treat separately unless explicit guarantee exists. For example, separate bank holding company debt from bank operating company debt.

\item \textbf{Guaranteed debt:} Include in parent's bond set if guarantee is from Ultimate Parent and unconditional.

\item \textbf{Secured vs unsecured debt:} Analyze separately---different recovery rates imply different Merton elasticities even for same issuer.

\item \textbf{Acquisition/merger situations:} When issuer A acquires issuer B, bonds may trade differently during transition period. Flag issuer-weeks within 6 months of major corporate actions.

\item \textbf{Multi-national issuers:} Some issuers have bonds in multiple currencies or jurisdictions. Focus on USD-denominated bonds in primary analysis; cross-currency effects are a separate robustness check.
\end{enumerate}

\subsubsection{Validation procedure}

For the 50 largest issuers by amount outstanding (covering approximately 40\% of index market value):

\begin{enumerate}
\item Manual review of corporate structure
\item Verification that all bonds sharing Ultimate Parent ID have similar credit profiles
\item Documentation of any bonds excluded due to structural subordination or guarantee differences
\item Cross-check against rating agency issuer hierarchies
\end{enumerate}

\subsubsection{Sample filters for within-issuer analysis}

For each issuer-week $(i,t)$, include in within-issuer analysis if:

\begin{enumerate}
\item $J_{i,t} \geq 2$ (at least two bonds outstanding from same Ultimate Parent + Seniority)
\item All bonds have valid OAS, maturity, and spread data
\item Maturity dispersion $\geq$ 2 years: $\max_j T_{ij} - \min_j T_{ij} \geq 2$
\item No bonds within 1 year of maturity (avoid pull-to-par distortions and front-end liquidity effects)
\item Spread changes $|\Delta s_{ij}/s_{ij}| < 200\%$ (exclude extreme outliers likely due to data errors or restructuring events)
\end{enumerate}


\section{Structural Model Priors for $\lambda_{i,t}$}
\label{sec:structural-priors}

Before proceeding to empirical estimation, we establish theoretical benchmarks from structural credit models that provide strong priors for the functional form and magnitude of $\lambda_{i,t}$. The theoretical framework developed in \cite{Wuebben2025} provides precise predictions for when and why proportional spread movements fail.

\subsection{The Merton framework and spread elasticity}

In the Merton (1974) structural model, credit spreads arise from the embedded put option in risky debt. The key quantity governing spread dynamics is the \textbf{elasticity} of the spread with respect to firm value:

\begin{equation}
\varepsilon_i \equiv \frac{\partial s_i}{\partial V} \cdot \frac{V}{s_i}
\end{equation}

This measures the percentage change in spread for a 1\% change in firm value. For bonds from the same issuer experiencing a common firm value shock $\Delta V/V$, percentage spread changes are:

\begin{equation}
\frac{\Delta s_i}{s_i} \approx \varepsilon_i \cdot \frac{\Delta V}{V}
\end{equation}

Therefore, proportional spread movements require $\varepsilon_i = \varepsilon_j$ for all bonds $i,j$. The Merton model delivers an exact formula:

\begin{equation}
\varepsilon_i = -\frac{R_i}{T_i \cdot s_i}
\label{eq:merton-elasticity}
\end{equation}

where $T_i$ is maturity, $s_i$ is spread, and $R_i$ is the ratio:

\begin{equation}
R_i = \frac{e^{x_i + rT_i} N(-d_{1,i})}{N(d_{2,i}) + e^{x_i + rT_i} N(-d_{1,i})}
\end{equation}

with $x_i = \ln(V_i/D_i)$ the log-leverage ratio and $d_{1,i}, d_{2,i}$ the standard Black-Scholes parameters.

\textbf{Economic interpretation:} $R_i$ measures the fraction of debt value coming from default states—the ``value at risk.'' For IG bonds, $R \approx 0.05$--$0.30$; for HY, $R \approx 0.30$--$0.80$; for distressed, $R \to 1$.

See \cite{Wuebben2025} for detailed derivations and calibrations of these relationships.

\subsection{Adjustment factors from theory}

Define the adjustment factor for bond $i$ relative to a reference bond $j$ as the ratio of elasticities:

\begin{equation}
\lambda_{i,j} \equiv \frac{\varepsilon_i}{\varepsilon_j} = \frac{R_i \cdot T_j \cdot s_j}{R_j \cdot T_i \cdot s_i}
\label{eq:lambda-theory}
\end{equation}

For practical implementation, we decompose into \textbf{maturity adjustments} (same issuer, different $T$) and \textbf{credit quality adjustments} (different issuers, same $T$).

\subsubsection{Maturity adjustment factors}

For bonds from the same issuer with reference maturity $T_{\text{ref}} = 5$ years:

\begin{equation}
\lambda_T(T; s^*) = \frac{R(T, s^*) \cdot T_{\text{ref}} \cdot s(T_{\text{ref}}, s^*)}{R(T_{\text{ref}}, s^*) \cdot T \cdot s(T, s^*)}
\end{equation}

where $s^*$ is the calibrated spread at $T_{\text{ref}}$ determining the issuer's leverage.

\textbf{Key theoretical predictions:}

\begin{enumerate}
\item At IG spread levels (50--300 bps), $\lambda_T(1y) \approx 2.4$--$3.6$ (short bonds much more sensitive)
\item At IG spread levels, $\lambda_T(10y) \approx 0.6$--$0.7$ (long bonds less sensitive)
\item Adjustment factors converge toward 1.0 as spreads widen to HY levels
\item The cross-maturity effect dominates: elasticity ratio 1y/10y ranges from 5.9 at 50 bps to 1.56 at 1000 bps
\end{enumerate}

Table~\ref{tab:merton-lambda-T} summarizes theoretical predictions.

\begin{table}[ht]
\centering
\caption{Theoretical Maturity Adjustment Factors $\lambda_T$ from Merton Model. Source: \cite{Wuebben2025}.}\label{tab:merton-lambda-T}
\begin{tabular}{lrrrrr}
\toprule
\textbf{Spread Level} & \multicolumn{5}{c}{\textbf{Maturity Adjustment $\lambda_T(T; 5y)$}} \\
\textbf{(bps)} & 1y & 3y & 5y & 7y & 10y \\
\midrule
50 (AAA) & 3.62 & 1.47 & 1.00 & 0.79 & 0.61 \\
100 (AA) & 3.27 & 1.42 & 1.00 & 0.80 & 0.64 \\
200 (A-) & 2.78 & 1.36 & 1.00 & 0.82 & 0.67 \\
300 (BBB-) & 2.40 & 1.30 & 1.00 & 0.84 & 0.70 \\
\midrule
500 (BB) & 1.91 & 1.25 & 1.00 & 0.86 & 0.73 \\
1000 (B-) & 1.26 & 1.12 & 1.00 & 0.91 & 0.81 \\
2000 (CCC) & 1.08 & 1.05 & 1.00 & 0.97 & 0.93 \\
\bottomrule
\end{tabular}
\end{table}

\subsubsection{Credit quality adjustment factors}

For bonds with same maturity $T=5y$ but different spread levels, relative to reference $s_{\text{ref}}=100$ bps:

\begin{equation}
\lambda_s(s; 100) = \frac{R(s) \cdot 100}{R(100) \cdot s}
\end{equation}

\textbf{Key theoretical predictions:}

\begin{enumerate}
\item IG bonds safer than 100 bps have $\lambda_s > 1$ (higher sensitivity)
\item IG bonds riskier than 100 bps have $\lambda_s < 1$ (lower sensitivity)
\item Effect is moderate in IG: from 50 to 300 bps, $\lambda_s$ varies from 1.15 to 0.75 (only 35\% range)
\item Power-law approximation: $\lambda_s(s) \approx (s/100)^{-0.25}$ achieves $R^2=0.92$
\end{enumerate}

Table~\ref{tab:merton-lambda-s} summarizes.

\begin{table}[ht]
\centering
\caption{Theoretical Credit Quality Adjustment Factors $\lambda_s$ from Merton Model. Source: \cite{Wuebben2025}.}\label{tab:merton-lambda-s}
\begin{tabular}{lrr}
\toprule
\textbf{Spread (bps)} & \textbf{Exact Merton} & \textbf{Power Law $(s/100)^{-0.25}$} \\
\midrule
50 & 1.145 & 1.189 \\
100 & 1.000 & 1.000 \\
200 & 0.847 & 0.841 \\
300 & 0.746 & 0.760 \\
\midrule
500 & 0.635 & 0.669 \\
1000 & 0.468 & 0.562 \\
2000 & 0.299 & 0.473 \\
\bottomrule
\end{tabular}
\end{table}

\subsection{Regime-specific predictions}

The Merton model predicts distinct behavior across five regimes \cite{Wuebben2025}:

\begin{enumerate}[leftmargin=*, label=\textbf{Regime \arabic*:}]
\item \textbf{IG with narrow maturity range} ($s < 300$ bps, $\Delta T < 2y$): Standard DTS works reasonably. Maturity effects not applicable. Credit quality variation causes 20--35\% deviation—acceptable.

\item \textbf{IG with wide maturity range} ($s < 300$ bps, $\Delta T > 3y$): \textbf{Primary failure mode}. Cross-maturity $\lambda$ ratios of 3--6$\times$ create 300--500\% deviations. This is where DTS models systematically fail.

\item \textbf{HY with narrow maturity range} ($300 < s < 1000$ bps, $\Delta T < 2y$): Cross-maturity effects reduced to 50--160\% but still substantial. Same-maturity credit quality variation increases to 40--73\%.

\item \textbf{HY with wide maturity range} ($300 < s < 1000$ bps, $\Delta T > 3y$): Both effects large. Comprehensive adjustments required.

\item \textbf{Distressed} ($s > 1000$ bps): Proportionality paradoxically improves. Both cross-maturity and same-maturity deviations decline.
\end{enumerate}

\subsection{Empirical testing strategy}

The structural model priors suggest a specific empirical testing sequence:

\begin{mdframed}[backgroundcolor=green!5!white,roundcorner=4pt]
\textbf{Empirical Questions for Each Stage:}

These questions test the theoretical predictions established in \cite{Wuebben2025}:

\begin{enumerate}
\item \textbf{Do raw data exhibit Merton-predicted patterns?} (Stage 0) Before any regression, compute elasticity ratios and compare to theory using three complementary approaches:
\begin{itemize}
\item Bucket-level: Do aggregate patterns match?
\item Within-issuer: Does the mechanism work at issuer level?
\item Sector interactions: Are there systematic industry deviations?
\end{itemize}

\item \textbf{Is there cross-sectional variation to explain?} (Stage A) Document that betas differ. If not, standard DTS sufficient.

\item \textbf{Does Merton explain the variation?} (Stage B) Test whether theoretical $\lambda$ matches empirical patterns.

\item \textbf{Is the relationship stable over time?} (Stage C) If Stage B works, test whether static $\lambda$ suffices.

\item \textbf{Where does theory fail?} (Stage D) Diagnose failures: tails? liquidity? specific regimes?

\item \textbf{What production spec is adequate?} (Stage E) Hierarchical selection guided by theory, with sector adjustments where warranted.
\end{enumerate}
\end{mdframed}

\section{Stage 0: Raw Validation Using Three Complementary Approaches}
\label{sec:stage0}

\subsection{Objective and overview}

Stage 0 provides assumption-free tests of Merton predictions before any complex regression analysis. We ask: \emph{Do bonds exhibit spread sensitivities consistent with structural theory, and where do systematic deviations occur?}

To answer this comprehensively, we employ three complementary methodologies:

\begin{enumerate}
\item \textbf{Bucket-level analysis:} Groups bonds by observable characteristics (rating $\times$ maturity $\times$ sector) and estimates aggregate DTS sensitivities. Provides strong statistical power and tests whether Merton predictions hold on average across the market.

\item \textbf{Within-issuer analysis:} Examines bonds from the same issuer but different maturities, holding credit quality constant. Provides the cleanest test of Merton's cross-maturity predictions by isolating pure maturity effects.

\item \textbf{Sector interaction analysis:} Pools all bonds and formally tests whether sectors differ systematically in DTS sensitivity. Distinguishes real structural differences from sampling noise and identifies where sector-specific adjustments may be necessary.
\end{enumerate}

\begin{table}[h]
\centering
\caption{Complementary Roles of Three Stage 0 Approaches}
\label{tab:stage0-approaches}
\begin{tabular}{p{2.8cm}p{5.5cm}p{5.5cm}}
\toprule
\textbf{Approach} & \textbf{Strength} & \textbf{Limitation} \\
\midrule
Bucket-level & 
Large samples per bucket; strong statistical power; tests aggregate patterns & 
Mixes different issuers; cannot isolate pure maturity effect; descriptive sector comparisons only \\
\midrule
Within-issuer & 
Cleanest Merton test; same credit quality; pure maturity variation & 
Smaller samples per issuer-week; requires multiple bonds per issuer; cannot test cross-issuer patterns \\
\midrule
Sector interaction & 
Formal significance tests; distinguishes real from noise; identifies systematic deviations & 
Requires pooling across buckets; assumes common structure; less powerful per-sector \\
\bottomrule
\end{tabular}
\end{table}

\textbf{Recommended workflow:}
\begin{enumerate}
\item Run bucket-level analysis to establish aggregate patterns
\item Run within-issuer analysis to validate Merton predictions hold at issuer level
\item Run sector interaction tests to determine if sector-specific adjustments are necessary
\item Synthesize findings across all three approaches for Stage 0 conclusion
\end{enumerate}


\subsection{Bucket-Level Analysis}
\label{sec:stage0-bucket}

\subsubsection{Methodology}

\textbf{Step 1: Define buckets}

Define buckets $k$ by:
\begin{itemize}
\item \textbf{Rating}: AAA/AA, A, BBB for IG; BB, B, CCC for HY
\item \textbf{Maturity}: 1--2y, 2--3y, 3--5y, 5--7y, 7--10y, 10y+
\item \textbf{Sector}: Industrial, Financial, Utility, Energy
\end{itemize}

For each bucket $k$, compute representative characteristics using bonds in that bucket:
\begin{itemize}
\item $\bar{T}_k$: median maturity
\item $\bar{s}_k$: median spread
\item $\lambda^{\text{Merton}}_k = \lambda_T(\bar{T}_k; 5y, \bar{s}_k) \times \lambda_s(\bar{s}_k; 100)$ using Tables~\ref{tab:merton-lambda-T}--\ref{tab:merton-lambda-s}
\end{itemize}

This creates $3 \times 6 \times 4 = 72$ buckets for IG and 72 for HY (some may be sparse or empty).

\textbf{Step 2: Pooled regression by bucket}

For each bucket $k$, estimate using all bond-week observations in that bucket:
\begin{equation}
y_{i,t} = \alpha^{(k)} + \beta^{(k)} \cdot f^{(U)}_{DTS,t} + \varepsilon^{(k)}_{i,t}, \quad i \in \text{bucket } k
\label{eq:bucket-regression}
\end{equation}

where:
\begin{itemize}
\item $y_{i,t} = \Delta OAS_{i,t} / OAS_{i,t-1}$: percentage spread change for bond $i$ in week $t$
\item $f^{(U)}_{DTS,t} = \Delta OAS^{(U)}_t / OAS^{(U)}_{t-1}$: index-level percentage spread change
\item $\beta^{(k)}$: empirical sensitivity for bonds in bucket $k$
\end{itemize}

\textbf{Estimation}: Pooled OLS within each bucket.

\textbf{Standard errors}: Cluster by week to allow arbitrary correlation across bonds within each week.

\textbf{Sample size}: Each major bucket (e.g., BBB 3--5y Industrials) has 5,000--20,000 bond-week observations over 2013--2025, providing strong statistical power.

\textbf{Step 3: Compare empirical to theoretical}

For each bucket $k$, we compare:
\begin{itemize}
\item $\hat{\beta}^{(k)}$: empirical DTS sensitivity from regression
\item $\lambda^{\text{Merton}}_k$: theoretical prediction from Merton model
\end{itemize}

\subsubsection{Statistical tests for bucket-level analysis}

\textbf{Test 1 (Level):} Is $\hat{\beta}^{(k)} \approx \lambda^{\text{Merton}}_k$?

Use $t$-test: $H_0: \beta^{(k)} = \lambda^{\text{Merton}}_k$ with clustered standard errors.

\textbf{Test 2 (Cross-maturity pattern):} Across maturity buckets (holding rating and sector fixed), do empirical betas exhibit predicted monotonicity?

\emph{Merton prediction}: $\hat{\beta}^{(1-2y)} > \hat{\beta}^{(3-5y)} > \hat{\beta}^{(5-7y)} > \hat{\beta}^{(10y+)}$

\emph{Test}: Spearman rank correlation between empirical $\hat{\beta}^{(k)}$ and theoretical $\lambda^{\text{Merton}}_k$ across maturity buckets. Expected strong positive correlation (short bonds more sensitive).

\textbf{Test 3 (Regime pattern):} Does the pattern hold strongly in IG but weaken in HY as theory predicts?

Split sample by spread level:
\begin{itemize}
\item IG spreads ($<$300 bps): Expect large cross-maturity dispersion
\item HY spreads (300--1000 bps): Expect moderate dispersion  
\item Distressed ($>$1000 bps): Expect convergence toward $\beta \approx 1$
\end{itemize}

Test whether cross-maturity standard deviation of $\hat{\beta}^{(k)}$ declines as spreads widen (Regime 5 prediction).

\textbf{Aggregated test across all buckets:}

Pool all buckets and test whether average deviation from Merton is zero:
\begin{equation}
\text{Mean Deviation} = \frac{1}{K} \sum_{k=1}^{K} \left( \hat{\beta}^{(k)} - \lambda^{\text{Merton}}_k \right)
\end{equation}

Test $H_0: \text{Mean Deviation} = 0$ using bootstrap standard errors (resampling weeks).

\textbf{Variation test:}

Does most of the cross-bucket variation in $\hat{\beta}^{(k)}$ come from maturity differences as Merton predicts?

Compute:
\begin{itemize}
\item Total variation: $\text{Var}(\hat{\beta}^{(k)})$ across all buckets
\item Within-rating variation: $\text{Var}(\hat{\beta}^{(k)} | \text{rating})$ holding rating constant
\end{itemize}

If within-rating variation is large and dominated by maturity, this confirms Merton's emphasis on maturity effects.

\subsubsection{Deliverables for bucket-level analysis}

\begin{itemize}
\item \textbf{Table 0.1:} Bucket-level results for key buckets
\begin{itemize}
\item Rows: Rating $\times$ Maturity combinations (e.g., BBB 1--2y, BBB 3--5y, BBB 7--10y)
\item Columns: $\hat{\beta}^{(k)}$ (empirical), $\lambda^{\text{Merton}}_k$ (theory), Ratio, $t$-stat for $H_0: \beta=\lambda$, Sample size
\item Separate panels for IG and HY
\item Highlight cells where ratio outside [0.8, 1.2]
\end{itemize}

\item \textbf{Table 0.2:} Cross-maturity pattern tests
\begin{itemize}
\item For each rating class, show $\hat{\beta}^{(k)}$ across maturity buckets
\item Spearman correlation with theoretical $\lambda^{\text{Merton}}_k$
\item Test whether monotonicity holds (short $>$ medium $>$ long)
\end{itemize}

\item \textbf{Figure 0.1:} Scatter plot: Empirical $\hat{\beta}^{(k)}$ (y-axis) vs Theoretical $\lambda^{\text{Merton}}_k$ (x-axis) for all buckets
\begin{itemize}
\item 45-degree line for perfect agreement
\item Point size proportional to sample size
\item Color-code by spread level: IG (blue), HY (orange), Distressed (red)
\item Annotate outliers
\end{itemize}

\item \textbf{Figure 0.2:} Cross-maturity patterns by rating
\begin{itemize}
\item Separate panels for AAA/AA, A, BBB, BB, B
\item X-axis: Maturity (1y, 3y, 5y, 7y, 10y)
\item Y-axis: $\hat{\beta}$ (empirical, solid line with points) and $\lambda^{\text{Merton}}$ (theoretical, dashed line)
\item Shows whether empirical pattern matches theoretical prediction
\end{itemize}

\item \textbf{Figure 0.3:} Regime patterns
\begin{itemize}
\item X-axis: Average spread level of bucket
\item Y-axis: Cross-maturity dispersion (std dev of $\hat{\beta}^{(k)}$ across maturity buckets)
\item Shows whether dispersion declines as spreads widen (convergence to Regime 5)
\end{itemize}
\end{itemize}


\subsection{Within-Issuer Analysis}
\label{sec:stage0-within-issuer}

\subsubsection{Theoretical motivation}

The Merton (1974) structural model makes its strongest predictions about the \emph{same issuer's} bonds at different maturities. For issuer $i$ with bonds $j=1,\dots,J_i$ outstanding, all bonds share:

\begin{itemize}
\item Same firm value $V_i$
\item Same asset volatility $\sigma_i$
\item Same leverage ratio $L_i = D_i/V_i$
\item Same recovery rate $R_i$
\end{itemize}

The \textbf{only} difference is time to maturity $T_{ij}$.

When a shock hits firm value $\Delta V_i/V_i$, the percentage spread change for bond $j$ is:

\begin{equation}
\frac{\Delta s_{ij}}{s_{ij}} = \varepsilon_{ij} \cdot \frac{\Delta V_i}{V_i} + \text{idiosyncratic noise}
\end{equation}

where $\varepsilon_{ij}$ is the elasticity from equation \eqref{eq:merton-elasticity}.

\textbf{Key prediction}: The ratio of elasticities for two bonds from the same issuer depends \emph{only} on their maturities and spreads:

\begin{equation}
\frac{\varepsilon_{i1}}{\varepsilon_{i2}} = \frac{R_{i1} \cdot T_{i2} \cdot s_{i2}}{R_{i2} \cdot T_{i1} \cdot s_{i1}} \equiv \lambda_{i1,i2}
\label{eq:within-issuer-ratio}
\end{equation}

This is the \textbf{purest test} of Merton because all issuer-specific factors cancel out.

\subsubsection{Empirical specification}

\textbf{Within-issuer, within-week fixed effects}

For each issuer $i$ in week $t$ with $J_{i,t} \geq 2$ bonds outstanding, estimate:

\begin{equation}
\frac{\Delta s_{ij,t}}{s_{ij,t-1}} = \alpha_{i,t} + \beta \cdot \lambda^{\text{Merton}}_{ij,t} + \varepsilon_{ij,t}
\label{eq:within-issuer-fe}
\end{equation}

where:
\begin{itemize}
\item $\alpha_{i,t}$: Issuer-week fixed effect (absorbs common shock $\Delta V_i/V_i$)
\item $\beta$: Coefficient on Merton adjustment factor
\item $\lambda^{\text{Merton}}_{ij,t}$: Predicted elasticity from Merton model for bond $j$ of issuer $i$
\end{itemize}

\textbf{Identification}: Variation comes \emph{entirely} from cross-maturity differences within the same issuer in the same week.

\textbf{Theory prediction}: $\beta = 1$ (Merton correctly predicts elasticity ratios)

\textbf{Null hypothesis}: $H_0: \beta = 1$

\subsubsection{Pooled estimation across issuer-weeks}

After estimating equation \eqref{eq:within-issuer-fe} separately for each issuer-week with sufficient bonds, pool results using inverse-variance weighting:

\begin{equation}
\hat{\beta}_{\text{pooled}} = \frac{\sum_{i,t} w_{i,t} \hat{\beta}_{i,t}}{\sum_{i,t} w_{i,t}}
\end{equation}

where $w_{i,t} = 1/\text{se}(\hat{\beta}_{i,t})^2$.

Pooled standard error:
\begin{equation}
\text{se}(\hat{\beta}_{\text{pooled}}) = \frac{1}{\sqrt{\sum_{i,t} w_{i,t}}}
\end{equation}

Test statistic:
\begin{equation}
t = \frac{\hat{\beta}_{\text{pooled}} - 1}{\text{se}(\hat{\beta}_{\text{pooled}})} \sim t(df)
\end{equation}

where $df$ is the total number of issuer-weeks minus parameters.

\subsubsection{Sample construction}

Using the issuer identification methodology from Section~\ref{sec:issuer-identification}, for each issuer-week $(i,t)$, include if:

\begin{enumerate}
\item $J_{i,t} \geq 2$ (at least two bonds outstanding from same Ultimate Parent + Seniority)
\item All bonds have valid OAS, maturity, and spread data
\item Maturity dispersion $\geq$ 2 years: $\max_j T_{ij} - \min_j T_{ij} \geq 2$
\item No bonds within 1 year of maturity (avoid pull-to-par distortions)
\item Spread changes $|\Delta s_{ij}/s_{ij}| < 200\%$ (exclude extreme outliers)
\end{enumerate}

\textbf{Expected sample size}:

\begin{table}[h]
\centering
\begin{tabular}{lrrr}
\toprule
\textbf{Universe} & \textbf{Issuers with 2+ bonds} & \textbf{Weeks} & \textbf{Issuer-weeks} \\
\midrule
Investment Grade & 800 & 600 & 180,000 \\
High Yield & 250 & 600 & 45,000 \\
\midrule
Total & 1,050 & 600 & 225,000 \\
\bottomrule
\end{tabular}
\caption{Expected sample size for within-issuer analysis}
\label{tab:within-issuer-sample}
\end{table}

This provides substantial statistical power despite smaller samples per issuer-week.

\subsubsection{Concrete example: Illustrating within-issuer vs bucket-level}

Suppose Ford Motor Company has these bonds on March 15, 2020 (during COVID shock):

\begin{table}[h]
\centering
\begin{tabular}{lrrrr}
\toprule
\textbf{Bond} & \textbf{Maturity} & \textbf{Spread} & \textbf{$\lambda^{\text{Merton}}$} & \textbf{$\Delta s/s$} \\
\midrule
Ford 2021 & 1y & 150 bps & 3.27 & +60\% \\
Ford 2025 & 5y & 180 bps & 1.00 & +25\% \\
Ford 2030 & 10y & 200 bps & 0.64 & +18\% \\
\bottomrule
\end{tabular}
\caption{Hypothetical Ford bonds during COVID shock}
\end{table}

\textbf{Within-issuer test}:

\begin{align}
\text{Elasticity ratio (1y/5y)} &= \frac{60\%/150}{25\%/180} = \frac{0.40}{0.139} = 2.88 \\
\text{Merton prediction} &= \frac{\lambda^{\text{Merton}}_{1y}}{\lambda^{\text{Merton}}_{5y}} = \frac{3.27}{1.00} = 3.27 \\
\text{Match} &= \frac{2.88}{3.27} = 0.88 \quad (\text{88\% of predicted})
\end{align}

This is a \textbf{clean test}: same credit event (Ford's deterioration), only maturity differs.

\textbf{Bucket-level test} (what Section~\ref{sec:stage0-bucket} does):

Compare all 1-year BBB bonds to all 5-year BBB bonds, mixing:
\begin{itemize}
\item Ford, GM, Boeing, AT\&T, etc. (different credit stories)
\item Different sectors (Auto, Aerospace, Telecom have different shock exposures)
\item Different liquidity profiles
\item Different idiosyncratic news
\end{itemize}

This is a \textbf{noisier test}: confounds maturity effect with issuer heterogeneity.

\subsubsection{Diagnostic analyses}

\textbf{Heterogeneity by spread level}

Merton predicts elasticity ratios converge toward 1.0 as spreads widen (Regime 5). Test:

\begin{equation}
\hat{\beta}_{i,t} = \gamma_0 + \gamma_s \log(\bar{s}_{i,t}) + u_{i,t}
\end{equation}

where $\bar{s}_{i,t}$ is the average spread across issuer $i$'s bonds in week $t$.

\textbf{Prediction}: $\gamma_s > 0$ (higher spreads $\Rightarrow$ $\beta$ closer to 1)

\textbf{Heterogeneity by maturity dispersion}

When an issuer has very wide maturity dispersion (e.g., 1y and 10y bonds), Merton predictions should be strongest. Test:

\begin{equation}
|\hat{\beta}_{i,t} - 1| = \delta_0 + \delta_{\Delta T} (\max_j T_{ij} - \min_j T_{ij}) + v_{i,t}
\end{equation}

\textbf{Prediction}: $\delta_{\Delta T} < 0$ (wider dispersion $\Rightarrow$ smaller deviation from theory)

\textbf{Intuition}: When issuer has both 1y and 10y bonds, the 4--6$\times$ elasticity difference is stark and harder to miss empirically.

\textbf{Crisis vs normal periods}

Does within-issuer proportionality break down during crises (liquidity evaporation, correlation breakdowns)?

Split sample:
\begin{itemize}
\item Normal: VIX $< 20$
\item Stress: VIX $\in [20, 30]$
\item Crisis: VIX $> 30$
\end{itemize}

Estimate $\hat{\beta}_{\text{pooled}}$ separately for each regime. Test:
\begin{equation}
H_0: \beta_{\text{normal}} = \beta_{\text{crisis}}
\end{equation}

\textbf{Alternative hypothesis}: $\beta_{\text{crisis}} < 1$ (Merton over-predicts front-end sensitivity in crises due to liquidity freezes)

\subsubsection{Deliverables for within-issuer analysis}

\begin{itemize}
\item \textbf{Table 0.3:} Within-issuer pooled results
\begin{itemize}
\item Rows: Full sample, IG only, HY only, Normal (VIX$<$20), Crisis (VIX$>$30)
\item Columns: $\hat{\beta}$, SE, $t$-stat for $H_0: \beta=1$, $p$-value, N (issuer-weeks)
\end{itemize}

\item \textbf{Table 0.4:} Diagnostic regressions
\begin{itemize}
\item Spread level effect: $\hat{\gamma}_s$, SE, interpretation
\item Maturity dispersion effect: $\hat{\delta}_{\Delta T}$, SE, interpretation
\end{itemize}

\item \textbf{Figure 0.4:} Distribution of issuer-week estimates
\begin{itemize}
\item Histogram of $\hat{\beta}_{i,t}$ across all issuer-weeks
\item Vertical line at $\beta = 1$ (theory prediction)
\item Shaded region for pooled estimate $\pm$ 2 SE
\item Annotate median, IQR, fraction within [0.8, 1.2]
\end{itemize}

\item \textbf{Figure 0.5:} Spread level diagnostic
\begin{itemize}
\item Scatter plot: X-axis = average issuer spread $\bar{s}_{i,t}$, Y-axis = $\hat{\beta}_{i,t}$
\item Overlay: LOWESS smooth
\item Prediction: Convergence toward $\beta=1$ as spreads widen
\end{itemize}

\item \textbf{Figure 0.6:} Case studies for major issuers
\begin{itemize}
\item For 5--10 major issuers (e.g., Apple, Ford, AT\&T, JPMorgan)
\item Time series of $\hat{\beta}_{i,t}$ over sample period
\item Identify weeks where $\hat{\beta}_{i,t}$ deviates sharply (crisis episodes?)
\end{itemize}
\end{itemize}


\subsection{Sector Interaction Analysis}
\label{sec:stage0-sector}

\subsubsection{Motivation: Descriptive vs inferential}

The bucket-level approach creates \textbf{separate regressions} for each sector:

\begin{align}
\text{Industrial:} \quad y_{i,t} &= \alpha^{(\text{Ind})} + \beta^{(\text{Ind})} f_{DTS,t} + \varepsilon_{i,t} \\
\text{Financial:} \quad y_{i,t} &= \alpha^{(\text{Fin})} + \beta^{(\text{Fin})} f_{DTS,t} + \varepsilon_{i,t} \\
\text{Utility:} \quad y_{i,t} &= \alpha^{(\text{Util})} + \beta^{(\text{Util})} f_{DTS,t} + \varepsilon_{i,t}
\end{align}

This yields \textbf{point estimates} such as $\hat{\beta}^{(\text{Ind})} = 1.02$, $\hat{\beta}^{(\text{Fin})} = 1.35$, $\hat{\beta}^{(\text{Util})} = 0.98$.

\textbf{Descriptive observation}: ``Financials have higher $\beta$ than Industrials.''

\textbf{Unanswered question}: \emph{Is this difference statistically significant}, or just sampling noise?

The sector interaction specification answers this formally.

\subsubsection{The sector interaction specification}

Pool all sectors together and model differences explicitly:

\begin{equation}
y_{i,t} = \alpha + \beta_0 [\lambda^{\text{Merton}}_{i,t} \cdot f_{DTS,t}] + \sum_{s \in S} \beta_s \cdot \mathbbm{1}_{\{i \in s\}} \cdot [\lambda^{\text{Merton}}_{i,t} \cdot f_{DTS,t}] + \varepsilon_{i,t}
\label{eq:sector-interaction}
\end{equation}

where:
\begin{itemize}
\item $\beta_0$: Base DTS sensitivity (reference sector = Industrial)
\item $\beta_s$: \textbf{Additional} sensitivity for sector $s$ relative to reference
\item $\mathbbm{1}_{\{i \in s\}}$: Indicator = 1 if bond $i$ is in sector $s$
\item $S = \{\text{Financial, Utility, Energy, \dots}\}$: Set of non-reference sectors
\end{itemize}

\textbf{Interpretation}:
\begin{itemize}
\item Total sensitivity for Industrial bonds: $\beta_{\text{Ind}} = \beta_0$
\item Total sensitivity for Financial bonds: $\beta_{\text{Fin}} = \beta_0 + \beta_{\text{Financial}}$
\item Total sensitivity for Utility bonds: $\beta_{\text{Util}} = \beta_0 + \beta_{\text{Utility}}$
\end{itemize}

\subsubsection{Statistical tests}

\textbf{Test 1: Do sectors differ at all?}

Joint null hypothesis:
\begin{equation}
H_0: \beta_{\text{Financial}} = \beta_{\text{Utility}} = \beta_{\text{Energy}} = \cdots = 0
\end{equation}

Use Wald test (F-test):
\begin{equation}
F = \frac{(RSS_{\text{restricted}} - RSS_{\text{unrestricted}})/q}{RSS_{\text{unrestricted}}/df} \sim F(q, df)
\end{equation}

where $q$ is the number of sector dummies.

\textbf{Interpretation}:
\begin{itemize}
\item Reject ($p < 0.05$): Sectors have significantly different DTS sensitivities
\item Fail to reject: Can use common $\beta$ across sectors (parsimony)
\end{itemize}

\textbf{Test 2: Does specific sector differ from reference?}

For Financial sector:
\begin{equation}
H_0: \beta_{\text{Financial}} = 0
\end{equation}

Use $t$-test:
\begin{equation}
t = \frac{\hat{\beta}_{\text{Financial}}}{\text{se}(\hat{\beta}_{\text{Financial}})} \sim t(df)
\end{equation}

\textbf{Test 3: Do two non-reference sectors differ from each other?}

For Financial vs Utility:
\begin{equation}
H_0: \beta_{\text{Financial}} - \beta_{\text{Utility}} = 0
\end{equation}

Use Wald test on linear combination:
\begin{equation}
t = \frac{\hat{\beta}_{\text{Financial}} - \hat{\beta}_{\text{Utility}}}{\sqrt{\text{Var}(\hat{\beta}_{\text{Financial}}) + \text{Var}(\hat{\beta}_{\text{Utility}}) - 2\text{Cov}(\hat{\beta}_{\text{Financial}}, \hat{\beta}_{\text{Utility}})}}
\end{equation}

\subsubsection{Economic rationale for sector differences}

\textbf{Why Financials might differ:}

\begin{enumerate}
\item \textbf{Correlation risk:} Banks and insurance companies have correlated credit risk. When one bank struggles, contagion affects entire financial sector. Merton assumes firm-specific risk, but financials have systemic component.

\emph{Prediction}: $\beta_{\text{Financial}} > 0$ (amplified sensitivity, especially during crises)

\item \textbf{Regulatory capital effects:} Mark-to-market accounting for trading books and regulatory capital requirements force deleveraging when spreads widen, creating mechanical amplification beyond fundamental credit risk.

\item \textbf{Liquidity provision role:} Financials are liquidity providers in credit markets. When their own credit deteriorates, market-wide liquidity dries up, creating feedback loops.

\item \textbf{Balance sheet opacity:} Complex derivatives exposures and off-balance-sheet vehicles create information asymmetry greater than for industrial firms.
\end{enumerate}

\textbf{Why Utilities might differ:}

\begin{enumerate}
\item \textbf{Regulatory protection:} Rate regulation provides cash flow stability. Essential service status limits default risk. Counter-cyclical: demand stable even during recessions.

\emph{Prediction}: $\beta_{\text{Utility}} < 0$ (dampened sensitivity to macro shocks)

\item \textbf{Rate sensitivity:} Long-duration assets (power plants, transmission infrastructure). Credit quality tied to regulatory environment, not economic cycles.

\item \textbf{Capital intensity:} High fixed costs, low variable costs. Leverage ratios higher than industrials, but leverage is accepted by regulators due to stable cash flows.
\end{enumerate}

\textbf{Why Energy might differ:}

\begin{enumerate}
\item \textbf{Commodity price pass-through:} Energy bond spreads driven primarily by oil/gas prices. This is idiosyncratic to energy, not captured by broad DTS factor.

\emph{Prediction}: $\beta_{\text{Energy}}$ lower (idiosyncratic $>$ systematic shocks)

\item \textbf{Boom-bust cycles:} Energy sector subject to extreme cyclicality. Default rates spike during commodity downturns (2015--16, 2020).
\end{enumerate}

\subsubsection{Specification variants}

\textbf{Variant 1: Sector interactions with Merton scaling (preferred)}

As in equation \eqref{eq:sector-interaction}. Tests whether sectors differ in how much they deviate from Merton baseline.

\textbf{Variant 2: Sector interactions without Merton scaling}

\begin{equation}
y_{i,t} = \alpha + \beta_0 f_{DTS,t} + \sum_{s \in S} \beta_s \cdot \mathbbm{1}_{\{i \in s\}} \cdot f_{DTS,t} + \varepsilon_{i,t}
\end{equation}

Tests whether sectors differ in raw DTS sensitivity (ignoring Merton). Use if Stage 0 shows Merton doesn't work well.

\textbf{Variant 3: Rating-specific sector interactions}

Allow sector effects to differ by rating:

\begin{equation}
y_{i,t} = \alpha + \beta_0 [\lambda_i f_{DTS,t}] + \sum_{s \in S} \sum_{r \in R} \beta_{s,r} \cdot \mathbbm{1}_{\{i \in s, i \in r\}} \cdot [\lambda_i f_{DTS,t}] + \varepsilon_{i,t}
\end{equation}

where $R = \{\text{AAA/AA, A, BBB, BB, B, CCC}\}$.

\textbf{Rationale}: Financial IG may differ from Financial HY in systematic ways (e.g., TBTF implicit guarantees for large bank IG bonds).

\textbf{Cost}: Many parameters ($|S| \times |R| \approx 4 \times 6 = 24$ interaction terms).

\textbf{Decision rule}: Only use if Variant 1 shows significant sector effects \emph{and} you have theoretical reason to expect rating heterogeneity.

\subsubsection{Implementation considerations}

\textbf{Clustering of standard errors}

Even in pooled specification, use week clustering:
\begin{itemize}
\item All bonds (across all sectors) in week $t$ experience common shocks
\item Within-week correlation violates independence assumption
\item Cluster by week (not by sector, not by bond) to get correct inference
\end{itemize}

\textbf{Sample balance across sectors}

Potential issue: Unbalanced samples can affect precision of interaction estimates.

\begin{table}[h]
\centering
\begin{tabular}{lrrr}
\toprule
\textbf{Sector} & \textbf{N (bonds)} & \textbf{N (obs)} & \textbf{\% of sample} \\
\midrule
Industrial & 3,500 & 1,200,000 & 60\% \\
Financial & 1,200 & 400,000 & 20\% \\
Utility & 600 & 200,000 & 10\% \\
Energy & 400 & 150,000 & 8\% \\
Other & 300 & 50,000 & 2\% \\
\midrule
Total & 6,000 & 2,000,000 & 100\% \\
\bottomrule
\end{tabular}
\caption{Typical sector distribution in IG universe}
\label{tab:sector-distribution}
\end{table}

\textbf{Implications}:
\begin{itemize}
\item Industrial (reference) has most observations $\Rightarrow$ $\hat{\beta}_0$ most precise
\item Financial has 1/3 the sample $\Rightarrow$ $\hat{\beta}_{\text{Financial}}$ less precise (wider SE)
\item Energy/Other have smallest samples $\Rightarrow$ interaction estimates noisy
\end{itemize}

\textbf{Robustness check}: Re-estimate with balanced subsamples (downsample Industrial to match Financial). If results qualitatively similar, sample imbalance not driving findings.

\textbf{Time-varying sector effects}

Sector differences may be regime-dependent:
\begin{itemize}
\item Financials: Amplified sensitivity during financial crises (2008, 2020), normal otherwise
\item Energy: Amplified sensitivity during oil price crashes (2015--16, 2020), normal otherwise
\item Utilities: Stable across all regimes
\end{itemize}

\textbf{Extended specification}:
\begin{equation}
\beta_s \cdot \mathbbm{1}_{\{i \in s\}} \rightarrow [\beta_{s,0} + \beta_{s,\text{VIX}} \cdot \text{VIX}_t] \cdot \mathbbm{1}_{\{i \in s\}}
\end{equation}

\textbf{Test}: Is $\beta_{\text{Financial},\text{VIX}} > \beta_{\text{Industrial},\text{VIX}}$? (Financials more stress-sensitive)

\subsubsection{Deliverables for sector interaction analysis}

\begin{itemize}
\item \textbf{Table 0.5:} Sector interaction estimates
\begin{itemize}
\item Rows: Industrial (reference), Financial, Utility, Energy, Other
\item Columns: $\hat{\beta}_s$, SE, $t$-stat, $p$-value, Total $\beta$ (= $\beta_0 + \beta_s$)
\item Joint test: $F$-statistic, $p$-value for all $\beta_s = 0$
\end{itemize}

\item \textbf{Table 0.6:} Pairwise sector comparisons
\begin{itemize}
\item Test Financial vs Utility, Financial vs Energy, etc.
\item Report difference, SE of difference, $p$-value
\end{itemize}

\item \textbf{Figure 0.7:} Total $\beta$ by sector
\begin{itemize}
\item Bar chart with sectors on x-axis, total $\beta$ on y-axis
\item Error bars: 95\% confidence intervals
\item Horizontal line at $\beta = 1$ (Merton prediction)
\item Color-code: Blue for reference, Red for significant ($p < 0.05$), Gray for insignificant
\end{itemize}

\item \textbf{Figure 0.8:} Sector deviations from reference
\begin{itemize}
\item Bar chart with non-reference sectors on x-axis, $\beta_s$ on y-axis
\item Error bars: 95\% CI
\item Horizontal line at 0 (no difference from reference)
\item Annotate $p$-values
\end{itemize}

\item \textbf{Figure 0.9:} Regime-dependent sector effects (if time-varying specification estimated)
\begin{itemize}
\item Time series plot: X-axis = Date, Y-axis = Implied sector-specific $\beta$
\item Multiple lines: Industrial, Financial, Utility, Energy
\item Shade crisis periods (2020 COVID, 2022 rate shock)
\end{itemize}
\end{itemize}


\subsection{Synthesis and Integration}
\label{sec:stage0-synthesis}

\subsubsection{Decision tree for Stage 0 conclusion}

After completing all three analyses, synthesize findings using the following framework:

\begin{mdframed}[backgroundcolor=green!5!white,roundcorner=4pt]
\textbf{Stage 0 Synthesis Framework}

\vspace{0.3cm}

\textbf{Question 1: Does Merton work at bucket level?}

From Section~\ref{sec:stage0-bucket}:
\begin{itemize}
\item If median ratio $\hat{\beta}^{(k)}/\lambda^{\text{Merton}}_k \in [0.8, 1.2]$ and $>$80\% of buckets within $\pm$20\%: \textbf{Yes}
\item Otherwise: \textbf{Partially} or \textbf{No}
\end{itemize}

\vspace{0.2cm}

\textbf{Question 2: Does Merton work within issuers?}

From Section~\ref{sec:stage0-within-issuer}:
\begin{itemize}
\item If pooled $\hat{\beta} \in [0.9, 1.1]$ and $p$-value for $H_0: \beta=1$ $>$ 0.10: \textbf{Yes}
\item Otherwise: \textbf{No}
\end{itemize}

\vspace{0.2cm}

\textbf{Question 3: Do sectors differ systematically?}

From Section~\ref{sec:stage0-sector}:
\begin{itemize}
\item If joint $F$-test $p < 0.05$ and $\geq$ 1 sector has $|\hat{\beta}_s| > 0.2$: \textbf{Yes}
\item Otherwise: \textbf{No}
\end{itemize}
\end{mdframed}

\subsubsection{Interpretation matrix}

\begin{table}[h]
\centering
\caption{Stage 0 Synthesis: Interpretation of Combined Results}
\label{tab:stage0-synthesis}
\begin{tabular}{p{2cm}p{2cm}p{2cm}p{7.5cm}}
\toprule
\textbf{Bucket} & \textbf{Within-Issuer} & \textbf{Sectors} & \textbf{Conclusion and Recommendation} \\
\midrule
Yes & Yes & No & Merton works well across the board. Use pure Merton tables for production. Proceed to Stages A--C with high confidence in theoretical baseline. \\
\midrule
Yes & Yes & Yes & Merton works but needs sector adjustments. Use $\lambda^{\text{prod}}_i = \lambda^{\text{Merton}}_i \times (1 + \hat{\beta}_{\text{sector}_i})$. Test sector effect stability in Stage C. \\
\midrule
Yes & No & No & Merton works in aggregate but not within issuers. Cross-issuer heterogeneity exists beyond rating/maturity. Need issuer-specific calibration or finer bucket definitions. \\
\midrule
Partial & Yes & No & Bucket approach too coarse; within-issuer Merton works. Use issuer-level analysis for production. Consider continuous characteristics (Stage A, Spec A.2). \\
\midrule
Partial & Yes & Yes & Merton mechanism correct but bucket aggregation masks issuer and sector heterogeneity. Use within-issuer Merton with sector adjustments. \\
\midrule
No & No & --- & Merton fundamentally fails. Proceed with unrestricted empirical approach in Stages B--C. Investigate alternative mechanisms in Stage D (liquidity, jumps, correlations). \\
\bottomrule
\end{tabular}
\end{table}

\subsubsection{Resolving apparent contradictions}

\textbf{Case 1: Bucket-level Merton fails, but within-issuer succeeds}

\textbf{Scenario}:
\begin{itemize}
\item Bucket-level: Median ratio = 0.75, large dispersion
\item Within-issuer: Pooled $\hat{\beta} = 0.98$, tight distribution
\end{itemize}

\textbf{Diagnosis}: Bucket approach confounds Merton's cross-maturity effect with cross-issuer heterogeneity within buckets. When we isolate pure within-issuer variation, Merton works.

\textbf{Implication}: Problem is not Merton theory but bucket construction. Need finer buckets (e.g., add spread level as dimension) or move to continuous characteristics (Stage A, Specification A.2).

\textbf{Case 2: Both bucket and within-issuer show Merton works, but sectors differ}

\textbf{Scenario}:
\begin{itemize}
\item Bucket-level: Median ratio = 1.05
\item Within-issuer: Pooled $\hat{\beta} = 1.02$
\item Sector interaction: $\hat{\beta}_{\text{Financial}} = 0.35$ ($p < 0.001$)
\end{itemize}

\textbf{Diagnosis}: Merton's cross-maturity predictions hold (both bucket and within-issuer validate), but there's a \textbf{level shift} for certain sectors. Financials move more than Merton predicts, but their cross-maturity \emph{ratios} still obey Merton.

\textbf{Implication}: Use Merton cross-maturity adjustment $\lambda_T$, but add sector multiplier:
\begin{equation}
\lambda^{\text{prod}}_i = \lambda_T(T_i) \times \lambda_s(s_i) \times (1 + \hat{\beta}_{\text{sector}_i})
\end{equation}

This preserves theoretical structure while accommodating sector-specific amplification.

\textbf{Case 3: Within-issuer shows Merton works in IG but fails in HY}

\textbf{Scenario}:
\begin{itemize}
\item Within-issuer IG: $\hat{\beta} = 1.02$ ($p = 0.65$)
\item Within-issuer HY: $\hat{\beta} = 0.75$ ($p = 0.02$)
\end{itemize}

\textbf{Diagnosis}: Merton's continuous diffusion assumption breaks down in HY. Jump-to-default risk, discrete credit events, and liquidity issues dominate. Front-end HY bonds don't amplify as much as Merton predicts.

\textbf{Implication}: Use Merton adjustments for IG only. For HY, either:
\begin{itemize}
\item Use calibrated Merton with $\hat{\beta}_{\text{HY}} = 0.75$ as scaling factor
\item Move to fully empirical approach for HY (Stage B, Specification B.3)
\end{itemize}

\subsubsection{Complementary insights}

Each analysis provides unique information that the others cannot:

\begin{itemize}
\item \textbf{Bucket-level}: Aggregate patterns across market. Most powerful for detecting overall success/failure of Merton. Establishes baseline for Stages A--C.

\item \textbf{Within-issuer}: Validates that Merton's \emph{mechanism} is correct: same issuer, different maturities behave as predicted. Isolates pure maturity effect from other heterogeneity.

\item \textbf{Sector interaction}: Identifies systematic deviations by industry. Reveals where Merton needs augmentation (e.g., Financial amplification) vs where it works universally.
\end{itemize}

\textbf{Combined narrative for Stage 0 report} (example):

\begin{quote}
Bucket-level analysis shows Merton predictions approximately hold in aggregate (median ratio 1.05), with stronger performance in IG than HY. Within-issuer analysis confirms Merton's cross-maturity predictions work cleanly when holding credit quality constant, validating the theoretical mechanism. Sector interaction analysis reveals Financials have 33\% amplified sensitivity beyond Merton baseline, likely due to correlation risk and regulatory capital effects; other sectors do not differ significantly.

\textbf{Synthesis}: Merton provides excellent starting point. Recommend \emph{Calibrated Merton with sector adjustments} for production: Use theoretical $\lambda_T$ and $\lambda_s$, add Financial sector multiplier of 1.33. This combines parsimony (theory-based) with empirical refinement (sector factor).
\end{quote}


\subsection{Combined Deliverables for Stage 0}
\label{sec:stage0-deliverables}

\subsubsection{Tables}

\begin{enumerate}
\item \textbf{Table 0.1}: Bucket-level $\hat{\beta}^{(k)}$ for key rating $\times$ maturity combinations
\item \textbf{Table 0.2}: Cross-maturity pattern tests and Spearman correlations
\item \textbf{Table 0.3}: Within-issuer pooled results by sample split
\item \textbf{Table 0.4}: Within-issuer diagnostic regressions
\item \textbf{Table 0.5}: Sector interaction estimates with joint test
\item \textbf{Table 0.6}: Pairwise sector comparisons
\item \textbf{Table 0.7}: Stage 0 synthesis summary (one-page overview of all three approaches)
\end{enumerate}

\subsubsection{Figures}

\begin{enumerate}
\item \textbf{Figure 0.1}: Scatter of empirical vs theoretical $\beta$ (bucket-level)
\item \textbf{Figure 0.2}: Cross-maturity patterns by rating
\item \textbf{Figure 0.3}: Regime convergence (dispersion vs spread level)
\item \textbf{Figure 0.4}: Distribution of within-issuer $\hat{\beta}_{i,t}$
\item \textbf{Figure 0.5}: Within-issuer $\hat{\beta}$ vs spread level
\item \textbf{Figure 0.6}: Case studies for major issuers
\item \textbf{Figure 0.7}: Total $\beta$ by sector (bar chart)
\item \textbf{Figure 0.8}: Sector deviations from reference
\item \textbf{Figure 0.9}: Regime-dependent sector effects (if applicable)
\item \textbf{Figure 0.10}: Stage 0 synthesis visual (combining key findings from all three approaches)
\end{enumerate}

\subsubsection{Written summary}

The Stage 0 written summary (5--7 pages) should address:

\begin{enumerate}
\item \textbf{Bucket-level findings} (1--2 pages):
\begin{itemize}
\item Does Merton predict bucket-level sensitivities?
\item Is cross-maturity pattern correct?
\item Does pattern differ by spread level (regime convergence)?
\item Where do largest deviations occur?
\end{itemize}

\item \textbf{Within-issuer findings} (1--2 pages):
\begin{itemize}
\item Does Merton work within issuer capital structures?
\item How much heterogeneity exists across issuers?
\item Does spread level matter (Regime 5 test)?
\item Do crises break proportionality?
\end{itemize}

\item \textbf{Sector interaction findings} (1--2 pages):
\begin{itemize}
\item Do sectors differ significantly?
\item Which sectors differ and by how much?
\item Economic interpretation of sector patterns
\item Are differences stable or regime-dependent?
\end{itemize}

\item \textbf{Synthesis and recommendation} (1--2 pages):
\begin{itemize}
\item Reconciliation across the three approaches
\item Overall assessment of Merton adequacy
\item Recommended baseline for Stages A--C
\item Specific adjustments needed (sectors, regimes, IG vs HY)
\end{itemize}
\end{enumerate}


\subsection{Decision Point}
\label{sec:stage0-decision}

Based on Stage 0 results across all three approaches:

\begin{mdframed}[backgroundcolor=yellow!10!white,roundcorner=4pt]
\textbf{Path 1: Theory works well (all three approaches validate)}

\textit{Conditions}: 
\begin{itemize}
\item Bucket-level median ratio $\in [0.9, 1.1]$
\item Within-issuer pooled $\hat{\beta} \in [0.9, 1.1]$ with $p > 0.10$ for $H_0: \beta = 1$
\item Sector joint test $p > 0.10$ (no significant sector differences)
\end{itemize}

\textit{Action}: Proceed to Stages A--C with theory-constrained specifications. Use $\lambda^{\text{Merton}}$ as baseline. High confidence in theoretical foundation.

\vspace{0.3cm}

\textbf{Path 2: Theory works with sector adjustments}

\textit{Conditions}: 
\begin{itemize}
\item Bucket-level and within-issuer validate Merton
\item Sector joint test $p < 0.05$ with $\geq 1$ sector having $|\hat{\beta}_s| > 0.2$
\end{itemize}

\textit{Action}: Adopt $\lambda^{\text{prod}}_i = \lambda^{\text{Merton}}_i \times (1 + \hat{\beta}_{\text{sector}_i})$. Proceed to Stage C to test stability of sector effects over time.

\vspace{0.3cm}

\textbf{Path 3: Theory needs calibration}

\textit{Conditions}: 
\begin{itemize}
\item Within-issuer $\hat{\beta}$ outside $[0.9, 1.1]$ but patterns match (high $R^2$, residuals unsystematic)
\item Bucket-level shows systematic bias (all ratios $> 1.2$ or $< 0.8$)
\end{itemize}

\textit{Action}: Adopt calibrated Merton: $\lambda^{\text{prod}} = \hat{\beta}_{\text{Merton}} \cdot \lambda^{\text{Merton}}$. Theory has right structure, wrong scale. Proceed to Stage C to test stability of calibration coefficient.

\vspace{0.3cm}

\textbf{Path 4: Theory captures structure but misses details}

\textit{Conditions}: 
\begin{itemize}
\item Moderate bucket-level fit (e.g., 60--80\% of buckets within $\pm$20\%)
\item Within-issuer works for IG but not HY
\item Some systematic sector or regime patterns
\end{itemize}

\textit{Action}: Proceed to Stages A--C with both theory-guided and unrestricted tracks in parallel. Use Merton for IG, empirical for HY. Stage C will reveal if time-variation or regime-switching helps.

\vspace{0.3cm}

\textbf{Path 5: Theory fundamentally fails}

\textit{Conditions}: 
\begin{itemize}
\item Wrong patterns (e.g., long bonds more sensitive than short)
\item Within-issuer $\hat{\beta}$ far from 1 with systematic deviations
\item Bucket-level $R^2_{\text{Merton}} < 0.5 \times R^2_{\text{buckets}}$
\end{itemize}

\textit{Action}: Skip theory-constrained specifications. Proceed directly to Stage D (robustness) to diagnose \emph{why} theory fails (liquidity? tails? specific shocks?). Then Stage E with unrestricted specification only.

\textit{Implication}: Report that structural models don't provide adequate guidance for DTS adjustments in this market. Empirical approach necessary.
\end{mdframed}

\textbf{Key principle}: The three-pronged Stage 0 approach provides robust evidence for choosing among these paths. No single analysis is sufficient---bucket-level can be confounded by issuer heterogeneity, within-issuer has smaller samples, and sector interactions require pooling assumptions. Together, they provide a comprehensive diagnostic.


\section{Stage A: Establish Cross-Sectional Variation}
\label{sec:stage-a}

\subsection{Objective}

Stage A establishes the empirical fact that DTS betas differ across bonds \emph{before} testing whether Merton explains why. This separation is critical:

\begin{itemize}
\item If no significant variation exists, stop here—standard DTS is adequate
\item If variation exists but Merton doesn't explain it, we know theory fails
\item If variation exists and Merton explains it, theory provides parsimonious structure
\end{itemize}

\textbf{Relationship to Stage 0}: Stage 0 tested whether Merton predictions hold using three complementary approaches. Stage A takes a step back and asks the more fundamental question: \emph{Is there any systematic variation in DTS betas to explain?} This question is logically prior to testing whether theory explains the variation.

\subsection{Specification A.1: Bucket-level betas}

For each bucket $k$ (defined by rating $\times$ maturity $\times$ sector), estimate:
\begin{equation}
y_{i,t} = \alpha^{(k)} + \beta^{(k)} f^{(U)}_{DTS,t} + \varepsilon^{(k)}_{i,t}, \quad i \in \text{bucket } k
\end{equation}

\textbf{Bucket definitions}:
\begin{itemize}
\item Rating: AAA/AA, A, BBB for IG; BB, B, CCC for HY
\item Maturity: 1--2y, 2--3y, 3--5y, 5--7y, 7--10y, 10y+
\item Sector: Industrials, Financials, Utilities, Energy (can add more)
\end{itemize}

Creates $3 \times 6 \times 4 = 72$ buckets for IG, $3 \times 6 \times 4 = 72$ for HY (some will be sparse).

\textbf{Estimation}: Pooled OLS within each bucket, cluster standard errors by week.

\textbf{Key outputs}:
\begin{itemize}
\item Table A.1: $\hat{\beta}^{(k)}$ for all buckets with standard errors and $t$-statistics
\item $F$-test for equality across buckets: $H_0: \beta^{(1)} = \beta^{(2)} = \cdots = \beta^{(K)}$
\item \textbf{Critical decision}: If $F$-test fails to reject ($p > 0.10$), no significant variation—stop here, use standard DTS
\end{itemize}

\subsection{Specification A.2: Continuous characteristics}

Instead of discrete buckets, estimate how beta varies with continuous characteristics.

\textbf{Two-step procedure}:

\textbf{Step 1}: For each bond $i$, estimate bond-specific beta using rolling 2-year windows:
\begin{equation}
y_{i,t} = \alpha_i + \beta_i f_{DTS,t} + \varepsilon_{i,t}
\end{equation}

Yields time-series of $\hat{\beta}_{i,\tau}$ for bond $i$ at window midpoint $\tau$.

\textbf{Step 2}: Cross-sectional regression of estimated betas on characteristics:
\begin{equation}
\hat{\beta}_{i,\tau} = \gamma_0 + \gamma_M M_{i,\tau} + \gamma_s s_{i,\tau} + \gamma_{M^2} M_{i,\tau}^2 + \gamma_{Ms} M_{i,\tau} \cdot s_{i,\tau} + u_{i,\tau}
\end{equation}

where $M_{i,\tau}$ is maturity, $s_{i,\tau}$ is spread at window midpoint.

\textbf{Standard errors}: Bootstrap or cluster by bond (account for multiple windows per bond).

\textbf{Key outputs}:
\begin{itemize}
\item Coefficient estimates $\hat{\gamma}_M, \hat{\gamma}_s, \hat{\gamma}_{M^2}, \hat{\gamma}_{Ms}$
\item $R^2$ of cross-sectional fit
\item \textbf{Interpretation}: Do betas vary systematically with maturity and spread?
\end{itemize}

\subsection{Establishing the stylized fact}

\textbf{Deliverable}: Document that DTS betas are not constant. Show:
\begin{enumerate}
\item Statistical significance of variation ($F$-test from A.1, $R^2$ from A.2)
\item Economic significance: Range of $\hat{\beta}^{(k)}$ across buckets (e.g., min 0.6, max 1.8 implies 3$\times$ variation)
\item Pattern: Short-maturity / low-spread bonds have higher betas (preliminary observation, not causal claim)
\end{enumerate}

\textbf{Comparison with Stage 0}: Stage A results should be consistent with Stage 0 bucket-level analysis. The key difference is framing:
\begin{itemize}
\item Stage 0 asked: Do empirical betas match Merton predictions?
\item Stage A asks: Do betas vary at all, regardless of whether they match theory?
\end{itemize}

If Stage 0 found Merton works well (Path 1), Stage A should find variation that aligns with Merton predictions. If Stage 0 found Merton partially works or fails, Stage A documents the variation that theory fails to fully explain.

\subsection{Deliverables for Stage A}

\begin{itemize}
\item \textbf{Table A.1:} Bucket-level $\hat{\beta}^{(k)}$ estimates
\begin{itemize}
\item Rows: Maturity buckets
\item Columns: Rating buckets
\item Separate panels for IG and HY (and optionally by sector)
\item Include standard errors, $t$-statistics, sample size per bucket
\end{itemize}

\item \textbf{Table A.2:} Tests of beta equality
\begin{itemize}
\item $F$-test for $H_0:$ all $\beta^{(k)}$ equal, overall and by dimension:
\begin{itemize}
\item Across maturities (holding rating constant)
\item Across ratings (holding maturity constant)
\item Across sectors
\end{itemize}
\item Report $F$-statistic, degrees of freedom, $p$-value
\item \textbf{Decision rule}: If all $p > 0.10$, declare standard DTS adequate
\end{itemize}

\item \textbf{Table A.3:} Continuous characteristic regression (Specification A.2)
\begin{itemize}
\item Coefficients: $\hat{\gamma}_0, \hat{\gamma}_M, \hat{\gamma}_s, \hat{\gamma}_{M^2}, \hat{\gamma}_{Ms}$
\item Standard errors, $t$-statistics, $R^2$
\item Separate for IG and HY
\end{itemize}

\item \textbf{Figure A.1:} Heatmap of $\hat{\beta}^{(k)}$ by maturity (x-axis) $\times$ rating (y-axis). Color intensity represents beta magnitude. Separate panels for IG and HY.

\item \textbf{Figure A.2:} Implied beta surface from Specification A.2: 3D plot or contour plot with maturity on x-axis, spread on y-axis, predicted $\hat{\beta}$ on z-axis/color.

\item \textbf{Diagnostic summary (2 pages):}
\begin{enumerate}
\item Is variation statistically significant? ($F$-test results)
\item Is variation economically meaningful? (Range and IQR of $\hat{\beta}^{(k)}$)
\item What characteristics drive variation? (Maturity vs spread vs sector)
\item Does IG show more variation than HY? (Regime 2 prediction from Stage 0)
\item \textbf{Recommendation}: Proceed to Stage B to test whether Merton explains patterns, or stop if no variation.
\end{enumerate}
\end{itemize}

\subsection{Decision point}

\begin{itemize}
\item \textbf{If $F$-test $p < 0.01$ and $R^2 > 0.15$ in A.2}: Strong evidence of systematic variation. Proceed to Stage B with high confidence.

\item \textbf{If $F$-test $0.01 < p < 0.10$}: Marginal variation. Proceed to Stage B but may find theory sufficient.

\item \textbf{If $F$-test $p > 0.10$ and $R^2 < 0.05$}: No meaningful variation. \textbf{Stop here—standard DTS is adequate. Report this as primary finding.}
\end{itemize}

\textbf{Consistency check with Stage 0}: If Stage 0 found significant patterns (e.g., within-issuer $\hat{\beta} \neq 1$ or significant sector interactions), Stage A should find significant variation. If Stage A finds no variation but Stage 0 found patterns, revisit Stage 0 methodology—possible issue with bucket definitions or sample restrictions.


\section{Stage B: Does Merton Explain the Variation?}
\label{sec:stage-b}

\subsection{Objective}

Having established in Stage A that DTS betas vary across bonds, Stage B tests whether Merton's structural predictions explain this variation. This is the \emph{core empirical test} of the theoretical framework.

\textbf{Critical distinction}: Stage A documented the \emph{fact} of variation. Stage B tests whether \emph{theory explains} it.

\textbf{Relationship to Stage 0}: Stage 0 provided preliminary evidence on whether Merton predictions hold. Stage B conducts more rigorous tests with formal model comparisons and benchmarking against the unrestricted variation documented in Stage A.

\subsection{Specification B.1: Merton as offset (constrained)}

Use Merton-predicted $\lambda^{\text{Merton}}_{i,t}$ as known adjustment factor:
\begin{equation}
y_{i,t} = \alpha + \beta_{\text{Merton}} \cdot [\lambda^{\text{Merton}}_{i,t} \cdot f_{DTS,t}] + \varepsilon_{i,t}
\end{equation}

where $\lambda^{\text{Merton}}_{i,t} = \lambda_T(T_i; 5y, s_{i,t}) \times \lambda_s(s_{i,t}; 100\text{bps})$ using Tables~\ref{tab:merton-lambda-T}--\ref{tab:merton-lambda-s}.

\textbf{Theory prediction}: If Merton is exactly correct, $\beta_{\text{Merton}} = 1$.

\textbf{Test}: Wald test $H_0: \beta = 1$ with clustered standard errors (week $\times$ issuer).

\textbf{Interpretation}:
\begin{itemize}
\item $\hat{\beta}_{\text{Merton}} \in [0.9, 1.1]$: Merton predictions unbiased, theory works
\item $\hat{\beta}_{\text{Merton}} \in [0.8, 1.2]$: Close enough for practical purposes
\item $\hat{\beta}_{\text{Merton}} > 1.2$ or $< 0.8$: Systematic bias, need calibration
\item Low $R^2$ despite $\hat{\beta} \approx 1$: Merton captures mean but misses dispersion
\end{itemize}

\subsection{Specification B.2: Decomposed components}

Test maturity vs credit quality effects separately:
\begin{equation}
y_{i,t} = \alpha + \beta_T [\lambda_T(T_i; 5y, s_{i,t}) \cdot f_{DTS,t}] + \beta_s [\lambda_s(s_{i,t}; 100) \cdot f_{DTS,t}] + \varepsilon_{i,t}
\end{equation}

\textbf{Theory predictions}:
\begin{itemize}
\item $\beta_T \approx 1$: Maturity adjustment works
\item $\beta_s \approx 1$: Credit quality adjustment works
\item If both hold: Merton decomposition empirically valid
\end{itemize}

\textbf{Diagnostic patterns}:
\begin{itemize}
\item $\beta_T \approx 1$ but $\beta_s \neq 1$: Maturity effects correct, quality effects need recalibration
\item $\beta_T \neq 1$ but $\beta_s \approx 1$: Quality effects correct, maturity functional form wrong
\item Both $\neq 1$: Need to reconsider entire Merton structure
\end{itemize}

\subsection{Specification B.3: With sector adjustments}

Building on Stage 0 sector interaction findings, test whether adding sector adjustments to Merton improves fit:

\begin{equation}
y_{i,t} = \alpha + \beta_0 [\lambda^{\text{Merton}}_{i,t} \cdot f_{DTS,t}] + \sum_{s \in S} \beta_s \cdot \mathbbm{1}_{\{i \in s\}} \cdot [\lambda^{\text{Merton}}_{i,t} \cdot f_{DTS,t}] + \varepsilon_{i,t}
\end{equation}

\textbf{Comparison with Stage 0}: This is identical to the sector interaction specification from Stage 0 (equation \ref{eq:sector-interaction}). In Stage B, we use it to assess incremental explanatory power:
\begin{itemize}
\item Does adding sector interactions improve $R^2$ significantly beyond pure Merton?
\item Are sector coefficients stable when estimated jointly with Merton baseline?
\end{itemize}

\subsection{Benchmarking against Stage A}

\textbf{Key comparison}: Does theory-constrained Specification B.1 perform comparably to unrestricted bucket regressions from Stage A?

\textbf{Metrics}:
\begin{enumerate}
\item \textbf{$R^2$ comparison}:
\begin{itemize}
\item $R^2_{\text{buckets}}$ from Stage A (upper bound—fully flexible)
\item $R^2_{\text{Merton}}$ from Specification B.1
\item $R^2_{\text{Merton+Sector}}$ from Specification B.3
\item If $R^2_{\text{Merton}} > 0.9 \times R^2_{\text{buckets}}$: Theory captures 90\%+ of explainable variation
\end{itemize}

\item \textbf{Bucket-level residuals}: For each bucket $k$, compute:
\begin{equation}
\text{Residual}_k = \hat{\beta}^{(k)}_{\text{Stage A}} - \lambda^{\text{Merton}}_k
\end{equation}
If most residuals small (e.g., $<$0.2) and unsystematic, theory adequate.

\item \textbf{RMSE comparison}: Root mean squared error of spread change predictions:
\begin{equation}
\text{RMSE} = \sqrt{\frac{1}{N}\sum_{i,t} (y_{i,t} - \hat{y}_{i,t})^2}
\end{equation}
Compare Merton vs Merton+Sector vs bucket-based predictions.
\end{enumerate}

\subsection{Specification B.4: Unrestricted for comparison}

Estimate fully flexible functional form:
\begin{equation}
\lambda_i = \beta_0 + \beta_M M_i + \beta_{M^2} M_i^2 + \beta_s s_i + \beta_{s^2} s_i^2 + \beta_{Ms} M_i \cdot s_i + \sum_{\text{rating}} \beta_r + \sum_{\text{sector}} \beta_{\text{sec}}
\end{equation}

Then:
\begin{equation}
y_{i,t} = \alpha + [\hat{\lambda}_i \cdot f_{DTS,t}] + \varepsilon_{i,t}
\end{equation}

\textbf{Purpose}: Assess incremental explanatory power beyond theory. If $R^2_{\text{unrestricted}} \gg R^2_{\text{Merton}}$, theory misses important patterns.

\subsection{Theory vs reality table}

\textbf{Critical deliverable}: Direct comparison of empirical betas to Merton predictions from \cite{Wuebben2025}.

\begin{table}[h]
\centering
\caption{Do Empirical Betas Match Merton Predictions? (Illustrative)}
\label{tab:theory-reality}
\begin{tabular}{lccc}
\toprule
\textbf{Bucket} & $\hat{\beta}^{(k)}$ (Stage A) & $\lambda^{\text{Merton}}_k$ (Theory) & Ratio \\
\midrule
1y, 50bps & 3.45 & 3.62 & 0.95 \\
1y, 100bps & 3.15 & 3.27 & 0.96 \\
1y, 200bps & 2.85 & 2.78 & 1.03 \\
\midrule
3y, 50bps & 1.52 & 1.47 & 1.03 \\
3y, 100bps & 1.38 & 1.42 & 0.97 \\
3y, 200bps & 1.29 & 1.36 & 0.95 \\
\midrule
10y, 50bps & 0.58 & 0.61 & 0.95 \\
10y, 100bps & 0.61 & 0.64 & 0.95 \\
10y, 200bps & 0.65 & 0.67 & 0.97 \\
\bottomrule
\end{tabular}
\end{table}

\textbf{Decision criteria}:
\begin{itemize}
\item \textbf{If 90\%+ of ratios in [0.8, 1.2]}: Merton provides excellent baseline. Use pure Merton or calibrated version for production.

\item \textbf{If systematic bias} (all ratios $> 1.2$ or $< 0.8$): Recalibrate with $\beta_{\text{Merton}}$ from Specification B.1. Theory has right structure, wrong scale.

\item \textbf{If high dispersion but no bias}: Heterogeneity beyond Merton dimensions. Proceed to unrestricted estimation, but Merton still useful as starting point.

\item \textbf{If wrong patterns} (e.g., long bonds have higher sensitivity than short): Theory fundamentally fails. Investigate alternative mechanisms.
\end{itemize}

\subsection{Deliverables for Stage B}

\begin{itemize}
\item \textbf{Table B.1:} Constrained Merton specifications
\begin{itemize}
\item Spec B.1: $\hat{\beta}_{\text{Merton}}$, standard error, Wald test $p$-value for $H_0: \beta=1$
\item Spec B.2: $\hat{\beta}_T$, $\hat{\beta}_s$, standard errors, joint test $p$-value for $H_0: (\beta_T, \beta_s) = (1,1)$
\item Spec B.3: $\hat{\beta}_0$, sector coefficients $\hat{\beta}_s$, joint test for sector effects
\item Separate panels for IG and HY
\end{itemize}

\item \textbf{Table B.2:} Model comparison
\begin{itemize}
\item Rows: Stage A buckets, Spec B.1 (Merton), Spec B.2 (decomposed), Spec B.3 (Merton+Sector), Spec B.4 (unrestricted)
\item Columns: $R^2$, RMSE, AIC, number of parameters
\item $\Delta R^2$ relative to Stage A buckets
\end{itemize}

\item \textbf{Table B.3:} Theory vs Reality (as in Table~\ref{tab:theory-reality})
\begin{itemize}
\item All maturity $\times$ spread bucket combinations
\item Empirical $\hat{\beta}^{(k)}$ from Stage A
\item Theoretical $\lambda^{\text{Merton}}_k$
\item Ratio and absolute deviation
\item Highlight cells where $|\text{Ratio} - 1| > 0.25$
\end{itemize}

\item \textbf{Figure B.1:} Scatter plot: Empirical $\hat{\beta}^{(k)}$ (y-axis) vs Theoretical $\lambda^{\text{Merton}}_k$ (x-axis) for all buckets. 45-degree line for perfect agreement. Color-code by:
\begin{itemize}
\item IG-narrow maturity (circles)
\item IG-wide maturity (squares)
\item HY-narrow (triangles)
\item HY-wide (diamonds)
\item Distressed (stars)
\end{itemize}

\item \textbf{Figure B.2:} Residual analysis: $\hat{\beta}^{(k)} - \lambda^{\text{Merton}}_k$ by:
\begin{itemize}
\item Panel A: By maturity (x-axis: 1y, 3y, 5y, 7y, 10y)
\item Panel B: By spread level (x-axis: 50, 100, 200, 300, 500, 1000 bps)
\item Panel C: By sector (x-axis: Industrials, Financials, Utilities, Energy)
\end{itemize}
Zero line indicates perfect Merton prediction. Look for systematic patterns.

\item \textbf{Figure B.3:} Implied $\lambda$ surface from Spec B.4 (unrestricted) vs Merton prediction:
\begin{itemize}
\item 3D surface plot or side-by-side contour plots
\item X-axis: Maturity (1--10 years)
\item Y-axis: Spread (50--1000 bps)
\item Z-axis/color: Predicted $\lambda$
\item Shows where unrestricted deviates from theory
\end{itemize}

\item \textbf{Diagnostic summary (3--4 pages):}
\begin{enumerate}
\item \textbf{Does Merton work?} (Spec B.1 results: $\hat{\beta}_{\text{Merton}}$ and $R^2$)

\item \textbf{Which component drives fit?} (Spec B.2 results: maturity vs quality)

\item \textbf{Do sector adjustments help?} (Spec B.3 results: incremental $R^2$ from sectors)

\item \textbf{Where does theory succeed?} (Table B.3 analysis: which regimes have ratios near 1.0)

\item \textbf{Where does theory fail?} (Residual patterns from Figure B.2)

\item \textbf{Is unrestricted necessary?} (Table B.2 comparison: $\Delta R^2$ and parameter efficiency)

\item \textbf{Practical recommendation}:
\begin{itemize}
\item Use pure Merton tables (simplest)
\item Use calibrated Merton with $\hat{\beta}_{\text{Merton}}$ (simple + data-driven)
\item Use Merton with sector adjustments (theory + sector factors)
\item Need full unrestricted (complex but necessary)
\end{itemize}

\item \textbf{Consistency with Stage 0}: Do Stage B findings align with Stage 0 conclusions? If Stage 0 found significant sector effects, does Spec B.3 confirm their importance?
\end{enumerate}
\end{itemize}

\subsection{Decision point}

Based on Stage B results:

\begin{mdframed}[backgroundcolor=yellow!10!white,roundcorner=4pt]
\textbf{Decision Tree for Stage C Entry:}

\textbf{Path 1: Theory works well}

\textit{Condition}: $\hat{\beta}_{\text{Merton}} \in [0.9, 1.1]$ and $R^2_{\text{Merton}} > 0.85 \times R^2_{\text{buckets}}$

\textit{Action}: Proceed to Stage C to test whether static $\lambda^{\text{Merton}}$ suffices or time-variation needed. High confidence in theoretical foundation.

\textbf{Path 2: Theory works with sector adjustments}

\textit{Condition}: $\hat{\beta}_{\text{Merton}} \in [0.9, 1.1]$ but $R^2_{\text{Merton+Sector}} > R^2_{\text{Merton}} + 0.03$ with significant sector coefficients

\textit{Action}: Adopt Merton with sector adjustments as baseline. Proceed to Stage C to test stability of both Merton coefficient and sector effects over time.

\textbf{Path 3: Theory needs calibration}

\textit{Condition}: $\hat{\beta}_{\text{Merton}}$ outside $[0.9, 1.1]$ but patterns match (high $R^2$, residuals unsystematic)

\textit{Action}: Adopt calibrated Merton: $\lambda^{\text{prod}} = \hat{\beta}_{\text{Merton}} \cdot \lambda^{\text{Merton}}$. Proceed to Stage C to test stability of $\hat{\beta}_{\text{Merton}}$ over time.

\textbf{Path 4: Theory captures structure but misses details}

\textit{Condition}: Moderate $R^2_{\text{Merton}}$ (e.g., 0.6--0.8 $\times$ buckets), some systematic residuals

\textit{Action}: Proceed to Stage C with both theory-guided and unrestricted tracks in parallel. Stage C will reveal if time-variation helps or if static unrestricted better.

\textbf{Path 5: Theory fundamentally fails}

\textit{Condition}: Wrong patterns (e.g., long bonds more sensitive than short), or $R^2_{\text{Merton}} < 0.5 \times R^2_{\text{buckets}}$

\textit{Action}: Skip Stage C (no point testing time-variation of failed model). Proceed directly to Stage D (robustness) to diagnose \emph{why} theory fails (liquidity? tails? specific shocks?). Then Stage E with unrestricted specification only.

\textit{Implication}: Report that structural models don't provide adequate guidance for DTS adjustments in this market. Empirical approach necessary.
\end{mdframed}


\section{Stage C: Does Static Merton Suffice or Do We Need Time-Variation?}
\label{sec:stage-c}

\subsection{Objective and prerequisite}

\textbf{Prerequisite}: Stage B showed that Merton $\lambda(s, T)$ explains cross-sectional variation (Paths 1--4 from Stage B decision tree).

\textbf{Objective}: Test whether the relationship between $\lambda$ and $(s, T)$ is stable over time, or whether macro state variables induce time-variation.

\textbf{Key principle}: Don't add time-variation until you've proven the simple static model fails.

\textbf{Extended objective}: If Stage 0 and Stage B found significant sector effects, also test whether sector adjustments are stable over time or regime-dependent.

\subsection{Rolling window stability test}

Divide sample into non-overlapping 1-year windows $w \in \{1, 2, \dots, W\}$. For each window:

\begin{equation}
y_{i,t} = \alpha_w + \beta_w \cdot [\lambda^{\text{Merton}}_{i,t} \cdot f_{DTS,t}] + \varepsilon_{i,t}, \quad t \in w
\end{equation}

This yields time series of $\hat{\beta}_w$ with standard errors $\text{se}(\hat{\beta}_w)$.

\textbf{Stability test}: Chow test for structural break:
\begin{equation}
H_0: \beta_1 = \beta_2 = \cdots = \beta_W
\end{equation}

Compute $F$-statistic comparing restricted (single $\beta$) vs unrestricted (separate $\beta_w$) models.

\textbf{Decision rule}:
\begin{itemize}
\item \textbf{If $p > 0.10$}: Static $\lambda$ sufficient. \textbf{Stop Stage C here.} No need for time-varying adjustments. Report that Merton provides stable baseline.

\item \textbf{If $0.01 < p < 0.10$}: Marginal instability. Proceed to investigate drivers but be skeptical of over-parameterization.

\item \textbf{If $p < 0.01$}: Significant time-variation. Proceed to macro driver analysis.
\end{itemize}

\subsection{Stability of sector effects}

If Stage B adopted Merton with sector adjustments (Path 2), test whether sector coefficients are stable:

\begin{equation}
y_{i,t} = \alpha_w + \beta_{0,w} [\lambda^{\text{Merton}}_{i,t} \cdot f_{DTS,t}] + \sum_{s \in S} \beta_{s,w} \cdot \mathbbm{1}_{\{i \in s\}} \cdot [\lambda^{\text{Merton}}_{i,t} \cdot f_{DTS,t}] + \varepsilon_{i,t}
\end{equation}

\textbf{Tests}:
\begin{enumerate}
\item Chow test for base coefficient: $H_0: \beta_{0,1} = \beta_{0,2} = \cdots = \beta_{0,W}$
\item Chow test for each sector: $H_0: \beta_{s,1} = \beta_{s,2} = \cdots = \beta_{s,W}$ for each $s$
\item Joint stability test: All coefficients stable across windows
\end{enumerate}

\textbf{Interpretation}:
\begin{itemize}
\item If base $\beta_0$ stable but sector $\beta_s$ vary: Sector effects are regime-dependent (e.g., Financials amplify more during crises)
\item If base $\beta_0$ varies but sector $\beta_s$ stable: Macro conditions affect all bonds similarly; sector differentials constant
\item If both vary: Complex regime structure; may need VIX or spread-level interactions
\end{itemize}

\subsection{Visual stability assessment}

\textbf{Time series plot}: $\hat{\beta}_w$ over time with 95\% confidence bands.

\textbf{Interpretation}:
\begin{itemize}
\item Confidence bands overlapping 1.0 throughout: Static Merton works, bands capture sampling variation
\item Confidence bands tight but $\hat{\beta}_w$ drifts (e.g., 0.9 in 2010s, 1.1 in 2020s): Systematic shift, investigate macro drivers
\item Wide swings during crises (2020, 2022) but stable otherwise: Regime-dependent but static in normal times
\end{itemize}

\subsection{Conditional on instability: Macro driver analysis}

\textbf{Only if Chow test rejects}, estimate second-stage regression:

\begin{equation}
\hat{\beta}_w = \delta_0 + \delta_{\text{VIX}} \cdot \overline{\text{VIX}}_w + \delta_{\text{OAS}} \cdot \log(\overline{\text{OAS}}_{index,w}) + \delta_r \cdot \overline{r}_{10y,w} + \eta_w
\end{equation}

where $\overline{X}_w$ denotes window-average of variable $X$.

\textbf{Theory-based predictions} \cite{Wuebben2025}:
\begin{enumerate}
\item $\delta_{\text{VIX}} > 0$: High volatility amplifies sensitivity, especially for short maturities (flight-to-quality concentrates in long end, front end whipsaws more)

\item $\delta_{\text{OAS}} < 0$: Wide spreads reduce dispersion in $\lambda$ (convergence to Regime 5 where all bonds near default, proportionality improves)

\item $\delta_r$: Ambiguous. Higher rates increase discount effect (reduce duration), possibly dampening elasticities. Empirical question.
\end{enumerate}

\textbf{Economic significance threshold}: Only declare time-variation meaningful if macro state changes $\lambda$ by $>$20\% over sample range.

Example: If $\delta_{\text{VIX}} = 0.01$ and VIX ranges from 10 to 40, the effect is $0.01 \times 30 = 0.30$, or 30\% change in $\beta$. This is economically large.

If $\delta_{\text{VIX}} = 0.002$, effect is 6\%—within noise, ignore.

\subsection{Maturity-specific time-variation}

Theory predicts time-variation should differ by maturity: short-maturity IG bonds most affected.

\textbf{Test}: Estimate rolling $\beta_w$ separately for maturity buckets:
\begin{equation}
y_{i,t} = \alpha_{w,m} + \beta_{w,m} \cdot [\lambda^{\text{Merton}}_{i,t} \cdot f_{DTS,t}] + \varepsilon_{i,t}, \quad i \in \text{maturity bucket } m, \, t \in w
\end{equation}

Then regress:
\begin{equation}
\hat{\beta}_{w,m} = \delta_{0,m} + \delta_{\text{VIX},m} \cdot \overline{\text{VIX}}_w + \eta_{w,m}
\end{equation}

\textbf{Prediction}: $\delta_{\text{VIX},1y} > \delta_{\text{VIX},5y} > \delta_{\text{VIX},10y}$ (short bonds more regime-dependent).

If this pattern holds, supports theory-based intuition about crisis dynamics.

\subsection{Sector-specific time-variation}

Building on Stage 0 sector findings, test whether sector effects are regime-dependent:

\begin{equation}
\hat{\beta}_{s,w} = \delta_{s,0} + \delta_{s,\text{VIX}} \cdot \overline{\text{VIX}}_w + \eta_{s,w}
\end{equation}

\textbf{Predictions}:
\begin{itemize}
\item $\delta_{\text{Financial},\text{VIX}} > 0$: Financial sector amplification increases during stress (correlation risk, liquidity spirals)
\item $\delta_{\text{Utility},\text{VIX}} \approx 0$: Utility defensive characteristics stable across regimes
\item $\delta_{\text{Energy},\text{VIX}}$: Ambiguous—energy may decouple from broad market during commodity-specific shocks
\end{itemize}

\subsection{Practical implication assessment}

\textbf{Question}: Even if time-variation is statistically significant, does it matter for portfolio management?

\textbf{Scenario analysis}: Compare static vs time-varying $\lambda$ for:
\begin{enumerate}
\item \textbf{Risk model accuracy}: Does time-varying $\lambda$ reduce tracking error in out-of-sample hedging?

\item \textbf{Crisis performance}: During 2020 COVID shock, did static $\lambda$ severely misprice front-end IG?

\item \textbf{Operational complexity}: Time-varying $\lambda$ requires daily macro state inputs and recalibration. Worth the cost?
\end{enumerate}

\textbf{Recommendation framework}:
\begin{itemize}
\item If time-variation changes risk estimates by $<$10\% except during rare crises: Use static $\lambda$, add crisis overlays manually

\item If time-variation changes risk estimates by $>$20\% routinely: Implement time-varying $\lambda$ with macro state

\item If time-variation important only for specific buckets (e.g., 1--2y IG): Use static for most bonds, time-varying for front end only

\item If sector effects are regime-dependent: Use static sector adjustments with VIX-contingent overlays for Financials
\end{itemize}

\subsection{Deliverables for Stage C}

\begin{itemize}
\item \textbf{Table C.1:} Rolling window stability test
\begin{itemize}
\item Rows: Time windows (2013--2014, 2014--2015, ..., 2024--2025)
\item Columns: $\hat{\beta}_w$, standard error, 95\% CI, sample size
\item Separate panels for IG and HY
\item Chow test: $F$-statistic, $p$-value
\end{itemize}

\item \textbf{Table C.2:} Sector coefficient stability (if applicable)
\begin{itemize}
\item Rows: Windows
\item Columns: $\hat{\beta}_{0,w}$, $\hat{\beta}_{\text{Financial},w}$, $\hat{\beta}_{\text{Utility},w}$, $\hat{\beta}_{\text{Energy},w}$
\item Chow tests for each coefficient series
\end{itemize}

\item \textbf{Table C.3:} Macro driver regression (conditional on instability)
\begin{itemize}
\item Coefficients: $\hat{\delta}_{\text{VIX}}$, $\hat{\delta}_{\text{OAS}}$, $\hat{\delta}_r$
\item Standard errors, $t$-statistics, $R^2$
\item Economic significance: Effect of 1 SD change in each macro variable on $\beta$
\item Test predicted signs: $\delta_{\text{VIX}} > 0$, $\delta_{\text{OAS}} < 0$
\end{itemize}

\item \textbf{Table C.4:} Maturity-specific time-variation
\begin{itemize}
\item Rows: Maturity buckets (1--2y, 3--5y, 7--10y)
\item Columns: $\hat{\delta}_{\text{VIX},m}$, standard error, $t$-statistic
\item Test: Is $\delta_{\text{VIX},1y}$ significantly larger than $\delta_{\text{VIX},10y}$?
\end{itemize}

\item \textbf{Table C.5:} Sector-specific time-variation (if applicable)
\begin{itemize}
\item Rows: Sectors (Financial, Utility, Energy)
\item Columns: $\hat{\delta}_{s,\text{VIX}}$, standard error, $t$-statistic
\item Test: Is $\delta_{\text{Financial},\text{VIX}} > 0$?
\end{itemize}

\item \textbf{Figure C.1:} Time series of $\hat{\beta}_w$ for IG and HY
\begin{itemize}
\item X-axis: Year (2013--2025)
\item Y-axis: $\hat{\beta}_w$
\item Point estimates with 95\% confidence bands
\item Horizontal line at $\beta = 1$ (theory prediction)
\item Shade crisis periods (2020 COVID, 2022 rate shock)
\item Interpretation: Does $\hat{\beta}_w$ spike during crises?
\end{itemize}

\item \textbf{Figure C.2:} $\hat{\beta}_w$ vs macro state variables
\begin{itemize}
\item Panel A: $\hat{\beta}_w$ (y-axis) vs $\overline{\text{VIX}}_w$ (x-axis)
\item Panel B: $\hat{\beta}_w$ (y-axis) vs $\log(\overline{\text{OAS}}_w)$ (x-axis)
\item Scatter with OLS fit line
\item Color-code by time period (pre-2020, COVID, post-COVID)
\item Shows whether macro variables predict time-variation
\end{itemize}

\item \textbf{Figure C.3:} Implied $\lambda_{i,t}$ for representative bonds over time
\begin{itemize}
\item Three lines: 1-year BBB, 5-year BBB, 10-year BBB (all industrial)
\item Static $\lambda$ (dashed) vs time-varying $\lambda_t$ (solid)
\item Shows when and how much time-variation matters
\end{itemize}

\item \textbf{Figure C.4:} Sector coefficient stability (if applicable)
\begin{itemize}
\item Time series of $\hat{\beta}_{\text{Financial},w}$, $\hat{\beta}_{\text{Utility},w}$, $\hat{\beta}_{\text{Energy},w}$
\item Shows whether sector effects are stable or regime-dependent
\end{itemize}

\item \textbf{Figure C.5:} Scenario analysis—crisis vs normal
\begin{itemize}
\item Histogram of spread changes during normal periods (VIX $<$ 20)
\item Histogram of spread changes during stress (VIX $>$ 30)
\item Overlay: Static Merton prediction, time-varying prediction
\item Shows whether static model fails systematically in crises
\end{itemize}

\item \textbf{Summary and recommendation (3--4 pages):}
\begin{enumerate}
\item \textbf{Is base relationship stable?} (Chow test results for $\beta_w$)

\item \textbf{Are sector effects stable?} (Chow test results for $\beta_{s,w}$)

\item \textbf{If unstable, what drives it?} (Macro driver analysis)

\item \textbf{Is instability economically meaningful?} (Effect size in \% terms)

\item \textbf{Does theory-based intuition hold?} (VIX amplifies front-end, OAS compresses dispersion)

\item \textbf{Practical recommendation}:
\begin{itemize}
\item Use static $\lambda$: Adequate for normal markets, simple implementation
\item Use time-varying $\lambda$: Necessary for crisis periods, worth complexity
\item Hybrid: Static baseline with crisis adjustments (VIX $>$ 30)
\item For sectors: Static sector adjustments vs regime-dependent
\end{itemize}

\item \textbf{Implication for production}: If static suffices, Stage E will select among pure/calibrated Merton with or without sector adjustments. If time-varying needed, add macro state to production spec.
\end{enumerate}
\end{itemize}

\subsection{Decision point}

\begin{itemize}
\item \textbf{If Chow test $p > 0.10$ for both base and sectors}: Static $\lambda$ sufficient. Proceed to Stage D (robustness) with confidence in stable baseline.

\item \textbf{If Chow test $p < 0.10$ for base but sectors stable}: Time-variation in overall level but not in sector differentials. Consider VIX overlay on base $\beta_0$.

\item \textbf{If Chow test $p > 0.10$ for base but sectors unstable}: Stable overall but regime-dependent sector effects. Use static base with VIX-contingent sector adjustments.

\item \textbf{If Chow test $p < 0.01$ and effects $>$20\% in crises}: Time-varying $\lambda$ necessary. Incorporate macro state in Stage E production specification.
\end{itemize}


\section{Stage D: Robustness and Extensions}
\label{sec:stage-d}

\subsection{Objective and positioning}

\textbf{Prerequisite}: Stages A--C established whether/how Merton predictions hold for \emph{mean} spread changes in standard conditions.

\textbf{Objective}: Test robustness across:
\begin{enumerate}
\item Tail events (quantile regression)
\item Shock types (systematic vs idiosyncratic decomposition)
\item Spread components (default vs liquidity)
\end{enumerate}

\textbf{Key framing}: These are \textbf{secondary} tests. If Stages A--C show Merton fails, Stage D helps diagnose \emph{why}. If Stages A--C validate Merton, Stage D confirms it's not just a mean effect.

\textbf{Relationship to Stage 0}: Stage D extends the robustness checks beyond what Stage 0's three-pronged approach covers. While Stage 0 tested Merton in aggregate (buckets), within issuers, and across sectors, Stage D examines distributional properties (tails), shock decomposition, and spread component separation.

\subsection{D.1: Tail behavior (quantile regression)}

\subsubsection{Motivation}

Merton model assumes normal shocks (geometric Brownian motion for firm value). If tails differ from mean, this suggests:
\begin{itemize}
\item Jump-to-default risk not captured by continuous diffusion
\item Liquidity evaporation in stress (left tail)
\item Asymmetric investor behavior (panic selling vs gradual buying)
\end{itemize}

\subsubsection{Specification}

For quantiles $\tau \in \{0.05, 0.10, 0.25, 0.50, 0.75, 0.90, 0.95\}$, estimate:

\begin{equation}
Q_{\tau}(y_{i,t} \mid f_{DTS,t}, \lambda^{\text{Merton}}_i) = \alpha_{\tau} + \beta_{\tau} \cdot [\lambda^{\text{Merton}}_i \cdot f_{DTS,t}]
\end{equation}

This models the $\tau$-th conditional quantile of spread changes, allowing elasticity to differ across the distribution.

\textbf{Merton prediction}: $\beta_{\tau} \approx 1$ for all $\tau$ (all quantiles respect structural elasticities).

\textbf{Diagnostics}:
\begin{itemize}
\item If $\beta_{0.05} \gg \beta_{0.95}$: Left tail (spread widening) has amplified sensitivity. Consistent with jump-to-default or liquidity spirals.

\item If $\beta_{0.95} > \beta_{0.05}$: Right tail (spread tightening) more sensitive. Suggests momentum/technical buying in rallies.

\item If $\beta_{\tau}$ U-shaped (high at both tails): Both extreme moves behave differently than moderate moves—non-linearity in spread dynamics.
\end{itemize}

\subsubsection{Sector-specific tail behavior}

Building on Stage 0 sector findings, estimate quantile regressions separately by sector:

\begin{equation}
Q_{\tau}(y_{i,t} \mid f_{DTS,t}, \lambda^{\text{Merton}}_i, \text{sector}_i = s) = \alpha_{\tau,s} + \beta_{\tau,s} \cdot [\lambda^{\text{Merton}}_i \cdot f_{DTS,t}]
\end{equation}

\textbf{Hypothesis}: If Stage 0 found Financials have amplified mean sensitivity, do they also have amplified tail sensitivity?
\begin{itemize}
\item $\beta_{0.05,\text{Financial}} > \beta_{0.05,\text{Industrial}}$: Financial left-tail risk exceeds mean amplification (correlation risk manifests in tails)
\item $\beta_{0.05,\text{Utility}} \approx \beta_{0.50,\text{Utility}}$: Utility defensive characteristics hold in tails
\end{itemize}

\subsubsection{Practical implications}

\textbf{Risk management}:
\begin{itemize}
\item Use $\beta_{0.05}$ for Value-at-Risk (VaR) and Expected Shortfall (ES) calculations
\item Use $\beta_{0.50}$ for expected return / attribution models
\item If $\beta_{0.05} = 1.5 \times \beta_{0.50}$: Tail risk 50\% larger than mean-based models predict
\end{itemize}

\textbf{Stress testing}:
\begin{itemize}
\item Standard Merton $\lambda$ may underestimate losses in left-tail scenarios
\item Adjust: $DTS^{*,\text{stress}}_{i,t} = \beta_{0.05} \cdot \lambda^{\text{Merton}}_i \cdot DTS_{i,t}$
\end{itemize}

\subsubsection{Deliverables for D.1}

\begin{itemize}
\item \textbf{Table D.1:} Quantile-specific $\beta_{\tau}$ estimates
\begin{itemize}
\item Rows: $\tau \in \{0.05, 0.10, 0.25, 0.50, 0.75, 0.90, 0.95\}$
\item Columns: $\hat{\beta}_{\tau}$, standard error, 95\% CI
\item Separate panels for IG and HY
\item Test: $H_0: \beta_{0.05} = \beta_{0.50}$ and $H_0: \beta_{0.95} = \beta_{0.50}$
\end{itemize}

\item \textbf{Table D.2:} Sector-specific tail behavior
\begin{itemize}
\item Rows: Sectors (Industrial, Financial, Utility, Energy)
\item Columns: $\beta_{0.05}$, $\beta_{0.50}$, $\beta_{0.95}$, Ratio $\beta_{0.05}/\beta_{0.50}$
\item Shows which sectors have amplified tail risk
\end{itemize}

\item \textbf{Figure D.1:} Plot of $\hat{\beta}_{\tau}$ across $\tau \in [0.05, 0.95]$
\begin{itemize}
\item X-axis: Quantile $\tau$
\item Y-axis: $\hat{\beta}_{\tau}$
\item Horizontal line at $\beta = 1$ (Merton prediction)
\item Confidence bands (bootstrap)
\item Interpretation: Flat line = Merton works across distribution. Upward/downward slope = asymmetry.
\end{itemize}

\item \textbf{Figure D.2:} Sector-specific quantile plots
\begin{itemize}
\item Separate lines for Industrial, Financial, Utility, Energy
\item Shows whether Financial tail amplification exceeds mean amplification
\end{itemize}

\item \textbf{Interpretation note}: Are tail deviations concentrated in specific regimes (e.g., front-end IG)? If so, suggests these bonds have additional jump risk beyond Merton's continuous framework.
\end{itemize}

\subsection{D.2: Shock decomposition}

\subsubsection{Motivation}

Merton model treats all firm value shocks identically—whether macro, sector, or idiosyncratic, the elasticity $\lambda_i$ should be the same because all operate through the firm's asset value.

Empirically test: Do different shock types exhibit different elasticities? If so, suggests mechanisms beyond firm fundamentals (e.g., liquidity contagion for sector shocks, information asymmetry for idiosyncratic shocks).

\subsubsection{Factor construction}

Decompose bond $i$'s spread change into orthogonal components:

\begin{align}
y_{i,t} &= \underbrace{f^{(G)}_{DTS,t}}_{\text{Global factor}} + \underbrace{f^{(S)}_{DTS,s(i),t}}_{\text{Sector factor}} + \underbrace{f^{(I)}_{DTS,i,t}}_{\text{Issuer-specific}} + \varepsilon_{i,t}
\end{align}

\textbf{Estimation procedure}:

\textbf{Step 1}: Global factor
\begin{equation}
f^{(G)}_{DTS,t} = \frac{\Delta OAS^{(U)}_t}{OAS^{(U)}_{t-1}}
\end{equation}

\textbf{Step 2}: Sector factors (orthogonalized to global)
\begin{equation}
f^{(S)}_{DTS,s,t} = \frac{\Delta OAS^{(U)}_{s,t}}{OAS^{(U)}_{s,t-1}} - f^{(G)}_{DTS,t}
\end{equation}

\textbf{Step 3}: Issuer-specific (residual)
\begin{equation}
f^{(I)}_{DTS,i,t} = y_{i,t} - f^{(G)}_{DTS,t} - f^{(S)}_{DTS,s(i),t}
\end{equation}

\subsubsection{Multi-factor regression with Merton baseline}

Estimate:
\begin{equation}
y_{i,t} = \beta^{(G)} [\lambda^{\text{Merton}}_i \cdot f^{(G)}_{DTS,t}] + \beta^{(S)} [\lambda^{\text{Merton}}_i \cdot f^{(S)}_{DTS,s(i),t}] + \beta^{(I)} [\lambda^{\text{Merton}}_i \cdot f^{(I)}_{DTS,i,t}] + \varepsilon_{i,t}
\end{equation}

\textbf{Constrained specification}: Impose $\beta^{(G)} = \beta^{(S)} = \beta^{(I)} = 1$, estimate single coefficient:
\begin{equation}
y_{i,t} = \beta \cdot \lambda^{\text{Merton}}_i \cdot [f^{(G)}_t + f^{(S)}_{s(i),t} + f^{(I)}_{i,t}] + \varepsilon_{i,t}
\end{equation}

\textbf{Merton prediction}: $\beta^{(G)} \approx \beta^{(S)} \approx \beta^{(I)} \approx 1$ (all shocks respect structural elasticities).

\subsubsection{Diagnostic patterns}

\begin{itemize}
\item \textbf{If $\beta^{(G)} \approx \beta^{(S)} \approx \beta^{(I)} \approx 1$}: Merton applies uniformly—all shocks operate through firm value.

\item \textbf{If $\beta^{(S)} > \beta^{(G)}$}: Sector shocks have amplified effects. Suggests contagion, correlation trading, or common liquidity factors beyond fundamentals.

\item \textbf{If $\beta^{(I)} \gg \beta^{(G)}$}: Idiosyncratic news has exaggerated spread impact. Consistent with information asymmetry, adverse selection in trading.

\item \textbf{If $\beta^{(G)} < 1$ but $\beta^{(I)} > 1$}: Bonds under-react to macro (diversified portfolios stabilize) but over-react to issuer-specific (concentrated positions, forced selling).
\end{itemize}

\subsubsection{Connection to Stage 0 sector findings}

If Stage 0 found Financials have amplified sensitivity, test whether this comes from:
\begin{itemize}
\item Global shocks: $\beta^{(G)}_{\text{Financial}} > \beta^{(G)}_{\text{Industrial}}$ (Financials more sensitive to macro)
\item Sector shocks: $\beta^{(S)}_{\text{Financial}} > \beta^{(S)}_{\text{Industrial}}$ (correlation/contagion within Financial sector)
\item Idiosyncratic shocks: $\beta^{(I)}_{\text{Financial}} > \beta^{(I)}_{\text{Industrial}}$ (information asymmetry in Financial credits)
\end{itemize}

This decomposition helps understand \emph{why} sector effects exist, informing production model design.

\subsubsection{Deliverables for D.2}

\begin{itemize}
\item \textbf{Table D.3:} Variance decomposition
\begin{itemize}
\item Rows: Rating $\times$ maturity buckets
\item Columns: \% variance from Global, Sector, Issuer-specific, Residual
\item Shows relative importance of each factor type
\item Separate for IG and HY (expect IG more global-driven, HY more issuer-specific)
\end{itemize}

\item \textbf{Table D.4:} Shock-specific elasticities
\begin{itemize}
\item Rows: $\beta^{(G)}$, $\beta^{(S)}$, $\beta^{(I)}$
\item Columns: Estimate, standard error, 95\% CI
\item Test: $H_0: \beta^{(G)} = \beta^{(S)} = \beta^{(I)} = 1$ (joint test)
\item Test: Pairwise $H_0: \beta^{(G)} = \beta^{(S)}$, etc.
\end{itemize}

\item \textbf{Table D.5:} Sector-specific shock decomposition
\begin{itemize}
\item Rows: Sectors (Industrial, Financial, Utility, Energy)
\item Columns: $\beta^{(G)}_s$, $\beta^{(S)}_s$, $\beta^{(I)}_s$
\item Shows which shock types drive each sector's behavior
\end{itemize}

\item \textbf{Figure D.3:} Bar chart of $\hat{\beta}$ by factor type
\begin{itemize}
\item Three bars: Global, Sector, Issuer-specific
\item Error bars for 95\% CI
\item Horizontal line at $\beta = 1$
\item Separate panels for IG and HY
\end{itemize}

\item \textbf{Interpretation}: If all three $\approx$ 1, Merton universally applicable. If sector/issuer deviate, need factor-specific adjustments in production models.
\end{itemize}

\subsection{D.3: Liquidity adjustment}

\subsubsection{Motivation}

OAS includes both default risk and liquidity premium. Merton model predicts elasticity for \emph{default component only}. If liquidity shocks don't respect structural elasticities, need to decompose.

\subsubsection{Liquidity-adjusted spread construction}

\textbf{Step 1}: Estimate liquidity component cross-sectionally each period:
\begin{equation}
s^{\text{liq}}_{i,t} = \phi_0 + \phi_1 \text{BidAsk}_{i,t} + \phi_2 \log(\text{Size}_i) + \phi_3 \log(\text{Turnover}_{i,t}) + \phi_4 \text{Age}_{i,t} + \eta_{i,t}
\end{equation}

Standard approach: Cross-sectional regression within each rating $\times$ maturity cell to avoid confounding credit quality with liquidity.

\textbf{Step 2}: Define default component as residual:
\begin{equation}
OAS^{\text{def}}_{i,t} = OAS_{i,t} - \widehat{OAS}^{\text{liq}}_{i,t}
\end{equation}

\textbf{Step 3}: Compute default-based spread changes:
\begin{equation}
y^{\text{def}}_{i,t} = \frac{\Delta OAS^{\text{def}}_{i,t}}{OAS^{\text{def}}_{i,t-1}}
\end{equation}

\subsubsection{Re-estimate Merton fit on default component}

Run Stage B regression using $y^{\text{def}}_{i,t}$ as dependent variable:
\begin{equation}
y^{\text{def}}_{i,t} = \alpha + \beta_{\text{def}} \cdot [\lambda^{\text{Merton}}_i \cdot f^{\text{def}}_{DTS,t}] + \varepsilon_{i,t}
\end{equation}

where $f^{\text{def}}_{DTS,t}$ is the index-level default spread factor (after removing liquidity).

\textbf{Theory prediction}: $\beta_{\text{def}} \approx 1$ (Merton works for default component) and $R^2_{\text{def}} > R^2_{\text{total}}$ (less noise without liquidity shocks).

\subsubsection{Diagnostic patterns}

\begin{itemize}
\item \textbf{If $\beta_{\text{def}} \approx 1$ and $\beta_{\text{total}} \approx 1$}: Liquidity adjustment unnecessary—Merton works on total OAS.

\item \textbf{If $\beta_{\text{def}} \approx 1$ but $\beta_{\text{total}} < 1$}: Liquidity shocks dampen spread movements. Total OAS underestimates default sensitivity.

\item \textbf{If $\beta_{\text{def}} < \beta_{\text{total}}$}: Liquidity shocks amplify spread movements beyond Merton. Need separate liquidity beta.

\item \textbf{If improvement concentrated in illiquid bonds}: Signals that liquidity noise matters for small size, low turnover bonds but not for liquid benchmarks.
\end{itemize}

\subsubsection{Sector-specific liquidity effects}

Building on Stage 0 findings, test whether liquidity adjustment differentially affects sectors:
\begin{itemize}
\item Financials: Often more liquid (active trading, market makers); liquidity adjustment may be less important
\item Utilities: Often less liquid (buy-and-hold investors); liquidity adjustment may help
\item Energy: Liquidity varies with oil price cycles; may need regime-dependent liquidity treatment
\end{itemize}

\subsubsection{Practical decision rule}

\begin{mdframed}[backgroundcolor=blue!5!white,roundcorner=4pt]
\textbf{When to Use Liquidity Adjustment:}

\textbf{Liquid IG (BidAsk $<$ 50bps, Size $>$ \$1B):} 
\begin{itemize}
\item Use total OAS, liquidity adjustment negligible ($<$5bps typical)
\item Liquidity $R^2$ low, not worth complexity
\end{itemize}

\textbf{Illiquid IG (BidAsk $>$ 100bps, Size $<$ \$500M):}
\begin{itemize}
\item Liquidity premium 20--50bps, material fraction of spread
\item Consider decomposition for risk models
\end{itemize}

\textbf{HY (BidAsk typically $>$ 200bps):}
\begin{itemize}
\item Decompose into $OAS^{\text{def}}$ and $OAS^{\text{liq}}$
\item Use $\lambda^{\text{def}}$ from Merton for default component
\item Add separate $\lambda^{\text{liq}}$ empirically estimated (not theory-based)
\end{itemize}
\end{mdframed}

\subsubsection{Deliverables for D.3}

\begin{itemize}
\item \textbf{Table D.6:} Liquidity model estimates
\begin{itemize}
\item Cross-sectional regression: $\hat{\phi}_1$ (bid-ask), $\hat{\phi}_2$ (size), $\hat{\phi}_3$ (turnover), $\hat{\phi}_4$ (age)
\item $R^2$ by rating $\times$ maturity bucket
\item Typical liquidity premium: Mean $\widehat{OAS}^{\text{liq}}$ by bucket
\end{itemize}

\item \textbf{Table D.7:} Merton fit comparison
\begin{itemize}
\item Rows: Total OAS, Default component only
\item Columns: $\hat{\beta}$, $R^2$, RMSE
\item Test: Is $\beta_{\text{def}}$ closer to 1 than $\beta_{\text{total}}$?
\item Improvement: $\Delta R^2 = R^2_{\text{def}} - R^2_{\text{total}}$
\end{itemize}

\item \textbf{Table D.8:} Improvement by liquidity regime
\begin{itemize}
\item Split sample by liquidity quartiles (bid-ask or turnover)
\item For each quartile: $\beta_{\text{total}}$, $\beta_{\text{def}}$, $\Delta R^2$
\item Shows whether adjustment helps mainly for illiquid bonds
\end{itemize}

\item \textbf{Table D.9:} Sector-specific liquidity effects
\begin{itemize}
\item Rows: Sectors
\item Columns: $\beta_{\text{total}}$, $\beta_{\text{def}}$, $\Delta R^2$
\item Shows which sectors benefit most from liquidity decomposition
\end{itemize}

\item \textbf{Figure D.4:} Scatter plot: $\beta_{\text{def}} - \beta_{\text{total}}$ (y-axis) vs average bid-ask (x-axis)
\begin{itemize}
\item Each point: a rating $\times$ maturity bucket
\item Interpretation: Positive slope = illiquid bonds benefit more from decomposition
\end{itemize}

\item \textbf{Interpretation and recommendation}:
\begin{itemize}
\item For which bond types is liquidity adjustment material?
\item Does Merton work better on default component than total OAS?
\item Production implication: Use total OAS for liquid IG, decompose for HY?
\end{itemize}
\end{itemize}

\subsection{Summary of Stage D}

Stage D is \textbf{diagnostic}, not \textbf{decisional}. It answers:

\begin{enumerate}
\item \textbf{Where does Merton fail?} (Tails? Specific shock types? Liquidity-contaminated spreads?)

\item \textbf{How large are the failures?} (20\% effect in tails? 50\%? Negligible?)

\item \textbf{Are failures systematic?} (Always front-end IG? Always sector shocks? Always illiquid bonds?)

\item \textbf{Do failures relate to Stage 0 findings?} (Is Financial amplification driven by tails? sector shocks? liquidity?)
\end{enumerate}

\textbf{Use Stage D results in Stage E} to decide:
\begin{itemize}
\item If tail effects large: Add quantile-specific $\lambda$ for VaR/ES
\item If sector shocks deviate: Consider sector-factor adjustments beyond static sector dummies
\item If liquidity matters: Decompose OAS for HY, keep total for IG
\item If Financial amplification comes from specific channel: Design targeted adjustment
\end{itemize}

But don't let Stage D derail the main program. If Stages A--C showed Merton works well on average (possibly with sector adjustments from Stage 0), Stage D refinements are enhancements, not requirements.


\section{Stage A: Establish Cross-Sectional Variation}
\label{sec:stage-a}

\subsection{Objective}

Stage A establishes the empirical fact that DTS betas differ across bonds \emph{before} testing whether Merton explains why. This separation is critical:

\begin{itemize}
\item If no significant variation exists, stop here—standard DTS is adequate
\item If variation exists but Merton doesn't explain it, we know theory fails
\item If variation exists and Merton explains it, theory provides parsimonious structure
\end{itemize}

\textbf{Relationship to Stage 0}: Stage 0 tested whether Merton predictions hold using three complementary approaches. Stage A takes a step back and asks the more fundamental question: \emph{Is there any systematic variation in DTS betas to explain?} This question is logically prior to testing whether theory explains the variation.

\subsection{Specification A.1: Bucket-level betas}

For each bucket $k$ (defined by rating $\times$ maturity $\times$ sector), estimate:
\begin{equation}
y_{i,t} = \alpha^{(k)} + \beta^{(k)} f^{(U)}_{DTS,t} + \varepsilon^{(k)}_{i,t}, \quad i \in \text{bucket } k
\end{equation}

\textbf{Bucket definitions}:
\begin{itemize}
\item Rating: AAA/AA, A, BBB for IG; BB, B, CCC for HY
\item Maturity: 1--2y, 2--3y, 3--5y, 5--7y, 7--10y, 10y+
\item Sector: Industrials, Financials, Utilities, Energy (can add more)
\end{itemize}

Creates $3 \times 6 \times 4 = 72$ buckets for IG, $3 \times 6 \times 4 = 72$ for HY (some will be sparse).

\textbf{Estimation}: Pooled OLS within each bucket, cluster standard errors by week.

\textbf{Key outputs}:
\begin{itemize}
\item Table A.1: $\hat{\beta}^{(k)}$ for all buckets with standard errors and $t$-statistics
\item $F$-test for equality across buckets: $H_0: \beta^{(1)} = \beta^{(2)} = \cdots = \beta^{(K)}$
\item \textbf{Critical decision}: If $F$-test fails to reject ($p > 0.10$), no significant variation—stop here, use standard DTS
\end{itemize}

\subsection{Specification A.2: Continuous characteristics}

Instead of discrete buckets, estimate how beta varies with continuous characteristics.

\textbf{Two-step procedure}:

\textbf{Step 1}: For each bond $i$, estimate bond-specific beta using rolling 2-year windows:
\begin{equation}
y_{i,t} = \alpha_i + \beta_i f_{DTS,t} + \varepsilon_{i,t}
\end{equation}

Yields time-series of $\hat{\beta}_{i,\tau}$ for bond $i$ at window midpoint $\tau$.

\textbf{Step 2}: Cross-sectional regression of estimated betas on characteristics:
\begin{equation}
\hat{\beta}_{i,\tau} = \gamma_0 + \gamma_M M_{i,\tau} + \gamma_s s_{i,\tau} + \gamma_{M^2} M_{i,\tau}^2 + \gamma_{Ms} M_{i,\tau} \cdot s_{i,\tau} + u_{i,\tau}
\end{equation}

where $M_{i,\tau}$ is maturity, $s_{i,\tau}$ is spread at window midpoint.

\textbf{Standard errors}: Bootstrap or cluster by bond (account for multiple windows per bond).

\textbf{Key outputs}:
\begin{itemize}
\item Coefficient estimates $\hat{\gamma}_M, \hat{\gamma}_s, \hat{\gamma}_{M^2}, \hat{\gamma}_{Ms}$
\item $R^2$ of cross-sectional fit
\item \textbf{Interpretation}: Do betas vary systematically with maturity and spread?
\end{itemize}

\subsection{Establishing the stylized fact}

\textbf{Deliverable}: Document that DTS betas are not constant. Show:
\begin{enumerate}
\item Statistical significance of variation ($F$-test from A.1, $R^2$ from A.2)
\item Economic significance: Range of $\hat{\beta}^{(k)}$ across buckets (e.g., min 0.6, max 1.8 implies 3$\times$ variation)
\item Pattern: Short-maturity / low-spread bonds have higher betas (preliminary observation, not causal claim)
\end{enumerate}

\textbf{Comparison with Stage 0}: Stage A results should be consistent with Stage 0 bucket-level analysis. The key difference is framing:
\begin{itemize}
\item Stage 0 asked: Do empirical betas match Merton predictions?
\item Stage A asks: Do betas vary at all, regardless of whether they match theory?
\end{itemize}

If Stage 0 found Merton works well (Path 1), Stage A should find variation that aligns with Merton predictions. If Stage 0 found Merton partially works or fails, Stage A documents the variation that theory fails to fully explain.

\subsection{Deliverables for Stage A}

\begin{itemize}
\item \textbf{Table A.1:} Bucket-level $\hat{\beta}^{(k)}$ estimates
\begin{itemize}
\item Rows: Maturity buckets
\item Columns: Rating buckets
\item Separate panels for IG and HY (and optionally by sector)
\item Include standard errors, $t$-statistics, sample size per bucket
\end{itemize}

\item \textbf{Table A.2:} Tests of beta equality
\begin{itemize}
\item $F$-test for $H_0:$ all $\beta^{(k)}$ equal, overall and by dimension:
\begin{itemize}
\item Across maturities (holding rating constant)
\item Across ratings (holding maturity constant)
\item Across sectors
\end{itemize}
\item Report $F$-statistic, degrees of freedom, $p$-value
\item \textbf{Decision rule}: If all $p > 0.10$, declare standard DTS adequate
\end{itemize}

\item \textbf{Table A.3:} Continuous characteristic regression (Specification A.2)
\begin{itemize}
\item Coefficients: $\hat{\gamma}_0, \hat{\gamma}_M, \hat{\gamma}_s, \hat{\gamma}_{M^2}, \hat{\gamma}_{Ms}$
\item Standard errors, $t$-statistics, $R^2$
\item Separate for IG and HY
\end{itemize}

\item \textbf{Figure A.1:} Heatmap of $\hat{\beta}^{(k)}$ by maturity (x-axis) $\times$ rating (y-axis). Color intensity represents beta magnitude. Separate panels for IG and HY.

\item \textbf{Figure A.2:} Implied beta surface from Specification A.2: 3D plot or contour plot with maturity on x-axis, spread on y-axis, predicted $\hat{\beta}$ on z-axis/color.

\item \textbf{Diagnostic summary (2 pages):}
\begin{enumerate}
\item Is variation statistically significant? ($F$-test results)
\item Is variation economically meaningful? (Range and IQR of $\hat{\beta}^{(k)}$)
\item What characteristics drive variation? (Maturity vs spread vs sector)
\item Does IG show more variation than HY? (Regime 2 prediction from Stage 0)
\item \textbf{Recommendation}: Proceed to Stage B to test whether Merton explains patterns, or stop if no variation.
\end{enumerate}
\end{itemize}

\subsection{Decision point}

\begin{itemize}
\item \textbf{If $F$-test $p < 0.01$ and $R^2 > 0.15$ in A.2}: Strong evidence of systematic variation. Proceed to Stage B with high confidence.

\item \textbf{If $F$-test $0.01 < p < 0.10$}: Marginal variation. Proceed to Stage B but may find theory sufficient.

\item \textbf{If $F$-test $p > 0.10$ and $R^2 < 0.05$}: No meaningful variation. \textbf{Stop here—standard DTS is adequate. Report this as primary finding.}
\end{itemize}

\textbf{Consistency check with Stage 0}: If Stage 0 found significant patterns (e.g., within-issuer $\hat{\beta} \neq 1$ or significant sector interactions), Stage A should find significant variation. If Stage A finds no variation but Stage 0 found patterns, revisit Stage 0 methodology—possible issue with bucket definitions or sample restrictions.


\section{Stage B: Does Merton Explain the Variation?}
\label{sec:stage-b}

\subsection{Objective}

Having established in Stage A that DTS betas vary across bonds, Stage B tests whether Merton's structural predictions explain this variation. This is the \emph{core empirical test} of the theoretical framework.

\textbf{Critical distinction}: Stage A documented the \emph{fact} of variation. Stage B tests whether \emph{theory explains} it.

\textbf{Relationship to Stage 0}: Stage 0 provided preliminary evidence on whether Merton predictions hold. Stage B conducts more rigorous tests with formal model comparisons and benchmarking against the unrestricted variation documented in Stage A.

\subsection{Specification B.1: Merton as offset (constrained)}

Use Merton-predicted $\lambda^{\text{Merton}}_{i,t}$ as known adjustment factor:
\begin{equation}
y_{i,t} = \alpha + \beta_{\text{Merton}} \cdot [\lambda^{\text{Merton}}_{i,t} \cdot f_{DTS,t}] + \varepsilon_{i,t}
\end{equation}

where $\lambda^{\text{Merton}}_{i,t} = \lambda_T(T_i; 5y, s_{i,t}) \times \lambda_s(s_{i,t}; 100\text{bps})$ using Tables~\ref{tab:merton-lambda-T}--\ref{tab:merton-lambda-s}.

\textbf{Theory prediction}: If Merton is exactly correct, $\beta_{\text{Merton}} = 1$.

\textbf{Test}: Wald test $H_0: \beta = 1$ with clustered standard errors (week $\times$ issuer).

\textbf{Interpretation}:
\begin{itemize}
\item $\hat{\beta}_{\text{Merton}} \in [0.9, 1.1]$: Merton predictions unbiased, theory works
\item $\hat{\beta}_{\text{Merton}} \in [0.8, 1.2]$: Close enough for practical purposes
\item $\hat{\beta}_{\text{Merton}} > 1.2$ or $< 0.8$: Systematic bias, need calibration
\item Low $R^2$ despite $\hat{\beta} \approx 1$: Merton captures mean but misses dispersion
\end{itemize}

\subsection{Specification B.2: Decomposed components}

Test maturity vs credit quality effects separately:
\begin{equation}
y_{i,t} = \alpha + \beta_T [\lambda_T(T_i; 5y, s_{i,t}) \cdot f_{DTS,t}] + \beta_s [\lambda_s(s_{i,t}; 100) \cdot f_{DTS,t}] + \varepsilon_{i,t}
\end{equation}

\textbf{Theory predictions}:
\begin{itemize}
\item $\beta_T \approx 1$: Maturity adjustment works
\item $\beta_s \approx 1$: Credit quality adjustment works
\item If both hold: Merton decomposition empirically valid
\end{itemize}

\textbf{Diagnostic patterns}:
\begin{itemize}
\item $\beta_T \approx 1$ but $\beta_s \neq 1$: Maturity effects correct, quality effects need recalibration
\item $\beta_T \neq 1$ but $\beta_s \approx 1$: Quality effects correct, maturity functional form wrong
\item Both $\neq 1$: Need to reconsider entire Merton structure
\end{itemize}

\subsection{Specification B.3: With sector adjustments}

Building on Stage 0 sector interaction findings, test whether adding sector adjustments to Merton improves fit:

\begin{equation}
y_{i,t} = \alpha + \beta_0 [\lambda^{\text{Merton}}_{i,t} \cdot f_{DTS,t}] + \sum_{s \in S} \beta_s \cdot \mathbbm{1}_{\{i \in s\}} \cdot [\lambda^{\text{Merton}}_{i,t} \cdot f_{DTS,t}] + \varepsilon_{i,t}
\end{equation}

\textbf{Comparison with Stage 0}: This is identical to the sector interaction specification from Stage 0 (equation \ref{eq:sector-interaction}). In Stage B, we use it to assess incremental explanatory power:
\begin{itemize}
\item Does adding sector interactions improve $R^2$ significantly beyond pure Merton?
\item Are sector coefficients stable when estimated jointly with Merton baseline?
\end{itemize}

\subsection{Benchmarking against Stage A}

\textbf{Key comparison}: Does theory-constrained Specification B.1 perform comparably to unrestricted bucket regressions from Stage A?

\textbf{Metrics}:
\begin{enumerate}
\item \textbf{$R^2$ comparison}:
\begin{itemize}
\item $R^2_{\text{buckets}}$ from Stage A (upper bound—fully flexible)
\item $R^2_{\text{Merton}}$ from Specification B.1
\item $R^2_{\text{Merton+Sector}}$ from Specification B.3
\item If $R^2_{\text{Merton}} > 0.9 \times R^2_{\text{buckets}}$: Theory captures 90\%+ of explainable variation
\end{itemize}

\item \textbf{Bucket-level residuals}: For each bucket $k$, compute:
\begin{equation}
\text{Residual}_k = \hat{\beta}^{(k)}_{\text{Stage A}} - \lambda^{\text{Merton}}_k
\end{equation}
If most residuals small (e.g., $<$0.2) and unsystematic, theory adequate.

\item \textbf{RMSE comparison}: Root mean squared error of spread change predictions:
\begin{equation}
\text{RMSE} = \sqrt{\frac{1}{N}\sum_{i,t} (y_{i,t} - \hat{y}_{i,t})^2}
\end{equation}
Compare Merton vs Merton+Sector vs bucket-based predictions.
\end{enumerate}

\subsection{Specification B.4: Unrestricted for comparison}

Estimate fully flexible functional form:
\begin{equation}
\lambda_i = \beta_0 + \beta_M M_i + \beta_{M^2} M_i^2 + \beta_s s_i + \beta_{s^2} s_i^2 + \beta_{Ms} M_i \cdot s_i + \sum_{\text{rating}} \beta_r + \sum_{\text{sector}} \beta_{\text{sec}}
\end{equation}

Then:
\begin{equation}
y_{i,t} = \alpha + [\hat{\lambda}_i \cdot f_{DTS,t}] + \varepsilon_{i,t}
\end{equation}

\textbf{Purpose}: Assess incremental explanatory power beyond theory. If $R^2_{\text{unrestricted}} \gg R^2_{\text{Merton}}$, theory misses important patterns.

\subsection{Theory vs reality table}

\textbf{Critical deliverable}: Direct comparison of empirical betas to Merton predictions from \cite{Wuebben2025}.

\begin{table}[h]
\centering
\caption{Do Empirical Betas Match Merton Predictions? (Illustrative)}
\label{tab:theory-reality}
\begin{tabular}{lccc}
\toprule
\textbf{Bucket} & $\hat{\beta}^{(k)}$ (Stage A) & $\lambda^{\text{Merton}}_k$ (Theory) & Ratio \\
\midrule
1y, 50bps & 3.45 & 3.62 & 0.95 \\
1y, 100bps & 3.15 & 3.27 & 0.96 \\
1y, 200bps & 2.85 & 2.78 & 1.03 \\
\midrule
3y, 50bps & 1.52 & 1.47 & 1.03 \\
3y, 100bps & 1.38 & 1.42 & 0.97 \\
3y, 200bps & 1.29 & 1.36 & 0.95 \\
\midrule
10y, 50bps & 0.58 & 0.61 & 0.95 \\
10y, 100bps & 0.61 & 0.64 & 0.95 \\
10y, 200bps & 0.65 & 0.67 & 0.97 \\
\bottomrule
\end{tabular}
\end{table}

\textbf{Decision criteria}:
\begin{itemize}
\item \textbf{If 90\%+ of ratios in [0.8, 1.2]}: Merton provides excellent baseline. Use pure Merton or calibrated version for production.

\item \textbf{If systematic bias} (all ratios $> 1.2$ or $< 0.8$): Recalibrate with $\beta_{\text{Merton}}$ from Specification B.1. Theory has right structure, wrong scale.

\item \textbf{If high dispersion but no bias}: Heterogeneity beyond Merton dimensions. Proceed to unrestricted estimation, but Merton still useful as starting point.

\item \textbf{If wrong patterns} (e.g., long bonds have higher sensitivity than short): Theory fundamentally fails. Investigate alternative mechanisms.
\end{itemize}

\subsection{Deliverables for Stage B}

\begin{itemize}
\item \textbf{Table B.1:} Constrained Merton specifications
\begin{itemize}
\item Spec B.1: $\hat{\beta}_{\text{Merton}}$, standard error, Wald test $p$-value for $H_0: \beta=1$
\item Spec B.2: $\hat{\beta}_T$, $\hat{\beta}_s$, standard errors, joint test $p$-value for $H_0: (\beta_T, \beta_s) = (1,1)$
\item Spec B.3: $\hat{\beta}_0$, sector coefficients $\hat{\beta}_s$, joint test for sector effects
\item Separate panels for IG and HY
\end{itemize}

\item \textbf{Table B.2:} Model comparison
\begin{itemize}
\item Rows: Stage A buckets, Spec B.1 (Merton), Spec B.2 (decomposed), Spec B.3 (Merton+Sector), Spec B.4 (unrestricted)
\item Columns: $R^2$, RMSE, AIC, number of parameters
\item $\Delta R^2$ relative to Stage A buckets
\end{itemize}

\item \textbf{Table B.3:} Theory vs Reality (as in Table~\ref{tab:theory-reality})
\begin{itemize}
\item All maturity $\times$ spread bucket combinations
\item Empirical $\hat{\beta}^{(k)}$ from Stage A
\item Theoretical $\lambda^{\text{Merton}}_k$
\item Ratio and absolute deviation
\item Highlight cells where $|\text{Ratio} - 1| > 0.25$
\end{itemize}

\item \textbf{Figure B.1:} Scatter plot: Empirical $\hat{\beta}^{(k)}$ (y-axis) vs Theoretical $\lambda^{\text{Merton}}_k$ (x-axis) for all buckets. 45-degree line for perfect agreement. Color-code by:
\begin{itemize}
\item IG-narrow maturity (circles)
\item IG-wide maturity (squares)
\item HY-narrow (triangles)
\item HY-wide (diamonds)
\item Distressed (stars)
\end{itemize}

\item \textbf{Figure B.2:} Residual analysis: $\hat{\beta}^{(k)} - \lambda^{\text{Merton}}_k$ by:
\begin{itemize}
\item Panel A: By maturity (x-axis: 1y, 3y, 5y, 7y, 10y)
\item Panel B: By spread level (x-axis: 50, 100, 200, 300, 500, 1000 bps)
\item Panel C: By sector (x-axis: Industrials, Financials, Utilities, Energy)
\end{itemize}
Zero line indicates perfect Merton prediction. Look for systematic patterns.

\item \textbf{Figure B.3:} Implied $\lambda$ surface from Spec B.4 (unrestricted) vs Merton prediction:
\begin{itemize}
\item 3D surface plot or side-by-side contour plots
\item X-axis: Maturity (1--10 years)
\item Y-axis: Spread (50--1000 bps)
\item Z-axis/color: Predicted $\lambda$
\item Shows where unrestricted deviates from theory
\end{itemize}

\item \textbf{Diagnostic summary (3--4 pages):}
\begin{enumerate}
\item \textbf{Does Merton work?} (Spec B.1 results: $\hat{\beta}_{\text{Merton}}$ and $R^2$)

\item \textbf{Which component drives fit?} (Spec B.2 results: maturity vs quality)

\item \textbf{Do sector adjustments help?} (Spec B.3 results: incremental $R^2$ from sectors)

\item \textbf{Where does theory succeed?} (Table B.3 analysis: which regimes have ratios near 1.0)

\item \textbf{Where does theory fail?} (Residual patterns from Figure B.2)

\item \textbf{Is unrestricted necessary?} (Table B.2 comparison: $\Delta R^2$ and parameter efficiency)

\item \textbf{Practical recommendation}:
\begin{itemize}
\item Use pure Merton tables (simplest)
\item Use calibrated Merton with $\hat{\beta}_{\text{Merton}}$ (simple + data-driven)
\item Use Merton with sector adjustments (theory + sector factors)
\item Need full unrestricted (complex but necessary)
\end{itemize}

\item \textbf{Consistency with Stage 0}: Do Stage B findings align with Stage 0 conclusions? If Stage 0 found significant sector effects, does Spec B.3 confirm their importance?
\end{enumerate}
\end{itemize}

\subsection{Decision point}

Based on Stage B results:

\begin{mdframed}[backgroundcolor=yellow!10!white,roundcorner=4pt]
\textbf{Decision Tree for Stage C Entry:}

\textbf{Path 1: Theory works well}

\textit{Condition}: $\hat{\beta}_{\text{Merton}} \in [0.9, 1.1]$ and $R^2_{\text{Merton}} > 0.85 \times R^2_{\text{buckets}}$

\textit{Action}: Proceed to Stage C to test whether static $\lambda^{\text{Merton}}$ suffices or time-variation needed. High confidence in theoretical foundation.

\textbf{Path 2: Theory works with sector adjustments}

\textit{Condition}: $\hat{\beta}_{\text{Merton}} \in [0.9, 1.1]$ but $R^2_{\text{Merton+Sector}} > R^2_{\text{Merton}} + 0.03$ with significant sector coefficients

\textit{Action}: Adopt Merton with sector adjustments as baseline. Proceed to Stage C to test stability of both Merton coefficient and sector effects over time.

\textbf{Path 3: Theory needs calibration}

\textit{Condition}: $\hat{\beta}_{\text{Merton}}$ outside $[0.9, 1.1]$ but patterns match (high $R^2$, residuals unsystematic)

\textit{Action}: Adopt calibrated Merton: $\lambda^{\text{prod}} = \hat{\beta}_{\text{Merton}} \cdot \lambda^{\text{Merton}}$. Proceed to Stage C to test stability of $\hat{\beta}_{\text{Merton}}$ over time.

\textbf{Path 4: Theory captures structure but misses details}

\textit{Condition}: Moderate $R^2_{\text{Merton}}$ (e.g., 0.6--0.8 $\times$ buckets), some systematic residuals

\textit{Action}: Proceed to Stage C with both theory-guided and unrestricted tracks in parallel. Stage C will reveal if time-variation helps or if static unrestricted better.

\textbf{Path 5: Theory fundamentally fails}

\textit{Condition}: Wrong patterns (e.g., long bonds more sensitive than short), or $R^2_{\text{Merton}} < 0.5 \times R^2_{\text{buckets}}$

\textit{Action}: Skip Stage C (no point testing time-variation of failed model). Proceed directly to Stage D (robustness) to diagnose \emph{why} theory fails (liquidity? tails? specific shocks?). Then Stage E with unrestricted specification only.

\textit{Implication}: Report that structural models don't provide adequate guidance for DTS adjustments in this market. Empirical approach necessary.
\end{mdframed}


\section{Stage C: Does Static Merton Suffice or Do We Need Time-Variation?}
\label{sec:stage-c}

\subsection{Objective and prerequisite}

\textbf{Prerequisite}: Stage B showed that Merton $\lambda(s, T)$ explains cross-sectional variation (Paths 1--4 from Stage B decision tree).

\textbf{Objective}: Test whether the relationship between $\lambda$ and $(s, T)$ is stable over time, or whether macro state variables induce time-variation.

\textbf{Key principle}: Don't add time-variation until you've proven the simple static model fails.

\textbf{Extended objective}: If Stage 0 and Stage B found significant sector effects, also test whether sector adjustments are stable over time or regime-dependent.

\subsection{Rolling window stability test}

Divide sample into non-overlapping 1-year windows $w \in \{1, 2, \dots, W\}$. For each window:

\begin{equation}
y_{i,t} = \alpha_w + \beta_w \cdot [\lambda^{\text{Merton}}_{i,t} \cdot f_{DTS,t}] + \varepsilon_{i,t}, \quad t \in w
\end{equation}

This yields time series of $\hat{\beta}_w$ with standard errors $\text{se}(\hat{\beta}_w)$.

\textbf{Stability test}: Chow test for structural break:
\begin{equation}
H_0: \beta_1 = \beta_2 = \cdots = \beta_W
\end{equation}

Compute $F$-statistic comparing restricted (single $\beta$) vs unrestricted (separate $\beta_w$) models.

\textbf{Decision rule}:
\begin{itemize}
\item \textbf{If $p > 0.10$}: Static $\lambda$ sufficient. \textbf{Stop Stage C here.} No need for time-varying adjustments. Report that Merton provides stable baseline.

\item \textbf{If $0.01 < p < 0.10$}: Marginal instability. Proceed to investigate drivers but be skeptical of over-parameterization.

\item \textbf{If $p < 0.01$}: Significant time-variation. Proceed to macro driver analysis.
\end{itemize}

\subsection{Stability of sector effects}

If Stage B adopted Merton with sector adjustments (Path 2), test whether sector coefficients are stable:

\begin{equation}
y_{i,t} = \alpha_w + \beta_{0,w} [\lambda^{\text{Merton}}_{i,t} \cdot f_{DTS,t}] + \sum_{s \in S} \beta_{s,w} \cdot \mathbbm{1}_{\{i \in s\}} \cdot [\lambda^{\text{Merton}}_{i,t} \cdot f_{DTS,t}] + \varepsilon_{i,t}
\end{equation}

\textbf{Tests}:
\begin{enumerate}
\item Chow test for base coefficient: $H_0: \beta_{0,1} = \beta_{0,2} = \cdots = \beta_{0,W}$
\item Chow test for each sector: $H_0: \beta_{s,1} = \beta_{s,2} = \cdots = \beta_{s,W}$ for each $s$
\item Joint stability test: All coefficients stable across windows
\end{enumerate}

\textbf{Interpretation}:
\begin{itemize}
\item If base $\beta_0$ stable but sector $\beta_s$ vary: Sector effects are regime-dependent (e.g., Financials amplify more during crises)
\item If base $\beta_0$ varies but sector $\beta_s$ stable: Macro conditions affect all bonds similarly; sector differentials constant
\item If both vary: Complex regime structure; may need VIX or spread-level interactions
\end{itemize}

\subsection{Visual stability assessment}

\textbf{Time series plot}: $\hat{\beta}_w$ over time with 95\% confidence bands.

\textbf{Interpretation}:
\begin{itemize}
\item Confidence bands overlapping 1.0 throughout: Static Merton works, bands capture sampling variation
\item Confidence bands tight but $\hat{\beta}_w$ drifts (e.g., 0.9 in 2010s, 1.1 in 2020s): Systematic shift, investigate macro drivers
\item Wide swings during crises (2020, 2022) but stable otherwise: Regime-dependent but static in normal times
\end{itemize}

\subsection{Conditional on instability: Macro driver analysis}

\textbf{Only if Chow test rejects}, estimate second-stage regression:

\begin{equation}
\hat{\beta}_w = \delta_0 + \delta_{\text{VIX}} \cdot \overline{\text{VIX}}_w + \delta_{\text{OAS}} \cdot \log(\overline{\text{OAS}}_{index,w}) + \delta_r \cdot \overline{r}_{10y,w} + \eta_w
\end{equation}

where $\overline{X}_w$ denotes window-average of variable $X$.

\textbf{Theory-based predictions} \cite{Wuebben2025}:
\begin{enumerate}
\item $\delta_{\text{VIX}} > 0$: High volatility amplifies sensitivity, especially for short maturities (flight-to-quality concentrates in long end, front end whipsaws more)

\item $\delta_{\text{OAS}} < 0$: Wide spreads reduce dispersion in $\lambda$ (convergence to Regime 5 where all bonds near default, proportionality improves)

\item $\delta_r$: Ambiguous. Higher rates increase discount effect (reduce duration), possibly dampening elasticities. Empirical question.
\end{enumerate}

\textbf{Economic significance threshold}: Only declare time-variation meaningful if macro state changes $\lambda$ by $>$20\% over sample range.

Example: If $\delta_{\text{VIX}} = 0.01$ and VIX ranges from 10 to 40, the effect is $0.01 \times 30 = 0.30$, or 30\% change in $\beta$. This is economically large.

If $\delta_{\text{VIX}} = 0.002$, effect is 6\%—within noise, ignore.

\subsection{Maturity-specific time-variation}

Theory predicts time-variation should differ by maturity: short-maturity IG bonds most affected.

\textbf{Test}: Estimate rolling $\beta_w$ separately for maturity buckets:
\begin{equation}
y_{i,t} = \alpha_{w,m} + \beta_{w,m} \cdot [\lambda^{\text{Merton}}_{i,t} \cdot f_{DTS,t}] + \varepsilon_{i,t}, \quad i \in \text{maturity bucket } m, \, t \in w
\end{equation}

Then regress:
\begin{equation}
\hat{\beta}_{w,m} = \delta_{0,m} + \delta_{\text{VIX},m} \cdot \overline{\text{VIX}}_w + \eta_{w,m}
\end{equation}

\textbf{Prediction}: $\delta_{\text{VIX},1y} > \delta_{\text{VIX},5y} > \delta_{\text{VIX},10y}$ (short bonds more regime-dependent).

If this pattern holds, supports theory-based intuition about crisis dynamics.

\subsection{Sector-specific time-variation}

Building on Stage 0 sector findings, test whether sector effects are regime-dependent:

\begin{equation}
\hat{\beta}_{s,w} = \delta_{s,0} + \delta_{s,\text{VIX}} \cdot \overline{\text{VIX}}_w + \eta_{s,w}
\end{equation}

\textbf{Predictions}:
\begin{itemize}
\item $\delta_{\text{Financial},\text{VIX}} > 0$: Financial sector amplification increases during stress (correlation risk, liquidity spirals)
\item $\delta_{\text{Utility},\text{VIX}} \approx 0$: Utility defensive characteristics stable across regimes
\item $\delta_{\text{Energy},\text{VIX}}$: Ambiguous—energy may decouple from broad market during commodity-specific shocks
\end{itemize}

\subsection{Practical implication assessment}

\textbf{Question}: Even if time-variation is statistically significant, does it matter for portfolio management?

\textbf{Scenario analysis}: Compare static vs time-varying $\lambda$ for:
\begin{enumerate}
\item \textbf{Risk model accuracy}: Does time-varying $\lambda$ reduce tracking error in out-of-sample hedging?

\item \textbf{Crisis performance}: During 2020 COVID shock, did static $\lambda$ severely misprice front-end IG?

\item \textbf{Operational complexity}: Time-varying $\lambda$ requires daily macro state inputs and recalibration. Worth the cost?
\end{enumerate}

\textbf{Recommendation framework}:
\begin{itemize}
\item If time-variation changes risk estimates by $<$10\% except during rare crises: Use static $\lambda$, add crisis overlays manually

\item If time-variation changes risk estimates by $>$20\% routinely: Implement time-varying $\lambda$ with macro state

\item If time-variation important only for specific buckets (e.g., 1--2y IG): Use static for most bonds, time-varying for front end only

\item If sector effects are regime-dependent: Use static sector adjustments with VIX-contingent overlays for Financials
\end{itemize}

\subsection{Deliverables for Stage C}

\begin{itemize}
\item \textbf{Table C.1:} Rolling window stability test
\begin{itemize}
\item Rows: Time windows (2013--2014, 2014--2015, ..., 2024--2025)
\item Columns: $\hat{\beta}_w$, standard error, 95\% CI, sample size
\item Separate panels for IG and HY
\item Chow test: $F$-statistic, $p$-value
\end{itemize}

\item \textbf{Table C.2:} Sector coefficient stability (if applicable)
\begin{itemize}
\item Rows: Windows
\item Columns: $\hat{\beta}_{0,w}$, $\hat{\beta}_{\text{Financial},w}$, $\hat{\beta}_{\text{Utility},w}$, $\hat{\beta}_{\text{Energy},w}$
\item Chow tests for each coefficient series
\end{itemize}

\item \textbf{Table C.3:} Macro driver regression (conditional on instability)
\begin{itemize}
\item Coefficients: $\hat{\delta}_{\text{VIX}}$, $\hat{\delta}_{\text{OAS}}$, $\hat{\delta}_r$
\item Standard errors, $t$-statistics, $R^2$
\item Economic significance: Effect of 1 SD change in each macro variable on $\beta$
\item Test predicted signs: $\delta_{\text{VIX}} > 0$, $\delta_{\text{OAS}} < 0$
\end{itemize}

\item \textbf{Table C.4:} Maturity-specific time-variation
\begin{itemize}
\item Rows: Maturity buckets (1--2y, 3--5y, 7--10y)
\item Columns: $\hat{\delta}_{\text{VIX},m}$, standard error, $t$-statistic
\item Test: Is $\delta_{\text{VIX},1y}$ significantly larger than $\delta_{\text{VIX},10y}$?
\end{itemize}

\item \textbf{Table C.5:} Sector-specific time-variation (if applicable)
\begin{itemize}
\item Rows: Sectors (Financial, Utility, Energy)
\item Columns: $\hat{\delta}_{s,\text{VIX}}$, standard error, $t$-statistic
\item Test: Is $\delta_{\text{Financial},\text{VIX}} > 0$?
\end{itemize}

\item \textbf{Figure C.1:} Time series of $\hat{\beta}_w$ for IG and HY
\begin{itemize}
\item X-axis: Year (2013--2025)
\item Y-axis: $\hat{\beta}_w$
\item Point estimates with 95\% confidence bands
\item Horizontal line at $\beta = 1$ (theory prediction)
\item Shade crisis periods (2020 COVID, 2022 rate shock)
\item Interpretation: Does $\hat{\beta}_w$ spike during crises?
\end{itemize}

\item \textbf{Figure C.2:} $\hat{\beta}_w$ vs macro state variables
\begin{itemize}
\item Panel A: $\hat{\beta}_w$ (y-axis) vs $\overline{\text{VIX}}_w$ (x-axis)
\item Panel B: $\hat{\beta}_w$ (y-axis) vs $\log(\overline{\text{OAS}}_w)$ (x-axis)
\item Scatter with OLS fit line
\item Color-code by time period (pre-2020, COVID, post-COVID)
\item Shows whether macro variables predict time-variation
\end{itemize}

\item \textbf{Figure C.3:} Implied $\lambda_{i,t}$ for representative bonds over time
\begin{itemize}
\item Three lines: 1-year BBB, 5-year BBB, 10-year BBB (all industrial)
\item Static $\lambda$ (dashed) vs time-varying $\lambda_t$ (solid)
\item Shows when and how much time-variation matters
\end{itemize}

\item \textbf{Figure C.4:} Sector coefficient stability (if applicable)
\begin{itemize}
\item Time series of $\hat{\beta}_{\text{Financial},w}$, $\hat{\beta}_{\text{Utility},w}$, $\hat{\beta}_{\text{Energy},w}$
\item Shows whether sector effects are stable or regime-dependent
\end{itemize}

\item \textbf{Figure C.5:} Scenario analysis—crisis vs normal
\begin{itemize}
\item Histogram of spread changes during normal periods (VIX $<$ 20)
\item Histogram of spread changes during stress (VIX $>$ 30)
\item Overlay: Static Merton prediction, time-varying prediction
\item Shows whether static model fails systematically in crises
\end{itemize}

\item \textbf{Summary and recommendation (3--4 pages):}
\begin{enumerate}
\item \textbf{Is base relationship stable?} (Chow test results for $\beta_w$)

\item \textbf{Are sector effects stable?} (Chow test results for $\beta_{s,w}$)

\item \textbf{If unstable, what drives it?} (Macro driver analysis)

\item \textbf{Is instability economically meaningful?} (Effect size in \% terms)

\item \textbf{Does theory-based intuition hold?} (VIX amplifies front-end, OAS compresses dispersion)

\item \textbf{Practical recommendation}:
\begin{itemize}
\item Use static $\lambda$: Adequate for normal markets, simple implementation
\item Use time-varying $\lambda$: Necessary for crisis periods, worth complexity
\item Hybrid: Static baseline with crisis adjustments (VIX $>$ 30)
\item For sectors: Static sector adjustments vs regime-dependent
\end{itemize}

\item \textbf{Implication for production}: If static suffices, Stage E will select among pure/calibrated Merton with or without sector adjustments. If time-varying needed, add macro state to production spec.
\end{enumerate}
\end{itemize}

\subsection{Decision point}

\begin{itemize}
\item \textbf{If Chow test $p > 0.10$ for both base and sectors}: Static $\lambda$ sufficient. Proceed to Stage D (robustness) with confidence in stable baseline.

\item \textbf{If Chow test $p < 0.10$ for base but sectors stable}: Time-variation in overall level but not in sector differentials. Consider VIX overlay on base $\beta_0$.

\item \textbf{If Chow test $p > 0.10$ for base but sectors unstable}: Stable overall but regime-dependent sector effects. Use static base with VIX-contingent sector adjustments.

\item \textbf{If Chow test $p < 0.01$ and effects $>$20\% in crises}: Time-varying $\lambda$ necessary. Incorporate macro state in Stage E production specification.
\end{itemize}


\section{Stage D: Robustness and Extensions}
\label{sec:stage-d}

\subsection{Objective and positioning}

\textbf{Prerequisite}: Stages A--C established whether/how Merton predictions hold for \emph{mean} spread changes in standard conditions.

\textbf{Objective}: Test robustness across:
\begin{enumerate}
\item Tail events (quantile regression)
\item Shock types (systematic vs idiosyncratic decomposition)
\item Spread components (default vs liquidity)
\end{enumerate}

\textbf{Key framing}: These are \textbf{secondary} tests. If Stages A--C show Merton fails, Stage D helps diagnose \emph{why}. If Stages A--C validate Merton, Stage D confirms it's not just a mean effect.

\textbf{Relationship to Stage 0}: Stage D extends the robustness checks beyond what Stage 0's three-pronged approach covers. While Stage 0 tested Merton in aggregate (buckets), within issuers, and across sectors, Stage D examines distributional properties (tails), shock decomposition, and spread component separation.

\subsection{D.1: Tail behavior (quantile regression)}

\subsubsection{Motivation}

Merton model assumes normal shocks (geometric Brownian motion for firm value). If tails differ from mean, this suggests:
\begin{itemize}
\item Jump-to-default risk not captured by continuous diffusion
\item Liquidity evaporation in stress (left tail)
\item Asymmetric investor behavior (panic selling vs gradual buying)
\end{itemize}

\subsubsection{Specification}

For quantiles $\tau \in \{0.05, 0.10, 0.25, 0.50, 0.75, 0.90, 0.95\}$, estimate:

\begin{equation}
Q_{\tau}(y_{i,t} \mid f_{DTS,t}, \lambda^{\text{Merton}}_i) = \alpha_{\tau} + \beta_{\tau} \cdot [\lambda^{\text{Merton}}_i \cdot f_{DTS,t}]
\end{equation}

This models the $\tau$-th conditional quantile of spread changes, allowing elasticity to differ across the distribution.

\textbf{Merton prediction}: $\beta_{\tau} \approx 1$ for all $\tau$ (all quantiles respect structural elasticities).

\textbf{Diagnostics}:
\begin{itemize}
\item If $\beta_{0.05} \gg \beta_{0.95}$: Left tail (spread widening) has amplified sensitivity. Consistent with jump-to-default or liquidity spirals.

\item If $\beta_{0.95} > \beta_{0.05}$: Right tail (spread tightening) more sensitive. Suggests momentum/technical buying in rallies.

\item If $\beta_{\tau}$ U-shaped (high at both tails): Both extreme moves behave differently than moderate moves—non-linearity in spread dynamics.
\end{itemize}

\subsubsection{Sector-specific tail behavior}

Building on Stage 0 sector findings, estimate quantile regressions separately by sector:

\begin{equation}
Q_{\tau}(y_{i,t} \mid f_{DTS,t}, \lambda^{\text{Merton}}_i, \text{sector}_i = s) = \alpha_{\tau,s} + \beta_{\tau,s} \cdot [\lambda^{\text{Merton}}_i \cdot f_{DTS,t}]
\end{equation}

\textbf{Hypothesis}: If Stage 0 found Financials have amplified mean sensitivity, do they also have amplified tail sensitivity?
\begin{itemize}
\item $\beta_{0.05,\text{Financial}} > \beta_{0.05,\text{Industrial}}$: Financial left-tail risk exceeds mean amplification (correlation risk manifests in tails)
\item $\beta_{0.05,\text{Utility}} \approx \beta_{0.50,\text{Utility}}$: Utility defensive characteristics hold in tails
\end{itemize}

\subsubsection{Practical implications}

\textbf{Risk management}:
\begin{itemize}
\item Use $\beta_{0.05}$ for Value-at-Risk (VaR) and Expected Shortfall (ES) calculations
\item Use $\beta_{0.50}$ for expected return / attribution models
\item If $\beta_{0.05} = 1.5 \times \beta_{0.50}$: Tail risk 50\% larger than mean-based models predict
\end{itemize}

\textbf{Stress testing}:
\begin{itemize}
\item Standard Merton $\lambda$ may underestimate losses in left-tail scenarios
\item Adjust: $DTS^{*,\text{stress}}_{i,t} = \beta_{0.05} \cdot \lambda^{\text{Merton}}_i \cdot DTS_{i,t}$
\end{itemize}

\subsubsection{Deliverables for D.1}

\begin{itemize}
\item \textbf{Table D.1:} Quantile-specific $\beta_{\tau}$ estimates
\begin{itemize}
\item Rows: $\tau \in \{0.05, 0.10, 0.25, 0.50, 0.75, 0.90, 0.95\}$
\item Columns: $\hat{\beta}_{\tau}$, standard error, 95\% CI
\item Separate panels for IG and HY
\item Test: $H_0: \beta_{0.05} = \beta_{0.50}$ and $H_0: \beta_{0.95} = \beta_{0.50}$
\end{itemize}

\item \textbf{Table D.2:} Sector-specific tail behavior
\begin{itemize}
\item Rows: Sectors (Industrial, Financial, Utility, Energy)
\item Columns: $\beta_{0.05}$, $\beta_{0.50}$, $\beta_{0.95}$, Ratio $\beta_{0.05}/\beta_{0.50}$
\item Shows which sectors have amplified tail risk
\end{itemize}

\item \textbf{Figure D.1:} Plot of $\hat{\beta}_{\tau}$ across $\tau \in [0.05, 0.95]$
\begin{itemize}
\item X-axis: Quantile $\tau$
\item Y-axis: $\hat{\beta}_{\tau}$
\item Horizontal line at $\beta = 1$ (Merton prediction)
\item Confidence bands (bootstrap)
\item Interpretation: Flat line = Merton works across distribution. Upward/downward slope = asymmetry.
\end{itemize}

\item \textbf{Figure D.2:} Sector-specific quantile plots
\begin{itemize}
\item Separate lines for Industrial, Financial, Utility, Energy
\item Shows whether Financial tail amplification exceeds mean amplification
\end{itemize}

\item \textbf{Interpretation note}: Are tail deviations concentrated in specific regimes (e.g., front-end IG)? If so, suggests these bonds have additional jump risk beyond Merton's continuous framework.
\end{itemize}

\subsection{D.2: Shock decomposition}

\subsubsection{Motivation}

Merton model treats all firm value shocks identically—whether macro, sector, or idiosyncratic, the elasticity $\lambda_i$ should be the same because all operate through the firm's asset value.

Empirically test: Do different shock types exhibit different elasticities? If so, suggests mechanisms beyond firm fundamentals (e.g., liquidity contagion for sector shocks, information asymmetry for idiosyncratic shocks).

\subsubsection{Factor construction}

Decompose bond $i$'s spread change into orthogonal components:

\begin{align}
y_{i,t} &= \underbrace{f^{(G)}_{DTS,t}}_{\text{Global factor}} + \underbrace{f^{(S)}_{DTS,s(i),t}}_{\text{Sector factor}} + \underbrace{f^{(I)}_{DTS,i,t}}_{\text{Issuer-specific}} + \varepsilon_{i,t}
\end{align}

\textbf{Estimation procedure}:

\textbf{Step 1}: Global factor
\begin{equation}
f^{(G)}_{DTS,t} = \frac{\Delta OAS^{(U)}_t}{OAS^{(U)}_{t-1}}
\end{equation}

\textbf{Step 2}: Sector factors (orthogonalized to global)
\begin{equation}
f^{(S)}_{DTS,s,t} = \frac{\Delta OAS^{(U)}_{s,t}}{OAS^{(U)}_{s,t-1}} - f^{(G)}_{DTS,t}
\end{equation}

\textbf{Step 3}: Issuer-specific (residual)
\begin{equation}
f^{(I)}_{DTS,i,t} = y_{i,t} - f^{(G)}_{DTS,t} - f^{(S)}_{DTS,s(i),t}
\end{equation}

\subsubsection{Multi-factor regression with Merton baseline}

Estimate:
\begin{equation}
y_{i,t} = \beta^{(G)} [\lambda^{\text{Merton}}_i \cdot f^{(G)}_{DTS,t}] + \beta^{(S)} [\lambda^{\text{Merton}}_i \cdot f^{(S)}_{DTS,s(i),t}] + \beta^{(I)} [\lambda^{\text{Merton}}_i \cdot f^{(I)}_{DTS,i,t}] + \varepsilon_{i,t}
\end{equation}

\textbf{Constrained specification}: Impose $\beta^{(G)} = \beta^{(S)} = \beta^{(I)} = 1$, estimate single coefficient:
\begin{equation}
y_{i,t} = \beta \cdot \lambda^{\text{Merton}}_i \cdot [f^{(G)}_t + f^{(S)}_{s(i),t} + f^{(I)}_{i,t}] + \varepsilon_{i,t}
\end{equation}

\textbf{Merton prediction}: $\beta^{(G)} \approx \beta^{(S)} \approx \beta^{(I)} \approx 1$ (all shocks respect structural elasticities).

\subsubsection{Diagnostic patterns}

\begin{itemize}
\item \textbf{If $\beta^{(G)} \approx \beta^{(S)} \approx \beta^{(I)} \approx 1$}: Merton applies uniformly—all shocks operate through firm value.

\item \textbf{If $\beta^{(S)} > \beta^{(G)}$}: Sector shocks have amplified effects. Suggests contagion, correlation trading, or common liquidity factors beyond fundamentals.

\item \textbf{If $\beta^{(I)} \gg \beta^{(G)}$}: Idiosyncratic news has exaggerated spread impact. Consistent with information asymmetry, adverse selection in trading.

\item \textbf{If $\beta^{(G)} < 1$ but $\beta^{(I)} > 1$}: Bonds under-react to macro (diversified portfolios stabilize) but over-react to issuer-specific (concentrated positions, forced selling).
\end{itemize}

\subsubsection{Connection to Stage 0 sector findings}

If Stage 0 found Financials have amplified sensitivity, test whether this comes from:
\begin{itemize}
\item Global shocks: $\beta^{(G)}_{\text{Financial}} > \beta^{(G)}_{\text{Industrial}}$ (Financials more sensitive to macro)
\item Sector shocks: $\beta^{(S)}_{\text{Financial}} > \beta^{(S)}_{\text{Industrial}}$ (correlation/contagion within Financial sector)
\item Idiosyncratic shocks: $\beta^{(I)}_{\text{Financial}} > \beta^{(I)}_{\text{Industrial}}$ (information asymmetry in Financial credits)
\end{itemize}

This decomposition helps understand \emph{why} sector effects exist, informing production model design.

\subsubsection{Deliverables for D.2}

\begin{itemize}
\item \textbf{Table D.3:} Variance decomposition
\begin{itemize}
\item Rows: Rating $\times$ maturity buckets
\item Columns: \% variance from Global, Sector, Issuer-specific, Residual
\item Shows relative importance of each factor type
\item Separate for IG and HY (expect IG more global-driven, HY more issuer-specific)
\end{itemize}

\item \textbf{Table D.4:} Shock-specific elasticities
\begin{itemize}
\item Rows: $\beta^{(G)}$, $\beta^{(S)}$, $\beta^{(I)}$
\item Columns: Estimate, standard error, 95\% CI
\item Test: $H_0: \beta^{(G)} = \beta^{(S)} = \beta^{(I)} = 1$ (joint test)
\item Test: Pairwise $H_0: \beta^{(G)} = \beta^{(S)}$, etc.
\end{itemize}

\item \textbf{Table D.5:} Sector-specific shock decomposition
\begin{itemize}
\item Rows: Sectors (Industrial, Financial, Utility, Energy)
\item Columns: $\beta^{(G)}_s$, $\beta^{(S)}_s$, $\beta^{(I)}_s$
\item Shows which shock types drive each sector's behavior
\end{itemize}

\item \textbf{Figure D.3:} Bar chart of $\hat{\beta}$ by factor type
\begin{itemize}
\item Three bars: Global, Sector, Issuer-specific
\item Error bars for 95\% CI
\item Horizontal line at $\beta = 1$
\item Separate panels for IG and HY
\end{itemize}

\item \textbf{Interpretation}: If all three $\approx$ 1, Merton universally applicable. If sector/issuer deviate, need factor-specific adjustments in production models.
\end{itemize}

\subsection{D.3: Liquidity adjustment}

\subsubsection{Motivation}

OAS includes both default risk and liquidity premium. Merton model predicts elasticity for \emph{default component only}. If liquidity shocks don't respect structural elasticities, need to decompose.

\subsubsection{Liquidity-adjusted spread construction}

\textbf{Step 1}: Estimate liquidity component cross-sectionally each period:
\begin{equation}
s^{\text{liq}}_{i,t} = \phi_0 + \phi_1 \text{BidAsk}_{i,t} + \phi_2 \log(\text{Size}_i) + \phi_3 \log(\text{Turnover}_{i,t}) + \phi_4 \text{Age}_{i,t} + \eta_{i,t}
\end{equation}

Standard approach: Cross-sectional regression within each rating $\times$ maturity cell to avoid confounding credit quality with liquidity.

\textbf{Step 2}: Define default component as residual:
\begin{equation}
OAS^{\text{def}}_{i,t} = OAS_{i,t} - \widehat{OAS}^{\text{liq}}_{i,t}
\end{equation}

\textbf{Step 3}: Compute default-based spread changes:
\begin{equation}
y^{\text{def}}_{i,t} = \frac{\Delta OAS^{\text{def}}_{i,t}}{OAS^{\text{def}}_{i,t-1}}
\end{equation}

\subsubsection{Re-estimate Merton fit on default component}

Run Stage B regression using $y^{\text{def}}_{i,t}$ as dependent variable:
\begin{equation}
y^{\text{def}}_{i,t} = \alpha + \beta_{\text{def}} \cdot [\lambda^{\text{Merton}}_i \cdot f^{\text{def}}_{DTS,t}] + \varepsilon_{i,t}
\end{equation}

where $f^{\text{def}}_{DTS,t}$ is the index-level default spread factor (after removing liquidity).

\textbf{Theory prediction}: $\beta_{\text{def}} \approx 1$ (Merton works for default component) and $R^2_{\text{def}} > R^2_{\text{total}}$ (less noise without liquidity shocks).

\subsubsection{Diagnostic patterns}

\begin{itemize}
\item \textbf{If $\beta_{\text{def}} \approx 1$ and $\beta_{\text{total}} \approx 1$}: Liquidity adjustment unnecessary—Merton works on total OAS.

\item \textbf{If $\beta_{\text{def}} \approx 1$ but $\beta_{\text{total}} < 1$}: Liquidity shocks dampen spread movements. Total OAS underestimates default sensitivity.

\item \textbf{If $\beta_{\text{def}} < \beta_{\text{total}}$}: Liquidity shocks amplify spread movements beyond Merton. Need separate liquidity beta.

\item \textbf{If improvement concentrated in illiquid bonds}: Signals that liquidity noise matters for small size, low turnover bonds but not for liquid benchmarks.
\end{itemize}

\subsubsection{Sector-specific liquidity effects}

Building on Stage 0 findings, test whether liquidity adjustment differentially affects sectors:
\begin{itemize}
\item Financials: Often more liquid (active trading, market makers); liquidity adjustment may be less important
\item Utilities: Often less liquid (buy-and-hold investors); liquidity adjustment may help
\item Energy: Liquidity varies with oil price cycles; may need regime-dependent liquidity treatment
\end{itemize}

\subsubsection{Practical decision rule}

\begin{mdframed}[backgroundcolor=blue!5!white,roundcorner=4pt]
\textbf{When to Use Liquidity Adjustment:}

\textbf{Liquid IG (BidAsk $<$ 50bps, Size $>$ \$1B):} 
\begin{itemize}
\item Use total OAS, liquidity adjustment negligible ($<$5bps typical)
\item Liquidity $R^2$ low, not worth complexity
\end{itemize}

\textbf{Illiquid IG (BidAsk $>$ 100bps, Size $<$ \$500M):}
\begin{itemize}
\item Liquidity premium 20--50bps, material fraction of spread
\item Consider decomposition for risk models
\end{itemize}

\textbf{HY (BidAsk typically $>$ 200bps):}
\begin{itemize}
\item Decompose into $OAS^{\text{def}}$ and $OAS^{\text{liq}}$
\item Use $\lambda^{\text{def}}$ from Merton for default component
\item Add separate $\lambda^{\text{liq}}$ empirically estimated (not theory-based)
\end{itemize}
\end{mdframed}

\subsubsection{Deliverables for D.3}

\begin{itemize}
\item \textbf{Table D.6:} Liquidity model estimates
\begin{itemize}
\item Cross-sectional regression: $\hat{\phi}_1$ (bid-ask), $\hat{\phi}_2$ (size), $\hat{\phi}_3$ (turnover), $\hat{\phi}_4$ (age)
\item $R^2$ by rating $\times$ maturity bucket
\item Typical liquidity premium: Mean $\widehat{OAS}^{\text{liq}}$ by bucket
\end{itemize}

\item \textbf{Table D.7:} Merton fit comparison
\begin{itemize}
\item Rows: Total OAS, Default component only
\item Columns: $\hat{\beta}$, $R^2$, RMSE
\item Test: Is $\beta_{\text{def}}$ closer to 1 than $\beta_{\text{total}}$?
\item Improvement: $\Delta R^2 = R^2_{\text{def}} - R^2_{\text{total}}$
\end{itemize}

\item \textbf{Table D.8:} Improvement by liquidity regime
\begin{itemize}
\item Split sample by liquidity quartiles (bid-ask or turnover)
\item For each quartile: $\beta_{\text{total}}$, $\beta_{\text{def}}$, $\Delta R^2$
\item Shows whether adjustment helps mainly for illiquid bonds
\end{itemize}

\item \textbf{Table D.9:} Sector-specific liquidity effects
\begin{itemize}
\item Rows: Sectors
\item Columns: $\beta_{\text{total}}$, $\beta_{\text{def}}$, $\Delta R^2$
\item Shows which sectors benefit most from liquidity decomposition
\end{itemize}

\item \textbf{Figure D.4:} Scatter plot: $\beta_{\text{def}} - \beta_{\text{total}}$ (y-axis) vs average bid-ask (x-axis)
\begin{itemize}
\item Each point: a rating $\times$ maturity bucket
\item Interpretation: Positive slope = illiquid bonds benefit more from decomposition
\end{itemize}

\item \textbf{Interpretation and recommendation}:
\begin{itemize}
\item For which bond types is liquidity adjustment material?
\item Does Merton work better on default component than total OAS?
\item Production implication: Use total OAS for liquid IG, decompose for HY?
\end{itemize}
\end{itemize}

\subsection{Summary of Stage D}

Stage D is \textbf{diagnostic}, not \textbf{decisional}. It answers:

\begin{enumerate}
\item \textbf{Where does Merton fail?} (Tails? Specific shock types? Liquidity-contaminated spreads?)

\item \textbf{How large are the failures?} (20\% effect in tails? 50\%? Negligible?)

\item \textbf{Are failures systematic?} (Always front-end IG? Always sector shocks? Always illiquid bonds?)

\item \textbf{Do failures relate to Stage 0 findings?} (Is Financial amplification driven by tails? sector shocks? liquidity?)
\end{enumerate}

\textbf{Use Stage D results in Stage E} to decide:
\begin{itemize}
\item If tail effects large: Add quantile-specific $\lambda$ for VaR/ES
\item If sector shocks deviate: Consider sector-factor adjustments beyond static sector dummies
\item If liquidity matters: Decompose OAS for HY, keep total for IG
\item If Financial amplification comes from specific channel: Design targeted adjustment
\end{itemize}

But don't let Stage D derail the main program. If Stages A--C showed Merton works well on average (possibly with sector adjustments from Stage 0), Stage D refinements are enhancements, not requirements.

\section{Stage E: Production Specification Selection}
\label{sec:stage-e}

\subsection{Objective}

Stage E selects the parsimonious production model that balances theoretical coherence, empirical fit, and implementation cost. 

\textbf{Key principle}: Use \emph{hierarchical testing} guided by theory. Stop at the simplest adequate model. Don't over-engineer.

\textbf{Philosophical stance}: Theory provides a strong prior. Only deviate when data strongly reject it. The burden of proof is on the more complex model.

\textbf{Integration of Stage 0 findings}: The three-pronged Stage 0 analysis (bucket-level, within-issuer, sector interaction) provides critical input for production specification. Sector adjustments identified in Stage 0 and validated in Stages B--C should be incorporated where statistically and economically significant.

\subsection{Decision framework: Hierarchical testing}

\begin{mdframed}[backgroundcolor=yellow!10!white,roundcorner=4pt]
\textbf{Level 1: Is standard DTS adequate?}

\textbf{Test}: Stage A, $F$-test for $H_0: \beta^{(1)} = \beta^{(2)} = \cdots = \beta^{(K)} = 1$

\textbf{Decision rule}:
\begin{itemize}
\item If fail to reject ($p > 0.10$): \textbf{Use standard DTS}. No adjustments needed. Done.
\item If reject: Proceed to Level 2
\end{itemize}

\textbf{Production spec}: $y_{i,t} \approx f_{DTS,t}$

\textbf{Parameters}: 0

\textbf{Implementation}: Trivial (already in all systems)

\vspace{0.3cm}

\textbf{Level 2: Does pure Merton suffice?}

\textbf{Test}: Stage B Specification B.1, $H_0: \beta_{\text{Merton}} = 1$

\textbf{Decision rule}:
\begin{itemize}
\item If $\hat{\beta}_{\text{Merton}} \in [0.9, 1.1]$ \textbf{and} $R^2_{\text{Merton}} > 0.9 \times R^2_{\text{buckets}}$ \textbf{and} Stage 0 sector interactions not significant:

\textbf{Use Pure Merton} (Tables~\ref{tab:merton-lambda-T}--\ref{tab:merton-lambda-s} from \cite{Wuebben2025}). Done.

\item If Stage 0 found significant sector effects: Proceed to Level 2a

\item If systematic bias ($\hat{\beta}_{\text{Merton}}$ outside $[0.9, 1.1]$): Proceed to Level 3

\item If poor fit ($R^2_{\text{Merton}} < 0.7 \times R^2_{\text{buckets}}$): Proceed to Level 4
\end{itemize}

\textbf{Production spec}: 
\begin{equation}
\lambda^{\text{prod}}_i = \lambda_T(T_i; 5y, s_i) \times \lambda_s(s_i; 100)
\end{equation}

\textbf{Parameters}: 0 (lookup tables from \cite{Wuebben2025})

\textbf{Implementation}: Read current $s_i, T_i$, look up $\lambda$ from tables or use closed-form $(s/100)^{-0.25}$

\vspace{0.3cm}

\textbf{Level 2a: Pure Merton with sector adjustments}

\textbf{Condition}: Stage 0 found significant sector effects (joint $F$-test $p < 0.05$, $\geq 1$ sector with $|\hat{\beta}_s| > 0.2$), confirmed in Stage B Specification B.3

\textbf{Test}: Stage B Specification B.3, joint test for sector coefficients

\textbf{Decision rule}:
\begin{itemize}
\item If $\hat{\beta}_0 \in [0.9, 1.1]$ and sector effects stable in Stage C: \textbf{Use Pure Merton with Sector Adjustments}. Done.
\item If sector effects regime-dependent (Stage C): Proceed to Level 2b
\item If $\hat{\beta}_0$ outside $[0.9, 1.1]$: Proceed to Level 3a
\end{itemize}

\textbf{Production spec}:
\begin{equation}
\lambda^{\text{prod}}_i = \lambda_T(T_i; 5y, s_i) \times \lambda_s(s_i; 100) \times (1 + \hat{\beta}_{\text{sector}_i})
\end{equation}

\textbf{Parameters}: $|S| - 1$ (sector dummies, typically 3--4)

\textbf{Implementation}: Read $s_i, T_i, \text{sector}_i$, apply Merton lookup, multiply by sector factor

\vspace{0.3cm}

\textbf{Level 2b: Pure Merton with regime-dependent sector adjustments}

\textbf{Condition}: Stage C found sector effects vary with VIX or spread level

\textbf{Production spec}:
\begin{equation}
\lambda^{\text{prod}}_{i,t} = \lambda^{\text{Merton}}_i \times (1 + \hat{\beta}_{s,0} + \hat{\beta}_{s,\text{VIX}} \cdot \text{VIX}_t)
\end{equation}

where $s = \text{sector}_i$.

\textbf{Parameters}: $2 \times (|S| - 1)$ (base and VIX coefficient per sector)

\textbf{Implementation}: Requires daily VIX feed, sector classification, lookup tables

\vspace{0.3cm}

\textbf{Level 3: Calibrated Merton (without sectors)}

\textbf{Condition}: Merton has right functional form but wrong scale, no significant sector effects

\textbf{Production spec}: 
\begin{equation}
\lambda^{\text{prod}}_i = c_0 \cdot \lambda_T(T_i; 5y, s_i) \times \lambda_s(s_i; 100)^{c_s}
\end{equation}

where:
\begin{itemize}
\item $c_0$: overall scaling factor (estimated from Stage B: $\hat{c}_0 = \hat{\beta}_{\text{Merton}}$)
\item $c_s$: adjust power-law exponent (theory suggests $-0.25$, data may differ)
\end{itemize}

\textbf{Estimation}: 
\begin{equation}
y_{i,t} = \alpha + \beta_0 [\lambda_T(T_i) \cdot f_{DTS,t}] + \beta_s [\lambda_s(s_i)^c \cdot f_{DTS,t}] + \varepsilon_{i,t}
\end{equation}

Grid search over $c \in [-0.5, 0]$ to maximize $R^2$, then calibrate $\beta_0, \beta_s$.

\textbf{Decision rule}:
\begin{itemize}
\item If $\hat{c}_0 \in [0.8, 1.2]$ and $\hat{c}_s \in [-0.35, -0.15]$: Theory approximately correct. Use calibrated version. Done.

\item If outside these ranges: Proceed to Level 4
\end{itemize}

\textbf{Parameters}: 2 ($c_0, c_s$)

\textbf{Implementation}: Requires one-time calibration using sample data, then fixed multipliers

\vspace{0.3cm}

\textbf{Level 3a: Calibrated Merton with sector adjustments}

\textbf{Condition}: Need both calibration (Level 3) and sector adjustments (Level 2a)

\textbf{Production spec}:
\begin{equation}
\lambda^{\text{prod}}_i = c_0 \cdot \lambda_T(T_i)^{c_T} \times \lambda_s(s_i)^{c_s} \times (1 + \hat{\beta}_{\text{sector}_i})
\end{equation}

\textbf{Parameters}: 2--3 (calibration) + 3--4 (sectors) = 5--7 total

\textbf{Implementation}: Moderate complexity; annual recalibration recommended

\vspace{0.3cm}

\textbf{Level 4: Full empirical (without sectors)}

\textbf{Condition}: Calibrated Merton inadequate, need unrestricted functional form, no significant sector effects

\textbf{Production spec}:
\begin{equation}
\lambda^{\text{prod}}_i = \exp(\beta_0 + \beta_M \log M_i + \beta_s \log s_i + \beta_{M^2} (\log M_i)^2 + \beta_{Ms} \log M_i \cdot \log s_i)
\end{equation}

Estimate via Stage B Specification B.4 (unrestricted).

\textbf{Decision rule}:
\begin{itemize}
\item If $R^2_{\text{unrestricted}} - R^2_{\text{calibrated}} > 0.05$: Empirical spec justified. Done.

\item Otherwise: Stay at Level 3 (principle of parsimony)
\end{itemize}

\textbf{Parameters}: 5--8 (maturity terms, spread terms, interactions)

\textbf{Implementation}: Requires regression estimation, periodic recalibration (annually)

\vspace{0.3cm}

\textbf{Level 4a: Full empirical with sector adjustments}

\textbf{Condition}: Need unrestricted functional form plus sector effects

\textbf{Production spec}:
\begin{equation}
\lambda^{\text{prod}}_i = \exp(\beta_0 + \beta_M \log M_i + \beta_s \log s_i + \beta_{M^2} (\log M_i)^2 + \beta_{Ms} \log M_i \cdot \log s_i + \sum_{\text{sector}} \beta_{\text{sec}})
\end{equation}

\textbf{Parameters}: 8--12 (maturity terms, spread terms, interactions, sector dummies)

\textbf{Implementation}: Complex; requires annual recalibration, sector classification

\vspace{0.3cm}

\textbf{Level 5: Time-varying (optional add-on to any level)}

\textbf{Condition}: Stage C showed significant instability ($p < 0.01$ in Chow test) \textbf{and} macro state changes $\lambda$ by $>$30\% during crises

\textbf{Production spec}:
\begin{equation}
\lambda^{\text{prod}}_{i,t} = \lambda^{\text{base}}_i \times \exp(\gamma_{\text{VIX}} \cdot \text{VIX}_t + \gamma_{\text{OAS}} \cdot \log(\text{OAS}_{index,t}))
\end{equation}

where $\lambda^{\text{base}}_i$ comes from Level 2, 2a, 3, 3a, 4, or 4a.

\textbf{Decision rule}:
\begin{itemize}
\item If improvement in crisis periods substantial (RMSE reduction $>$20\%) \textbf{and} operational complexity acceptable:

\textbf{Production spec} = \textbf{Time-varying} with base + 2 macro parameters

\item Otherwise: Use static $\lambda^{\text{base}}_i$ with manual crisis overlays
\end{itemize}

\textbf{Parameters}: Base parameters + 2 (macro state coefficients)

\textbf{Implementation}: Requires daily macro data feeds, dynamic recalculation
\end{mdframed}

\subsection{Summary of specification hierarchy}

\begin{table}[h]
\centering
\caption{Model Comparison: Parsimony vs Performance}
\label{tab:model-hierarchy}
\begin{tabular}{lcccp{4.5cm}}
\toprule
\textbf{Specification} & \textbf{Params} & \textbf{$R^2$} & \textbf{Impl.} & \textbf{Use When} \\
\midrule
Standard DTS & 0 & $R^2_0$ & Trivial & No cross-sectional variation \\
\midrule
Pure Merton & 0 & $R^2_M$ & Simple & Theory unbiased, no sector effects \\
\midrule
Pure Merton + Sectors & 3--4 & $R^2_{M+S}$ & Simple & Theory unbiased, stable sector effects \\
\midrule
Calibrated Merton & 2 & $R^2_M + \delta_1$ & Moderate & Theory right structure, needs scaling \\
\midrule
Calibrated + Sectors & 5--7 & $R^2_{M+S} + \delta_1$ & Moderate & Calibration needed, stable sector effects \\
\midrule
Empirical & 5--8 & $R^2_M + \delta_2$ & Complex & Theory inadequate, no sector effects \\
\midrule
Empirical + Sectors & 8--12 & $R^2_{M+S} + \delta_2$ & Complex & Theory inadequate, sector effects present \\
\midrule
Any + Time-varying & +2 & $+\delta_3$ & Very complex & Significant instability, large crisis effects \\
\bottomrule
\end{tabular}
\end{table}

\textbf{Expected incremental $R^2$ gains}:
\begin{itemize}
\item Pure Merton vs Standard DTS: +5--15\% (if Regime 2 prevalent)
\item Adding sector adjustments: +2--5\% (if Stage 0 found significant effects)
\item Calibrated vs Pure Merton: +1--3\% (scaling correction)
\item Empirical vs Calibrated: +2--5\% (if sectors or interactions matter beyond theory)
\item Time-varying vs Static: +1--2\% overall, +10--20\% in crises
\end{itemize}

\textbf{Implementation cost ranking}:
\begin{enumerate}
\item Standard DTS: Zero (already implemented)
\item Pure Merton: Low (lookup tables, one-time coding)
\item Pure Merton + Sectors: Low (add sector classification)
\item Calibrated Merton: Low-moderate (one-time calibration, quarterly review)
\item Calibrated + Sectors: Moderate (calibration plus sector tracking)
\item Empirical: Moderate-high (annual recalibration, more inputs)
\item Empirical + Sectors: High (many parameters, sector classification)
\item Time-varying: High (daily macro feeds, operational burden)
\end{enumerate}

\subsection{Out-of-sample validation}

For each candidate specification at its decision level, conduct rolling-window out-of-sample test.

\subsubsection{Methodology}

\textbf{Window structure}:
\begin{itemize}
\item Training window: 3 years (rolling)
\item Test window: 1 year (out-of-sample)
\item Roll forward by 1 year, repeat
\end{itemize}

\textbf{Procedure for each window}:
\begin{enumerate}
\item Estimate specification parameters using training data
\item Generate predictions $\hat{y}^{\text{model}}_{i,t} = \hat{\lambda}_{i,t-1} \cdot f_{DTS,t}$ for test window
\item Compute performance metrics
\item Roll window forward
\end{enumerate}

\subsubsection{Performance metrics}

\textbf{Metric 1: Forecast accuracy}
\begin{equation}
\text{RMSE}_{\text{OOS}} = \sqrt{\frac{1}{N_{\text{test}}} \sum_{i,t \in \text{test}} (y_{i,t} - \hat{y}_{i,t})^2}
\end{equation}

Compare across specifications. Lower is better.

\textbf{Metric 2: Out-of-sample $R^2$}
\begin{equation}
R^2_{\text{OOS}} = 1 - \frac{\sum (y_{i,t} - \hat{y}_{i,t})^2}{\sum (y_{i,t} - \bar{y})^2}
\end{equation}

Can be negative if model performs worse than mean. Compare to in-sample $R^2$ to assess overfitting.

\textbf{Metric 3: Hedge effectiveness}

Construct hedges based on model predictions:
\begin{itemize}
\item Portfolio: Long 1--2y BBB bonds
\item Hedge: Short $h_t = \lambda^{\text{model}}_{1y} / \lambda^{\text{model}}_{5y}$ units of 5y BBB bonds
\end{itemize}

Compute tracking error of hedged portfolio. Lower tracking error = better model.

\textbf{Metric 4: Sharpe ratio of mispricing signal}

Define mispricing:
\begin{equation}
\text{Mispricing}_{i,t} = y_{i,t} - \hat{y}^{\text{model}}_{i,t}
\end{equation}

If bond under-reacted last period (positive mispricing), predict mean-reversion. Trade on signal:
\begin{equation}
\text{Position}_{i,t+1} = -\text{sign}(\text{Mispricing}_{i,t})
\end{equation}

Better model $\Rightarrow$ stronger mean-reversion $\Rightarrow$ higher Sharpe on position.

\subsubsection{Regime-specific performance}

Evaluate each specification separately during:
\begin{enumerate}
\item \textbf{Normal periods}: VIX $<$ 20
\item \textbf{Stress periods}: VIX $\in [20, 30]$
\item \textbf{Crisis periods}: VIX $>$ 30
\end{enumerate}

\textbf{Key questions}:
\begin{itemize}
\item Does time-varying specification outperform static mainly in crises? If so, operational complexity may not be worth 98\% of the time.
\item Do sector adjustments help uniformly or mainly in specific regimes?
\item Does Financial sector amplification (if present) manifest primarily during stress?
\end{itemize}

\subsubsection{Sector-specific performance}

If Stage 0 found significant sector effects, evaluate performance by sector:

\begin{itemize}
\item Compare RMSE for Industrial, Financial, Utility, Energy separately
\item Test whether sector adjustments improve performance \emph{within} each sector
\item Identify if certain sectors drive overall improvement (e.g., sector adjustments help mainly for Financials)
\end{itemize}

\subsection{Recommended approach}

\begin{mdframed}[backgroundcolor=green!5!white,roundcorner=4pt]
\textbf{Decision Protocol:}

\textbf{Step 1}: Review Stage 0 findings
\begin{itemize}
\item Did within-issuer analysis validate Merton mechanism?
\item Did sector interaction analysis find significant sector effects?
\item Which sectors differ and by how much?
\end{itemize}

\textbf{Step 2}: Execute hierarchical testing (Levels 1--5)
\begin{itemize}
\item If Stage 0 found no sector effects: Test Levels 1 $\to$ 2 $\to$ 3 $\to$ 4
\item If Stage 0 found sector effects: Test Levels 1 $\to$ 2a $\to$ 3a $\to$ 4a
\item At each level, conduct stopping criterion test
\end{itemize}

\textbf{Step 3}: At the level where you stop, conduct out-of-sample validation against:
\begin{itemize}
\item Previous level (simpler)
\item Next level (more complex)
\end{itemize}

\textbf{Step 4}: Apply parsimony rule
\begin{itemize}
\item If next level improves out-of-sample RMSE by $<$5\%: Stick with current level
\item If next level improves out-of-sample RMSE by $>$10\%: Adopt more complex model
\item If improvement 5--10\%: Consider implementation costs
\end{itemize}

\textbf{Step 5}: Document production specification with:
\begin{itemize}
\item Parameter values (if any)
\item Sector adjustments (if any)
\item Implementation pseudo-code
\item Recalibration frequency
\item Performance benchmarks by regime and sector
\end{itemize}

\textbf{Guiding principle}: Occam's Razor—prefer simplest model with adequate fit. A 2-parameter model with $R^2=0.75$ beats a 20-parameter model with $R^2=0.78$.
\end{mdframed}

\subsection{Deliverables for Stage E}

\begin{itemize}
\item \textbf{Table E.1:} Hierarchical test results
\begin{itemize}
\item Rows: Levels 1, 2, 2a, 3, 3a, 4, 4a, 5
\item Columns: Test statistic, $p$-value, Decision (PASS / Proceed to next level / N/A)
\item Mark the stopping level
\item Note which path taken (with or without sectors)
\end{itemize}

\item \textbf{Table E.2:} Model comparison (all candidate specs)
\begin{itemize}
\item Rows: Standard DTS, Pure Merton, Pure Merton + Sectors, Calibrated, Calibrated + Sectors, Empirical, Empirical + Sectors, Time-varying variants
\item Columns: Parameters, In-sample $R^2$, OOS $R^2$, OOS RMSE, Hedge tracking error, Sharpe ratio
\item Highlight recommended specification
\end{itemize}

\item \textbf{Table E.3:} Performance by regime
\begin{itemize}
\item Panel A: Normal periods (VIX $<$ 20)
\item Panel B: Stress periods (VIX 20--30)
\item Panel C: Crisis periods (VIX $>$ 30)
\item For each panel: OOS $R^2$ and RMSE for each specification
\item Shows where complexity pays off
\end{itemize}

\item \textbf{Table E.4:} Performance by sector (if sector adjustments used)
\begin{itemize}
\item Rows: Industrial, Financial, Utility, Energy
\item Columns: RMSE (no sector adj), RMSE (with sector adj), Improvement
\item Shows which sectors drive benefit of sector adjustments
\end{itemize}

\item \textbf{Table E.5:} Recommended production specification
\begin{itemize}
\item Specification name and level
\item Parameter estimates with standard errors
\item Sector adjustment factors (if applicable)
\item Implementation formula
\item Recalibration frequency
\item Expected $R^2$ improvement over baseline
\item Complexity rating (Low / Moderate / High)
\end{itemize}

\item \textbf{Figure E.1:} Out-of-sample $R^2$ over rolling windows
\begin{itemize}
\item X-axis: Test window start date
\item Y-axis: OOS $R^2$
\item Multiple lines: Key specifications (Standard DTS, Pure Merton, Merton + Sectors, Calibrated, etc.)
\item Shade crisis periods
\item Shows stability and relative performance over time
\end{itemize}

\item \textbf{Figure E.2:} Forecast error distribution
\begin{itemize}
\item Histogram of $(y_{i,t} - \hat{y}_{i,t})$ for recommended spec
\item Overlay: Normal distribution with same mean/variance
\item Q-Q plot in corner panel
\item Check for unmodeled fat tails or asymmetry
\end{itemize}

\item \textbf{Figure E.3:} Scatter: Predicted vs actual spread changes
\begin{itemize}
\item X-axis: $\hat{y}_{i,t}$ (model prediction)
\item Y-axis: $y_{i,t}$ (actual)
\item 45-degree line
\item Color-code by regime (IG-narrow, IG-wide, HY, distressed)
\item Shows where model works best/worst
\end{itemize}

\item \textbf{Figure E.4:} Sector-specific performance (if applicable)
\begin{itemize}
\item Bar chart: RMSE by sector, with and without sector adjustments
\item Shows incremental benefit of sector factors
\end{itemize}

\item \textbf{Implementation blueprint (5--7 pages):}

\begin{enumerate}
\item \textbf{Algorithmic steps}:
\begin{itemize}
\item Input data required (bond OAS, maturity, rating, sector; macro state if time-varying)
\item Step-by-step calculation of $\lambda_i$ or $\lambda_{i,t}$
\item Application of sector adjustments (if applicable)
\item Computation of adjusted DTS: $DTS^*_{i,t} = \lambda_i \cdot DTS_{i,t}$
\end{itemize}

\item \textbf{Pseudo-code / function definitions}:

\begin{verbatim}
function compute_lambda(OAS, Maturity, Sector):
    # Pure Merton + Sectors example
    lambda_T = lookup_maturity_adjustment(Maturity, OAS)
    lambda_s = (OAS / 100) ^ (-0.25)
    lambda_base = lambda_T * lambda_s
    
    # Apply sector adjustment (from Stage 0/B)
    sector_factor = get_sector_factor(Sector)
    # e.g., Financial: 1.33, Utility: 1.0, Energy: 1.0
    
    lambda = lambda_base * sector_factor
    return lambda

function lookup_maturity_adjustment(T, s):
    # Bilinear interpolation on Table 3.1
    # Returns lambda_T(T; 5y, s)
    ...

function get_sector_factor(Sector):
    # From Stage 0 sector interaction estimates
    factors = {
        'Industrial': 1.00,  # reference
        'Financial': 1.33,   # beta_Financial = 0.33
        'Utility': 0.98,     # beta_Utility = -0.02
        'Energy': 1.15       # beta_Energy = 0.15
    }
    return factors.get(Sector, 1.00)
\end{verbatim}

\item \textbf{Lookup tables}: Full tables if using Pure Merton (or reference to theory paper), plus sector adjustment factors

\item \textbf{Recalibration protocol}:
\begin{itemize}
\item Frequency: Quarterly review for Pure Merton (check if spreads/maturities still in table range); Annual for Calibrated/Empirical; Semi-annual for sector factors
\item Procedure: Re-estimate using trailing 3-year window, compare to current parameters
\item Trigger for update: If new estimates differ by $>$20\%, adopt new parameters
\item Sector factor review: Re-run Stage 0 sector interaction test annually
\item Documentation: Maintain version history of parameter changes
\end{itemize}

\item \textbf{Edge case handling}:
\begin{itemize}
\item Very short maturity ($<$ 6 months): Use 6-month $\lambda$ as floor, don't extrapolate
\item Distressed spreads ($>$ 2000 bps): Cap $\lambda$ at 1.2 (proportionality improves in extreme stress)
\item Missing sector: Use aggregate cross-sector average (no sector adjustment)
\item New issues ($<$ 3 months old): Apply new issue adjustment $\gamma_{\text{new}}$ if Stage B found significant effect
\item Sector reclassification: Use new sector's adjustment factor immediately
\end{itemize}

\item \textbf{Integration with existing systems}:
\begin{itemize}
\item Risk models: Replace $DTS_{i,t}$ with $DTS^*_{i,t} = \lambda_i \cdot DTS_{i,t}$ in portfolio risk calculations
\item Attribution: Decompose factor return into $f_{DTS,t}$, cross-sectional $\lambda_i$ effect, and sector effect
\item Relative value: Flag bonds with large deviations from $\lambda$-adjusted fair value
\item Sector allocation: Account for differential DTS sensitivity when sizing sector bets
\end{itemize}

\item \textbf{Performance monitoring}:
\begin{itemize}
\item Track out-of-sample $R^2$ monthly, by sector
\item Alert if $R^2$ drops below threshold (e.g., $<$ 50\% of historical average)
\item Compare to benchmark (Standard DTS) quarterly
\item Monitor sector factor stability (flag if sector $\hat{\beta}_s$ changes by $>$0.1)
\end{itemize}
\end{enumerate}

\item \textbf{Comparative performance analysis (3--4 pages)}:

\begin{enumerate}
\item \textbf{Executive summary}: How much does recommended spec improve over baseline DTS?
\begin{itemize}
\item RMSE reduction: e.g., 25\% lower forecast errors
\item $R^2$ improvement: e.g., from 0.55 to 0.72
\item Hedge tracking error: e.g., 30 bps lower annualized
\item Contribution from Merton adjustment vs sector factors
\end{itemize}

\item \textbf{Which regimes see largest gains?}
\begin{itemize}
\item IG with wide maturity range: 40\% RMSE reduction (Regime 2 targeted)
\item IG with narrow maturity: 10\% reduction (smaller benefit)
\item HY: 15\% reduction (smaller maturity effects but credit quality variation)
\item Financial sector: Additional 10\% reduction from sector adjustment
\end{itemize}

\item \textbf{Economic value}: Translate statistical improvements to portfolio management gains
\begin{itemize}
\item \textbf{Example 1: Hedging efficiency}

``Portfolio manager long \$100M of 1-year BBB industrials, wants to hedge with 5-year BBB. 

Standard DTS: Hedge ratio = 1.0, residual tracking error = 120 bps/year

Merton-adjusted: Hedge ratio = 3.2, residual tracking error = 40 bps/year

Value: 80 bps lower tracking error, equivalent to \$800K lower unexpected P\&L volatility''

\item \textbf{Example 2: Relative value signals}

``Identifying rich/cheap bonds within issuer capital structure. 

Standard DTS: Assumes all bonds move proportionally, misses 300--500\% cross-maturity differences.

Merton-adjusted: Properly scales for maturity, identifies mispricings averaging 15 bps.

Trade: Long cheap (e.g. 10y), short rich (e.g. 1y), earn 30 bps as convergence occurs over 6 months.''

\item \textbf{Example 3: Sector allocation with DTS adjustment}

``Allocating risk budget across sectors.

Standard DTS: Treats Financial and Industrial DTS identically.

With sector adjustment: Recognizes Financial has 33\% amplified sensitivity.

Result: For same DTS exposure, reduce Financial allocation by 25\% to equalize true risk. Avoids unintended concentration in high-sensitivity sector.''

\item \textbf{Example 4: Portfolio construction}

``Constructing credit barbell (short + long maturity, avoid intermediate).

Standard DTS: Treats all durations equally, over-concentrates DTS risk in short end.

Merton-adjusted: Recognizes front-end 3--4$\times$ more sensitive, rebalances to equalize risk contribution.

Result: More stable returns, 25\% reduction in unexpected drawdowns during spread volatility spikes.''
\end{itemize}

\item \textbf{Limitations and caveats}:
\begin{itemize}
\item Out-of-sample performance may degrade if regime shifts outside historical experience
\item Liquidity crises (2008-style) may break Merton assumptions
\item Model requires clean data (OAS, maturity, sector, liquidity proxies)—quality control critical
\item Parameter drift possible—requires monitoring and periodic recalibration
\item Sector factors estimated on historical data may not capture future sector dynamics
\end{itemize}

\item \textbf{Sensitivity to implementation choices}:
\begin{itemize}
\item Bucket definitions: Finer buckets improve fit but reduce sample size per bucket
\item Frequency: Daily more noisy, monthly smoother but lags; weekly optimal balance
\item Clustering: Week clustering adequate, bond clustering more conservative but similar results
\item Sector classification: Results robust to minor reclassifications; major changes (e.g., splitting Financials into Banks/Insurance) may require re-estimation
\end{itemize}
\end{enumerate}
\end{itemize}


\section{Summary of Research Tasks}
\label{sec:summary-tasks}

This section provides an actionable checklist for executing the research program.

\begin{mdframed}[backgroundcolor=blue!3!white,roundcorner=4pt,innerleftmargin=10pt]
\textbf{Phase 0: Data and Universe Construction}

\begin{enumerate}[label=\arabic*., leftmargin=*]
\item Extract full histories of bond-level OAS, OASD, ratings, maturities, sectors, liquidity proxies for all constituents of Bloomberg Barclays U.S.\ Corporate IG and HY indices (2013--2025)

\item Construct weekly observation grid (Friday close or last trading day)

\item Apply bond-level filters: seniority, maturity $\geq$ 1y, liquidity $\geq$ 5 trades/month, price 30--150\% par

\item \textbf{Construct issuer identifiers} using Ultimate Parent + Seniority matching (Section~\ref{sec:issuer-identification})
\begin{itemize}
\item Validate for top 50 issuers by amount outstanding
\item Document edge cases (holding company vs subsidiary, secured vs unsecured)
\item Cross-check against rating agency issuer hierarchies
\end{itemize}

\item Compute $y_{i,t} = \Delta OAS_{i,t} / OAS_{i,t-1}$ and $f^{(U)}_{DTS,t} = \Delta OAS^{(U)}_t / OAS^{(U)}_{t-1}$

\item Document sample size: bonds per week, issuers with 2+ bonds, total issuer-weeks, bonds per sector

\item \textbf{Deliverable}: Sample construction memo (3--5 pages) + summary statistics table
\end{enumerate}

\medskip
\textbf{Phase 0.5: Structural Model Setup}

\begin{enumerate}[label=\arabic*., leftmargin=*, resume]
\item Implement Merton model functions: $\lambda_T(T; s)$ and $\lambda_s(s)$ using formulas from theory paper

\item Generate lookup tables (Tables~\ref{tab:merton-lambda-T}--\ref{tab:merton-lambda-s}) if not already available

\item For each bond-date, compute $\lambda^{\text{Merton}}_{i,t} = \lambda_T(T_i; 5y, s_i) \times \lambda_s(s_i; 100)$

\item Validate: Spot-check 10--20 bonds to ensure $\lambda^{\text{Merton}}$ values sensible (e.g., 1y IG $\lambda \approx 3$, 10y IG $\lambda \approx 0.6$)

\item \textbf{Deliverable}: Merton calculation functions (code) + validation memo
\end{enumerate}

\medskip
\textbf{Phase 1: Stage 0 --- Bucket-Level Analysis}

\begin{enumerate}[label=\arabic*., leftmargin=*, start=13]
\item Define buckets: Rating (AAA/AA, A, BBB for IG; BB, B, CCC for HY) $\times$ Maturity (1--2y, 2--3y, 3--5y, 5--7y, 7--10y, 10y+) $\times$ Sector (Industrial, Financial, Utility, Energy)

\item For each bucket $k$, estimate: $y_{i,t} = \alpha^{(k)} + \beta^{(k)} f_{DTS,t} + \varepsilon^{(k)}_{i,t}$

\item Compute $\lambda^{\text{Merton}}_k$ for each bucket using median characteristics

\item Statistical tests: $t$-test for $H_0: \beta^{(k)} = \lambda^{\text{Merton}}_k$, Spearman correlation (maturity vs $\lambda$), regime pattern test

\item \textbf{Deliverables}: Tables 0.1--0.2, Figures 0.1--0.3
\end{enumerate}

\medskip
\textbf{Phase 1.5: Stage 0 --- Within-Issuer Analysis}

\begin{enumerate}[label=\arabic*., leftmargin=*, start=18]
\item Identify all issuer-weeks with $\geq$ 2 bonds outstanding (same Ultimate Parent + Seniority, maturity dispersion $\geq$ 2 years)

\item For each issuer-week, estimate within-issuer regression:
\begin{equation*}
\frac{\Delta s_{ij,t}}{s_{ij,t-1}} = \alpha_{i,t} + \beta_{i,t} \cdot \lambda^{\text{Merton}}_{ij,t} + \epsilon_{ij,t}
\end{equation*}

\item Pool across issuer-weeks using inverse-variance weighting

\item Diagnostic tests: Spread level effect, maturity dispersion effect, crisis vs normal comparison

\item \textbf{Deliverables}: Tables 0.3--0.4, Figures 0.4--0.6, case studies for major issuers
\end{enumerate}

\medskip
\textbf{Phase 1.6: Stage 0 --- Sector Interaction Analysis}

\begin{enumerate}[label=\arabic*., leftmargin=*, start=23]
\item Estimate sector interaction specification:
\begin{equation*}
y_{i,t} = \alpha + \beta_0 [\lambda^{\text{Merton}}_{i,t} \cdot f_{DTS,t}] + \sum_{s \in S} \beta_s \cdot \mathbbm{1}_{\{i \in s\}} \cdot [\lambda^{\text{Merton}}_{i,t} \cdot f_{DTS,t}] + \varepsilon_{i,t}
\end{equation*}

\item Statistical tests: Joint $F$-test for all $\beta_s = 0$, individual $t$-tests, pairwise comparisons

\item If significant sector effects found, estimate regime-dependent specification with VIX interactions

\item \textbf{Deliverables}: Tables 0.5--0.6, Figures 0.7--0.9
\end{enumerate}

\medskip
\textbf{Phase 1.7: Stage 0 --- Synthesis}

\begin{enumerate}[label=\arabic*., leftmargin=*, start=27]
\item Apply synthesis framework (Section~\ref{sec:stage0-synthesis}): Answer Questions 1--3

\item Resolve any apparent contradictions between approaches

\item Determine path for subsequent stages (Paths 1--5 from Section~\ref{sec:stage0-decision})

\item \textbf{Deliverables}: Table 0.7 (synthesis summary), Figure 0.10, written summary (5--7 pages)

\item \textbf{Decision point}: Does Merton provide adequate baseline? Are sector adjustments needed?
\end{enumerate}

\medskip
\textbf{Phase 2: Stage A --- Establish Variation}

\begin{enumerate}[label=\arabic*., leftmargin=*, start=32]
\item Estimate bucket-level regressions (Spec A.1): $y_{i,t} = \alpha^{(k)} + \beta^{(k)} f_{DTS,t} + \varepsilon^{(k)}_{i,t}$

\item $F$-test for equality: $H_0: \beta^{(1)} = \cdots = \beta^{(K)}$

\item Two-step estimation (Spec A.2): Rolling window bond-specific betas, then cross-sectional regression on $(M_i, s_i, M_i^2, M_i \cdot s_i)$

\item \textbf{Deliverables}: Tables A.1--A.3, Figures A.1--A.2, diagnostic summary (2 pages)

\item \textbf{Decision point}: Is variation significant? (If $F$-test $p > 0.10$, stop—standard DTS adequate)
\end{enumerate}

\medskip
\textbf{Phase 3: Stage B --- Test Theory}

\begin{enumerate}[label=\arabic*., leftmargin=*, start=37]
\item Estimate constrained Merton (Spec B.1): $y_{i,t} = \alpha + \beta_{\text{Merton}} \cdot [\lambda^{\text{Merton}}_{i,t} \cdot f_{DTS,t}] + \varepsilon_{i,t}$

\item Test: Wald $H_0: \beta_{\text{Merton}} = 1$

\item Estimate decomposed Merton (Spec B.2): Separate $\beta_T$ and $\beta_s$

\item Estimate Merton with sector adjustments (Spec B.3): Include sector interaction terms from Stage 0

\item Estimate unrestricted (Spec B.4): Flexible $\lambda(M_i, s_i, \text{sector}_i)$

\item Construct Theory vs Reality table: Empirical $\hat{\beta}^{(k)}$ vs Theoretical $\lambda^{\text{Merton}}_k$ for all buckets

\item Compare $R^2$: Does $R^2_{\text{Merton}} > 0.9 \times R^2_{\text{buckets}}$? Does adding sectors help?

\item \textbf{Deliverables}: Tables B.1--B.3, Figures B.1--B.3, diagnostic summary (3--4 pages)

\item \textbf{Decision point}: Pure Merton / Merton + Sectors / Calibrated / Unrestricted? Follow decision tree from Section~\ref{sec:stage-b}
\end{enumerate}

\medskip
\textbf{Phase 4: Stage C --- Test Stability}

\begin{enumerate}[label=\arabic*., leftmargin=*, start=46]
\item Divide sample into non-overlapping 1-year windows

\item For each window, estimate: $y_{i,t} = \alpha_w + \beta_w \cdot [\lambda^{\text{Merton}}_{i,t} \cdot f_{DTS,t}] + \varepsilon_{i,t}$

\item Chow test: $H_0: \beta_1 = \beta_2 = \cdots = \beta_W$

\item If sector adjustments used, test stability of sector coefficients across windows

\item \textbf{If stable} ($p > 0.10$): Stop Stage C, use static $\lambda$ (with or without sector adjustments)

\item \textbf{If unstable}: Estimate macro driver regression: $\hat{\beta}_w = \delta_0 + \delta_{\text{VIX}} \overline{\text{VIX}}_w + \delta_{\text{OAS}} \log(\overline{\text{OAS}}_w) + \eta_w$

\item Assess economic significance: Does macro state change $\lambda$ by $>$20\%?

\item Maturity-specific and sector-specific time-variation tests

\item \textbf{Deliverables}: Tables C.1--C.5, Figures C.1--C.5, summary (3--4 pages)

\item \textbf{Decision point}: Static sufficient / Time-varying necessary? Follow thresholds from Section~\ref{sec:stage-c}
\end{enumerate}

\medskip
\textbf{Phase 5: Stage D --- Robustness}

\begin{enumerate}[label=\arabic*., leftmargin=*, start=56]
\item Quantile regression (D.1): Estimate $Q_{\tau}(y_{i,t}) = \alpha_{\tau} + \beta_{\tau} \cdot [\lambda^{\text{Merton}}_i \cdot f_{DTS,t}]$ for $\tau \in \{0.05, 0.50, 0.95\}$

\item Sector-specific tail behavior: Estimate quantile regressions by sector

\item Test: Is $\beta_{0.05}$ significantly different from $\beta_{0.50}$? Do sectors differ in tail behavior?

\item Shock decomposition (D.2): Construct orthogonalized factors $f^{(G)}_t, f^{(S)}_{s,t}, f^{(I)}_{i,t}$

\item Estimate: $y_{i,t} = \beta^{(G)} [\lambda^{\text{Merton}}_i f^{(G)}_t] + \beta^{(S)} [\lambda^{\text{Merton}}_i f^{(S)}_{s,t}] + \beta^{(I)} [\lambda^{\text{Merton}}_i f^{(I)}_{i,t}] + \varepsilon_{i,t}$

\item Sector-specific shock decomposition: Which shock types drive sector effects?

\item Test: $H_0: \beta^{(G)} = \beta^{(S)} = \beta^{(I)}$

\item Liquidity adjustment (D.3): Estimate $s^{\text{liq}}_{i,t}$ cross-sectionally, define $OAS^{\text{def}}_{i,t} = OAS_{i,t} - \widehat{OAS}^{\text{liq}}_{i,t}$

\item Re-estimate Stage B using $y^{\text{def}}_{i,t}$, compare $\beta_{\text{def}}$ to $\beta_{\text{total}}$

\item Sector-specific liquidity effects

\item \textbf{Deliverables}: Tables D.1--D.9, Figures D.1--D.4, interpretation notes

\item \textbf{Use in Stage E}: Identify where/why Merton fails, inform production spec refinements
\end{enumerate}

\medskip
\textbf{Phase 6: Stage E --- Production Specification}

\begin{enumerate}[label=\arabic*., leftmargin=*, start=68]
\item Review Stage 0 findings on sector effects

\item Execute hierarchical testing: Levels 1--5 (with or without sector track) from Section~\ref{sec:stage-e}

\item At each level, conduct stopping criterion test

\item Identify stopping level (e.g., Level 2a = Pure Merton + Sectors)

\item Rolling-window out-of-sample validation: 3-year training, 1-year test, roll forward

\item Compute performance metrics: OOS $R^2$, RMSE, hedge tracking error, Sharpe ratio of mispricing signal

\item Regime-specific and sector-specific performance evaluation

\item Compare recommended spec to previous and next levels

\item If next level improves RMSE by $<$5\%, stick with current (parsimony); if $>$10\%, adopt next level

\item Document production specification: Formula, parameters, sector factors, implementation, recalibration protocol, edge cases

\item Create implementation blueprint: Pseudo-code, lookup tables, sector adjustment factors, system integration guide

\item Comparative performance analysis: Economic value examples (hedging, relative value, sector allocation, portfolio construction)

\item \textbf{Deliverables}: Tables E.1--E.5, Figures E.1--E.4, implementation blueprint (5--7 pages), comparative analysis (3--4 pages), final recommendation template
\end{enumerate}

\medskip
\textbf{Phase 7: Final Report and Presentation}

\begin{enumerate}[label=\arabic*., leftmargin=*, start=81]
\item Compile all stage deliverables into comprehensive report (60--90 pages)

\item Executive summary (3--5 pages): Key findings, recommended spec, expected improvements, sector adjustment factors

\item Technical appendix: Detailed methodology, robustness checks, sensitivity analyses

\item Create presentation deck for stakeholders (20--30 slides):
\begin{itemize}
\item Motivation: Why DTS adjustments needed
\item Theory: Merton predictions in plain language
\item Stage 0 findings: Three-pronged validation results, sector effects
\item Results: What we found (stage by stage)
\item Recommendation: Production spec with performance comparison
\item Implementation: Roadmap and timeline
\end{itemize}

\item Supplementary materials: Code repository, data dictionary, replication instructions
\end{enumerate}
\end{mdframed}

\subsection{Resource requirements}

\begin{table}[h]
\centering
\begin{tabular}{lrr}
\toprule
\textbf{Analysis} & \textbf{Computation Time} & \textbf{Analyst Time} \\
\midrule
Phase 0: Data construction & 4--8 hours & 3--5 days \\
Phase 0.5: Merton setup & 1--2 hours & 1--2 days \\
\midrule
Phase 1: Bucket-level & 2--4 hours & 2--3 days \\
Phase 1.5: Within-issuer & 4--8 hours & 2--3 days \\
Phase 1.6: Sector interaction & 1--2 hours & 1--2 days \\
Phase 1.7: Synthesis & --- & 2--3 days \\
\midrule
Phase 2: Stage A & 2--4 hours & 2--3 days \\
Phase 3: Stage B & 2--4 hours & 3--4 days \\
Phase 4: Stage C & 2--4 hours & 2--3 days \\
Phase 5: Stage D & 4--8 hours & 3--4 days \\
Phase 6: Stage E & 4--8 hours & 4--5 days \\
Phase 7: Final report & --- & 5--7 days \\
\midrule
\textbf{Total} & \textbf{26--52 hours} & \textbf{30--44 days} \\
\bottomrule
\end{tabular}
\caption{Estimated resource requirements for complete research program}
\end{table}

\textbf{Notes}:
\begin{itemize}
\item Computation time assumes modern hardware (16+ cores, 64GB RAM) and efficient code
\item Within-issuer analysis most computationally intensive (225,000+ separate regressions)
\item Analyst time includes data preparation, code development, quality checks, interpretation, and documentation
\item Parallelization can reduce wall-clock time significantly
\item Total timeline: 6--10 weeks with dedicated analyst
\end{itemize}


\section{Conclusion}

This research program provides a comprehensive framework for enhancing the Duration-Times-Spread (DTS) model through theory-guided empirical estimation. The key innovations relative to standard approaches include:

\subsection{Methodological contributions}

\begin{enumerate}
\item \textbf{Three-pronged Stage 0 validation}: Rather than relying on a single approach, we test Merton predictions using bucket-level analysis (aggregate patterns), within-issuer analysis (pure maturity effects), and sector interaction analysis (formal inference on industry differences). This triangulation provides robust evidence for subsequent stages.

\item \textbf{Sequential testing with explicit decision points}: Rather than running all analyses in parallel, we proceed hierarchically with clear stopping rules. This prevents over-engineering and respects the principle that simpler models are preferable when adequate.

\item \textbf{Theory as baseline, not constraint}: Merton predictions provide strong priors that sharpen empirical tests and reduce parameter space, but we systematically evaluate when theory fails and data require more flexibility.

\item \textbf{Separation of existence and explanation}: Stage A establishes \emph{that} DTS betas vary before Stage B tests \emph{whether} theory explains it. This clarifies where standard DTS fails versus where our adjustments add value.

\item \textbf{Formal sector effect testing}: Moving beyond descriptive sector comparisons to formal statistical inference about whether Financial, Utility, and Energy bonds systematically differ in DTS sensitivity. This determines whether production models require sector-specific adjustments.

\item \textbf{Hierarchical model selection with sector track}: The Level 1--5 framework guided by theory now includes explicit paths for incorporating sector adjustments (Levels 2a, 3a, 4a). We stop at the simplest adequate model rather than maximizing in-sample fit.
\end{enumerate}

\subsection{Practical contributions}

\begin{enumerate}
\item \textbf{Implementable specifications}: Every stage produces actionable outputs—lookup tables, regression coefficients, sector adjustment factors, implementation pseudo-code—not just statistical tests.

\item \textbf{Economic significance thresholds}: We don't just test statistical significance but also whether effects are large enough to matter for portfolio management (e.g., 20\% threshold for time-variation, 0.2 threshold for sector effects).

\item \textbf{Sector-aware production specifications}: When Stage 0 identifies significant sector effects (e.g., Financial amplification), these are carried through to production with explicit adjustment factors and monitoring protocols.

\item \textbf{Production readiness}: Stage E delivers a complete implementation blueprint with recalibration protocols, edge case handling, sector classification requirements, and performance monitoring.

\item \textbf{Regime awareness}: Rather than assuming one-size-fits-all, we explicitly acknowledge that DTS adjustments are most critical in Regime 2 (IG with wide maturity range) and that sector effects may be regime-dependent.
\end{enumerate}

\subsection{Academic contributions}

\begin{enumerate}
\item \textbf{First comprehensive empirical test with proper identification}: This program provides the first market-wide test of structural model spread dynamics using modern bond data (2013--2025) with proper within-issuer identification of pure maturity effects.

\item \textbf{Theory-data dialogue}: Rather than pure theory (no empirics) or pure empirics (ignoring theory), we create a dialogue where theory guides specification and data discipline theory.

\item \textbf{Quantification of regime and sector effects}: We move beyond qualitative statements about when proportional spread movements fail to precise estimates of elasticity ratios by spread level, maturity, and sector.

\item \textbf{Decomposition of failures}: Stage D systematically diagnoses \emph{why} theory fails (tails? liquidity? specific shocks? specific sectors?) rather than just documenting that it does.

\item \textbf{Sector mechanism identification}: The combination of Stage 0 sector interactions with Stage D shock decomposition reveals \emph{why} sector effects exist (correlation risk? liquidity? information asymmetry?), informing both academic understanding and production model design.
\end{enumerate}

\subsection{Expected outcomes}

Based on prior research and the theoretical framework, we anticipate:

\begin{enumerate}
\item \textbf{Stage 0 will validate Merton for IG cross-maturity effects}: The structural model's prediction \cite{Wuebben2025} that 1-year IG bonds have 3--4$\times$ higher sensitivity than 10-year bonds should be confirmed by both bucket-level and within-issuer tests.

\item \textbf{Stage 0 will identify Financial sector amplification}: We expect $\hat{\beta}_{\text{Financial}} > 0$ (likely 0.2--0.4), indicating Financials have 20--40\% higher DTS sensitivity than Industrials, driven by correlation risk and regulatory capital effects.

\item \textbf{Stage A will show significant variation}: $F$-tests will reject beta equality with $p < 0.01$, motivating the need for adjustments.

\item \textbf{Stage B will show Merton explains most IG variation}: For investment-grade with wide maturity dispersion, $R^2_{\text{Merton}} > 0.85 \times R^2_{\text{buckets}}$ likely. Adding sector adjustments will provide incremental improvement ($R^2$ gain of 2--5\%).

\item \textbf{Stage C will find static relationships largely sufficient}: Time-variation likely marginal except in rare crises. Sector effects likely stable in normal periods but may amplify during stress (particularly for Financials).

\item \textbf{Stage D will identify Financial amplification sources}: Quantile regressions and shock decomposition will reveal whether Financial sector effects come from tail events, sector-wide contagion, or baseline amplification.

\item \textbf{Stage E will recommend Calibrated Merton with Sector Adjustments}: Most likely outcome is Level 2a or 3a, providing 25--35\% RMSE improvement over Standard DTS with moderate implementation complexity.
\end{enumerate}

\subsection{Broader implications}

This research program has implications beyond corporate credit DTS:

\begin{enumerate}
\item \textbf{Methodology template}: The three-pronged validation framework (aggregate, within-unit, interaction effects) applies to other markets where structural models exist and heterogeneity may be sector-specific.

\item \textbf{Theory-practice integration}: Demonstrates how academic models can inform practitioner tools when subjected to rigorous empirical validation rather than blind application.

\item \textbf{Risk model enhancement}: Improved DTS specifications with sector adjustments translate directly to better portfolio risk estimates, hedge ratios, and stress test scenarios.

\item \textbf{Alpha generation}: Properly scaled relative-value signals (rich/cheap within issuer capital structures, accounting for sector effects) can generate trading alpha when market prices deviate from theory-adjusted fair values.

\item \textbf{Sector allocation}: Understanding differential DTS sensitivity across sectors enables more accurate risk budgeting and avoids unintended concentration in high-sensitivity sectors.
\end{enumerate}

\subsection{Limitations and extensions}

Several important limitations should be acknowledged:

\begin{enumerate}
\item \textbf{Merton is approximate}: Even if empirically validated, the structural model rests on simplifying assumptions (log-normal firm value, constant parameters, no jumps). Stage D robustness checks partially address this.

\item \textbf{Parameter instability}: Coefficients estimated on 2013--2025 data may not hold if credit cycle or market structure changes. Sector effects particularly may evolve with regulatory changes or market structure shifts.

\item \textbf{Data quality dependence}: Results critically depend on accurate OAS, maturity, sector classification, and liquidity data. Sector reclassifications require monitoring.

\item \textbf{Sample period limitations}: 2013--2025 includes only one true crisis (COVID 2020). Performance in 2008-style liquidity breakdown uncertain. Sector effects during such crises may differ.

\item \textbf{Sector classification granularity}: Current framework uses 4 broad sectors. Finer distinctions (e.g., Banks vs Insurance, Exploration vs Integrated Energy) may reveal additional heterogeneity.
\end{enumerate}

\textbf{Future extensions} could include:

\begin{enumerate}
\item \textbf{International markets}: Apply framework to EUR, GBP, emerging market credit. Test whether Merton predictions and sector effects are universal or US-specific.

\item \textbf{Finer sector specialization}: Develop sub-sector adjustments for Financials (Banks, Insurance, REITs), Energy (Exploration, Integrated, Midstream), etc.

\item \textbf{Integration with equity signals}: Combine with equity volatility, CDS spreads, structural model equity-based default probabilities. Test whether sector effects in bonds align with equity market patterns.

\item \textbf{Machine learning augmentation}: Use ML to model $\lambda$ as non-linear function of characteristics, but benchmark against Merton baseline and sector-adjusted specifications to ensure interpretability.

\item \textbf{Transaction costs and implementation}: Assess whether improved DTS adjustments (including sector factors) survive real-world trading costs, market impact, and operational constraints.

\item \textbf{Dynamic sector effects}: Model how sector adjustment factors evolve with regulatory changes, market structure shifts, and macroeconomic conditions.
\end{enumerate}

\subsection{Final perspective}

The Duration-Times-Spread framework is elegantly simple and widely adopted, but its assumption of proportional spread movements breaks down systematically in investment-grade markets with maturity dispersion and exhibits sector-specific patterns that standard implementations ignore. This research program demonstrates that structural credit theory \cite{Wuebben2025}, particularly the Merton model, provides the key to understanding \emph{when} and \emph{why} proportionality fails, while rigorous empirical analysis reveals \emph{where} additional adjustments (notably for sectors like Financials) are necessary.

By integrating theoretical priors with a three-pronged empirical validation approach, we deliver not just statistical improvements but economically meaningful enhancements that respect both the wisdom of markets and the insights of financial theory. The hierarchical testing framework ensures we adopt only the complexity that data justify—whether that means pure Merton tables, sector-adjusted specifications, or more flexible empirical models—resulting in production specifications that are simultaneously theoretically grounded, empirically validated, and practically implementable.

The ultimate deliverable is not merely a set of regression coefficients but a complete framework for understanding credit spread dynamics—one that portfolio managers can use with confidence for hedging and relative value, risk managers can rely on for accurate measurements across sectors, and researchers can build upon for future innovations.

\bibliographystyle{apalike}
\begin{thebibliography}{Collin-Dufresne et~al., 2001}
\bibitem[Merton, 1974]{Merton1974} Merton, R.~C. (1974). On the pricing of corporate debt: The risk structure of interest rates. \emph{Journal of Finance}, 29(2), 449--470.

\bibitem[Black \& Cox, 1976]{Black1976} Black, F., \& Cox, J.~C. (1976). Valuing corporate securities: Some effects of bond indenture provisions. \emph{Journal of Finance}, 31(2), 351--367.

\bibitem[Leland, 1994]{Leland1994} Leland, H.~E. (1994). Corporate debt value, bond covenants, and optimal capital structure. \emph{Journal of Finance}, 49(4), 1213--1252.

\bibitem[Ben Dor et~al., 2010]{BenDor2010} Ben Dor, A., Dynkin, L., Hyman, J., Houweling, P., Van Leeuwen, E., \& Penninga, O. (2010). DTS (Duration Times Spread). \emph{Barclays Capital Quantitative Portfolio Strategy Research}.

\bibitem[Wooldridge, 2010]{Wooldridge2010} Wooldridge, J. (2010). \emph{Econometric Analysis of Cross Section and Panel Data}. MIT Press.

\bibitem[Koenker, 2005]{Koenker2005} Koenker, R. (2005). \emph{Quantile Regression}. Cambridge University Press.

\bibitem[Campbell \& Taksler, 2003]{Campbell2003} Campbell, J.~Y., \& Taksler, G.~B. (2003). Equity volatility and corporate bond yields. \emph{Journal of Finance}, 58(6), 2321--2350.

\bibitem[Chen et~al., 2007]{Chen2007} Chen, L., Lesmond, D.~A., \& Wei, J. (2007). Corporate yield spreads and bond liquidity. \emph{Journal of Finance}, 62(1), 119--149.

\bibitem[Collin-Dufresne et~al., 2001]{CollinDufresne2001} Collin-Dufresne, P., Goldstein, R.~S., \& Martin, J.~S. (2001). The determinants of credit spread changes. \emph{Journal of Finance}, 56(6), 2177--2207.

\bibitem[Wuebben, 2025]{Wuebben2025} Wuebben, B.~J. (2025). When do credit spreads move proportionally? A structural model analysis of the Merton framework. \emph{Working Paper}, AllianceBernstein.
\end{thebibliography}

\end{document}

