\documentclass[11pt]{article}
\usepackage[utf8]{inputenc}
\usepackage[T1]{fontenc}
\usepackage{geometry}
\geometry{letterpaper,margin=1in}
\usepackage{amsmath,amssymb,amsthm,mathtools,bm}
\usepackage{bbm}
\usepackage{booktabs}
\usepackage{graphicx}
\usepackage{hyperref}
\usepackage[square]{natbib}
\usepackage{enumitem}
\usepackage{caption}
\usepackage{xcolor}
\usepackage{mdframed}
\hypersetup{
    colorlinks=true,
    linkcolor=blue!60!black,
    citecolor=blue!60!black,
    urlcolor=blue!60!black
}

\setlist[itemize]{leftmargin=1.5em}
\setlist[enumerate]{leftmargin=1.5em}

\newcommand{\E}{\mathbb{E}}
\newcommand{\Var}{\mathbb{V}\mathrm{ar}}
\newcommand{\Cov}{\mathbb{C}\mathrm{ov}}
\newcommand{\R}{\mathbb{R}}

\title{A Research Program for State-Dependent DTS Scaling in Corporate Credit Spreads:\\
Integrating Structural Model Priors with Empirical Estimation}

\author{%
Bernd J. Wuebben \\
AllianceBernstein, New York\\
\texttt{bernd.wuebben@alliancebernstein.com}
}

\date{\today}

\begin{document}

\maketitle

\begin{abstract}
\noindent This document presents a comprehensive research program for enhancing the Duration-Times-Spread (DTS) framework by integrating theoretical predictions from structural credit models with empirical estimation. Recent theoretical work \cite{Wuebben2025} demonstrates that proportional spread movements fail systematically across maturities even for investment-grade bonds, with 1-year bonds exhibiting 4--6$\times$ higher percentage spread sensitivity than 10-year bonds from the same issuer. These structural model predictions provide strong priors for the state-dependent elasticity $\lambda_{i,t}$ that we seek to estimate empirically.

The research program proceeds sequentially: (1) \textbf{Raw validation} testing structural model predictions directly from spread changes before any regression analysis; (2) \textbf{Establishing variation} documenting that DTS betas differ systematically across bonds; (3) \textbf{Theory testing} evaluating whether Merton predictions explain observed variation; (4) \textbf{Time-variation analysis} assessing stability of relationships; (5) \textbf{Robustness checks} examining tail behavior, shock decomposition, and liquidity effects; (6) \textbf{Production specification} selecting parsimonious model using hierarchical testing framework.

Key improvements over standard approaches: (i) separation of ``is there variation?'' (Stage A) from ``does theory explain it?'' (Stage B); (ii) within-issuer, within-week methodology for Stage 0 avoiding sample size and representativeness issues; (iii) hierarchical model selection guided by theory rather than atheoretical horse-racing; (iv) explicit treatment of unbalanced panels as methodological choice, not separate research stage.

The research delivers both academic and practical value: academically, we provide the first comprehensive empirical test of structural model spread dynamics using market-wide bond data; practically, we determine whether simpler Merton-based adjustment factors suffice or whether complex empirical estimation is necessary for production risk systems.

\vspace{0.3cm}

\noindent \textbf{Structure:} Sections 1--2 cover objectives and data construction (with expanded detail on sample filters, issuer identification, and panel structure). Section 3 introduces structural model priors. Section 4 (Stage 0) conducts raw validation using full-sample within-issuer tests. Sections 5--6 (Stages A--B) establish and explain cross-sectional variation. Sections 7--11 cover time-variation, robustness, and production specification selection.
\end{abstract}

\newpage
\tableofcontents

\newpage

\section{Objectives and High-Level Framework}

\subsection{Motivation and constraints}

In practice, DTS is already embedded in risk systems, attribution frameworks, and portfolio construction for credit portfolios. Any enhanced modeling of spread dynamics must therefore:

\begin{itemize}
\item remain \emph{DTS-centric}: spreads and returns should be modeled as functions of $DTS_{i,t}$ and index-level DTS factors;
\item produce enhancements in the form of a \emph{multiplicative adjustment} $\lambda_{i,t}$, rather than replacing DTS altogether;
\item be implementable and interpretable for risk management, performance attribution, and relative-value work.
\end{itemize}

The empirical question is: \medskip

\emph{For a given macro environment, sector, rating, maturity, liquidity state, and idiosyncratic issuer profile, how do exogenous shocks---macro, sectoral, and issuer-specific---translate into spread movements, and how should these sensitivities be scaled relative to standard DTS?}

\subsection{Core modeling equation and DTS anchor}

We work in discrete time at observation dates $t \in \{1,\dots,T\}$. For each bond $i$ in universe $U$ (IG or HY), denote:

\begin{itemize}
\item $OAS_{i,t}$: option-adjusted spread;
\item $OASD_{i,t}$: OAS-based spread duration;
\item $DTS_{i,t} = OAS_{i,t} \cdot OASD_{i,t}$: standard DTS;
\item $OAS^{(U)}_{t}$: index-level OAS for universe $U$ (IG or HY);
\item $f^{(U)}_{DTS,t} = \frac{\Delta OAS^{(U)}_{t}}{OAS^{(U)}_{t-1}}$: index-level relative spread change, our DTS factor return.
\end{itemize}

We define bond-level relative spread changes as
\begin{equation}
y_{i,t} \equiv \frac{\Delta OAS_{i,t}}{OAS_{i,t-1}} = \frac{OAS_{i,t} - OAS_{i,t-1}}{OAS_{i,t-1}}.
\end{equation}

The canonical DTS assumption for bond $i$ in universe $U$ is
\begin{equation}
y_{i,t} \approx f^{(U)}_{DTS,t} + \varepsilon_{i,t},
\end{equation}
up to idiosyncratic error terms. We \emph{retain} the DTS factor-centric structure, but we generalize it to permit a state-dependent elasticity $\lambda_{i,t}$:
\begin{equation}
y_{i,t} \approx \lambda_{i,t} \, f^{(U)}_{DTS,t} + \varepsilon_{i,t}.
\label{eq:core-lambda}
\end{equation}

In excess-return space, letting $r^{e}_{i,t}$ denote the excess return of bond $i$ over a duration-matched Treasury or swap hedge, we want to write
\begin{equation}
r^{e}_{i,t} \approx OAS_{i,t-1}\Delta t + DTS_{i,t}\, \lambda_{i,t}\, f^{(U)}_{DTS,t} + \text{idiosyncratic term}.
\end{equation}

The research program is thus centered on specifying, estimating, and validating $\lambda_{i,t}$ as a function of observable bond characteristics and state variables.

\subsection{Universe split: IG and HY}

We can define the following two universes:

\begin{enumerate}
\item $U = \text{IG}$: the Bloomberg Barclays U.S.\ Corporate Investment Grade Index; and
\item $U = \text{HY}$: the Bloomberg Barclays U.S.\ High Yield 2\% Issuer Cap Index.
\end{enumerate}

It is not clear as to whether to pool IG and HY in estimation. To start with we pool IG  and HY bond observations. The aim is to produce a unified $\lambda$ specifications for IG and HY, with potentially different parameterizations and state dependencies. Depending on findings we may have to splie the universe into IG and HY bonds.

\subsection{Integration with structural model theory}

This research program explicitly \subsection{Integration with structural model theory}

This research program explicitly incorporates predictions from structural credit models (specifically the Merton framework) as theoretical priors for $\lambda_{i,t}$. Following \cite{Wuebben2025}, structural credit theory establishes that: 

\begin{enumerate}
\item \textbf{Cross-maturity proportionality fails severely in IG:} At spread levels below 300 bps, bonds with 1-year maturity have 4--6$\times$ higher percentage spread sensitivity than 10-year bonds from the same issuer.

\item \textbf{Maturity effects dominate credit quality effects:} In investment-grade markets, cross-maturity elasticity ratios of 400--500\% dwarf same-maturity credit quality variations of 20--35\%.

\item \textbf{Functional form predictions:} The Merton model implies specific functional forms for $\lambda$ as functions of spread level $s$, maturity $T$, and the ratio $R/(sT)$ where $R$ captures the fraction of debt value at risk.

\item \textbf{Regime-dependent patterns:} Proportionality failures are most severe in investment-grade (spreads $<$300 bps, maturity dispersion $>$3 years), moderate in high-yield (300--1000 bps), and paradoxically improve in distressed markets ($>$1000 bps).
\end{enumerate}

These theoretical predictions provide powerful benchmarks for empirical estimation. Rather than searching blindly for $\lambda$ patterns in data, we can:

\begin{itemize}
\item Test whether observed elasticity ratios match Merton predictions
\item Use structural adjustment factors as starting values for iterative estimation
\item Decompose empirical $\lambda$ into theory-consistent and residual components
\item Identify systematic deviations suggesting additional mechanisms (liquidity, sentiment, technical factors)
\end{itemize}

The research program therefore operates on two parallel tracks throughout:

\medskip
\begin{mdframed}[backgroundcolor=blue!5!white,roundcorner=4pt]
\textbf{Track 1: Theory-Guided Estimation}

Start from Merton-implied $\lambda^{\text{Merton}}(s_i, T_i)$ and test whether empirical data support these predictions. Estimate deviations and assess their economic significance.

\noindent \textbf{Track 2: Unrestricted Estimation}

Allow data to determine $\lambda$ coefficients without structural constraints. Compare to Track 1 to assess incremental explanatory power of unconstrained specifications.
\end{mdframed}

\subsection{Sequential research design}

The stages proceed hierarchically with explicit decision points:

\begin{mdframed}[backgroundcolor=green!5!white,roundcorner=4pt]
\textbf{Stage 0: Raw Validation}

Test Merton predictions directly from spread changes before any regression. \emph{Decision: Does theory provide adequate baseline?}

\noindent \textbf{Stage A: Establish Variation}

Document that DTS betas differ across bonds. \emph{Decision: If no variation, stop—standard DTS sufficient.}

\noindent \textbf{Stage B: Explain Variation}

Test whether Merton predictions match observed beta patterns. \emph{Decision: Pure Merton / Calibrated / Need flexibility?}

\noindent \textbf{Stage C: Test Stability}

Assess whether static $\lambda$ suffices or time-variation required. \emph{Decision: Add macro state or keep static?}

\noindent \textbf{Stage D: Robustness}

Only after Stages A--C conclusive: test tail behavior, shock decomposition, liquidity. \emph{Diagnose where/why theory fails.}

\noindent \textbf{Stage E: Production Specification}

Hierarchical selection: Standard DTS $\to$ Pure Merton $\to$ Calibrated $\to$ Empirical. \emph{Stop at simplest adequate model.}
\end{mdframed}

\textbf{Key principle:} Each stage conditional on previous results. Don't add complexity until simple models demonstrably fail.


\section{Data Construction and Sample Filters}

\subsection{Sample period}

\textbf{Primary sample}: January 2013 -- October 2025

\textbf{Rationale}:
\begin{itemize}
\item Excludes 2008--2009 crisis (extreme outlier, liquidity breakdown when Merton least reliable)
\item 15 years provides multiple credit cycles (2013 taper tantrum, 2015--16 energy crisis, 2020 COVID, 2022 rate shock)
\end{itemize}

\textbf{Robustness sample}: Use ICE data from January 2002 -- December 2024

Test whether results hold when including crisis period (separate analysis, not primary).

\subsection{Universe definition}

For each universe $U \in \{\text{IG}, \text{HY}\}$:

\begin{itemize}
\item At each date $t$, the cross-section $\mathcal{I}_{U,t}$ consists of all bonds that are constituents of the relevant Bloomberg Barclays index on that date.
\item We accept the full index methodology: new issues enter when eligible; bonds exit upon maturity, call, downgrade, or when they fail inclusion criteria.
\item There is \emph{no artificial restriction} to fixed-coupon non-call bullet bonds beyond what the index methodology already imposes.
\end{itemize}

Consequently, the panel $(i,t)$ is naturally \emph{unbalanced}: many bonds appear only for part of the sample; new issues are frequent, particularly in IG. We \emph{embrace} this unbalanced nature and design the econometric work to handle it explicitly (see Section~\ref{sec:panel-structure}).

\subsection{Expected sample size}

Based on Bloomberg Barclays index historical composition (guessing here):

\begin{table}[h]
\centering
\begin{tabular}{lrr}
\toprule
& \textbf{IG} & \textbf{HY} \\
\midrule
Avg bonds per month & 5,500 & 1,800 \\
Issuers with 2+ bonds outstanding & 800 & 250 \\
Issuer-weeks for same-issuer tests & 180,000 & 45,000 \\
\bottomrule
\end{tabular}
\caption{Expected sample size for primary sample (2012--2025)}
\end{table}

This provides adequate power for Stage 0 tests using full sample rather than restricting to extreme event weeks.

\subsection{Core variables}

For each bond $i \in \mathcal{I}_{U,t}$ and date $t$:

\begin{itemize}
\item $OAS_{i,t}$ (in basis points);
\item $OASD_{i,t}$ (in years);
\item $DTS_{i,t} = OAS_{i,t}\cdot OASD_{i,t}$;
\item clean price, accrued interest, yield-to-worst;
\item time to maturity $M_{i,t}$ (years);
\item index sector classification (e.g.\ Industrials, Financials, Utilities, etc.);
\item credit rating, mapped to coarse buckets (e.g.\ AAA/AA, A, BBB for IG; BB, B, CCC for HY);
\item issue size (amount outstanding), floating vs fixed, seniority;
\item liquidity proxies where available: TRACE trading volume, bid-ask spread, number of trades, turnover.
\end{itemize}

For the index level (per universe $U$):

\begin{itemize}
\item $OAS^{(U)}_t$ and $OASD^{(U)}_t$, to define $DTS^{(U)}_t$ and $f^{(U)}_{DTS,t}$;
\item sector- and rating-specific index OAS series $OAS^{(U)}_{s,t}$, $OAS^{(U)}_{r,t}$ for sector $s$ and rating $r$ (for Stage E decomposition).
\end{itemize}

\subsection{Time frequency and return definitions}

We start with \emph{weekly} frequency to mitigate microstructure noise:

\begin{itemize}
\item Define a weekly observation grid $t_1, t_2, \dots, t_T$ as e.g.\ Fridays or the last trading day of each week.
\item Construct $OAS_{i,t}$ and $OAS^{(U)}_t$ on that grid by taking end-of-day values.
\end{itemize}

Define:
\begin{align}
y_{i,t} &\equiv \frac{\Delta OAS_{i,t}}{OAS_{i,t-1}} = \frac{OAS_{i,t} - OAS_{i,t-1}}{OAS_{i,t-1}}, \\
f^{(U)}_{DTS,t} &\equiv \frac{\Delta OAS^{(U)}_{t}}{OAS^{(U)}_{t-1}} = \frac{OAS^{(U)}_{t} - OAS^{(U)}_{t-1}}{OAS^{(U)}_{t-1}}.
\end{align}

\textbf{Robustness}: Re-run key specifications (Stages A--B) at daily and monthly frequencies to assess sensitivity to horizon.

\subsection{Panel structure and econometric approach}
\label{sec:panel-structure}

\subsubsection{Unbalanced panel by design}

The Bloomberg Barclays indices have substantial turnover:
\begin{itemize}
\item New issues enter monthly (IG: $\sim$50--100/month, HY: $\sim$20--50/month)
\item Bonds mature, are called, or exit via downgrade/upgrade
\item Average bond stays in IG index 6.5 years, HY index 4.2 years
\end{itemize}

We \textbf{embrace} this unbalanced structure rather than restricting to balanced sub-panels, which would:
\begin{itemize}
\item Eliminate 60\%+ of observations
\item Introduce survivorship bias (old bonds differ systematically from new)
\item Miss new issue dynamics (relevant for practitioners)
\end{itemize}



\section{Structural Model Priors for $\lambda_{i,t}$}
\label{sec:structural-priors}

Before proceeding to empirical estimation, we establish theoretical benchmarks from structural credit models that provide strong priors for the functional form and magnitude of $\lambda_{i,t}$. The theoretical framework developed in \cite{Wuebben2025} provides precise predictions for when and why proportional spread movements fail.

\subsection{The Merton framework and spread elasticity}

In the Merton (1974) structural model, credit spreads arise from the embedded put option in risky debt. The key quantity governing spread dynamics is the \textbf{elasticity} of the spread with respect to firm value:

\begin{equation}
\varepsilon_i \equiv \frac{\partial s_i}{\partial V} \cdot \frac{V}{s_i}
\end{equation}

This measures the percentage change in spread for a 1\% change in firm value. For bonds from the same issuer experiencing a common firm value shock $\Delta V/V$, percentage spread changes are:

\begin{equation}
\frac{\Delta s_i}{s_i} \approx \varepsilon_i \cdot \frac{\Delta V}{V}
\end{equation}

Therefore, proportional spread movements require $\varepsilon_i = \varepsilon_j$ for all bonds $i,j$. The Merton model delivers an exact formula:

\begin{equation}
\varepsilon_i = -\frac{R_i}{T_i \cdot s_i}
\label{eq:merton-elasticity}
\end{equation}

where $T_i$ is maturity, $s_i$ is spread, and $R_i$ is the ratio:

\begin{equation}
R_i = \frac{e^{x_i + rT_i} N(-d_{1,i})}{N(d_{2,i}) + e^{x_i + rT_i} N(-d_{1,i})}
\end{equation}

with $x_i = \ln(V_i/D_i)$ the log-leverage ratio and $d_{1,i}, d_{2,i}$ the standard Black-Scholes parameters.

\textbf{Economic interpretation:} $R_i$ measures the fraction of debt value coming from default states—the ``value at risk.'' For IG bonds, $R \approx 0.05$--$0.30$; for HY, $R \approx 0.30$--$0.80$; for distressed, $R \to 1$.

See \cite{Wuebben2025} for detailed derivations and calibrations of these relationships.

\subsection{Adjustment factors from theory}

Define the adjustment factor for bond $i$ relative to a reference bond $j$ as the ratio of elasticities:

\begin{equation}
\lambda_{i,j} \equiv \frac{\varepsilon_i}{\varepsilon_j} = \frac{R_i \cdot T_j \cdot s_j}{R_j \cdot T_i \cdot s_i}
\label{eq:lambda-theory}
\end{equation}

For practical implementation, we decompose into \textbf{maturity adjustments} (same issuer, different $T$) and \textbf{credit quality adjustments} (different issuers, same $T$).

\subsubsection{Maturity adjustment factors}

For bonds from the same issuer with reference maturity $T_{\text{ref}} = 5$ years:

\begin{equation}
\lambda_T(T; s^*) = \frac{R(T, s^*) \cdot T_{\text{ref}} \cdot s(T_{\text{ref}}, s^*)}{R(T_{\text{ref}}, s^*) \cdot T \cdot s(T, s^*)}
\end{equation}

where $s^*$ is the calibrated spread at $T_{\text{ref}}$ determining the issuer's leverage.

\textbf{Key theoretical predictions:}

\begin{enumerate}
\item At IG spread levels (50--300 bps), $\lambda_T(1y) \approx 2.4$--$3.6$ (short bonds much more sensitive)
\item At IG spread levels, $\lambda_T(10y) \approx 0.6$--$0.7$ (long bonds less sensitive)
\item Adjustment factors converge toward 1.0 as spreads widen to HY levels
\item The cross-maturity effect dominates: elasticity ratio 1y/10y ranges from 5.9 at 50 bps to 1.56 at 1000 bps
\end{enumerate}

Table~\ref{tab:merton-lambda-T} summarizes theoretical predictions.

\begin{table}[ht]
\centering
\caption{Theoretical Maturity Adjustment Factors $\lambda_T$ from Merton Model. Source:\cite{Wuebben2025}.}\label{tab:merton-lambda-T}
\begin{tabular}{lrrrrr}
\toprule
\textbf{Spread Level} & \multicolumn{5}{c}{\textbf{Maturity Adjustment $\lambda_T(T; 5y)$}} \\
\textbf{(bps)} & 1y & 3y & 5y & 7y & 10y \\
\midrule
50 (AAA) & 3.62 & 1.47 & 1.00 & 0.79 & 0.61 \\
100 (AA) & 3.27 & 1.42 & 1.00 & 0.80 & 0.64 \\
200 (A-) & 2.78 & 1.36 & 1.00 & 0.82 & 0.67 \\
300 (BBB-) & 2.40 & 1.30 & 1.00 & 0.84 & 0.70 \\
\midrule
500 (BB) & 1.91 & 1.25 & 1.00 & 0.86 & 0.73 \\
1000 (B-) & 1.26 & 1.12 & 1.00 & 0.91 & 0.81 \\
2000 (CCC) & 1.08 & 1.05 & 1.00 & 0.97 & 0.93 \\
\bottomrule
\end{tabular}
\end{table}

\subsubsection{Credit quality adjustment factors}

For bonds with same maturity $T=5y$ but different spread levels, relative to reference $s_{\text{ref}}=100$ bps:

\begin{equation}
\lambda_s(s; 100) = \frac{R(s) \cdot 100}{R(100) \cdot s}
\end{equation}

\textbf{Key theoretical predictions:}

\begin{enumerate}
\item IG bonds safer than 100 bps have $\lambda_s > 1$ (higher sensitivity)
\item IG bonds riskier than 100 bps have $\lambda_s < 1$ (lower sensitivity)
\item Effect is moderate in IG: from 50 to 300 bps, $\lambda_s$ varies from 1.15 to 0.75 (only 35\% range)
\item Power-law approximation: $\lambda_s(s) \approx (s/100)^{-0.25}$ achieves $R^2=0.92$
\end{enumerate}

Table~\ref{tab:merton-lambda-s} summarizes.

\begin{table}[ht]
\centering
\caption{Theoretical Credit Quality Adjustment Factors $\lambda_s$ from Merton Model. Source: \cite{Wuebben2025}.}\label{tab:merton-lambda-s}
\begin{tabular}{lrr}
\toprule
\textbf{Spread (bps)} & \textbf{Exact Merton} & \textbf{Power Law $(s/100)^{-0.25}$} \\
\midrule
50 & 1.145 & 1.189 \\
100 & 1.000 & 1.000 \\
200 & 0.847 & 0.841 \\
300 & 0.746 & 0.760 \\
\midrule
500 & 0.635 & 0.669 \\
1000 & 0.468 & 0.562 \\
2000 & 0.299 & 0.473 \\
\bottomrule
\end{tabular}
\end{table}

\subsection{Regime-specific predictions}

The Merton model predicts distinct behavior across five regimes \cite{Wuebben2025}:

\begin{enumerate}[leftmargin=*, label=\textbf{Regime \arabic*:}]
\item \textbf{IG with narrow maturity range} ($s < 300$ bps, $\Delta T < 2y$): Standard DTS works reasonably. Maturity effects not applicable. Credit quality variation causes 20--35\% deviation—acceptable.

\item \textbf{IG with wide maturity range} ($s < 300$ bps, $\Delta T > 3y$): \textbf{Primary failure mode}. Cross-maturity $\lambda$ ratios of 3--6$\times$ create 300--500\% deviations. This is where DTS models systematically fail.

\item \textbf{HY with narrow maturity range} ($300 < s < 1000$ bps, $\Delta T < 2y$): Cross-maturity effects reduced to 50--160\% but still substantial. Same-maturity credit quality variation increases to 40--73\%.

\item \textbf{HY with wide maturity range} ($300 < s < 1000$ bps, $\Delta T > 3y$): Both effects large. Comprehensive adjustments required.

\item \textbf{Distressed} ($s > 1000$ bps): Proportionality paradoxically improves. Both cross-maturity and same-maturity deviations decline.
\end{enumerate}

\subsection{Empirical testing strategy}

The structural model priors suggest a specific empirical testing sequence:

\begin{mdframed}[backgroundcolor=green!5!white,roundcorner=4pt]
\textbf{Empirical Questions for Each Stage:}

These questions test the theoretical predictions established in \cite{Wuebben2025}:

\begin{enumerate}
\item \textbf{Do raw data exhibit Merton-predicted patterns?} (Stage 0) Before any regression, compute elasticity ratios and compare to theory.

\item \textbf{Is there cross-sectional variation to explain?} (Stage A) Document that betas differ. If not, standard DTS sufficient.

\item \textbf{Does Merton explain the variation?} (Stage B) Test whether theoretical $\lambda$ matches empirical patterns.

\item \textbf{Is the relationship stable over time?} (Stage C) If Stage B works, test whether static $\lambda$ suffices.

\item \textbf{Where does theory fail?} (Stage D) Diagnose failures: tails? liquidity? specific regimes?

\item \textbf{What production spec is adequate?} (Stage E) Hierarchical selection guided by theory.
\end{enumerate}
\end{mdframed}


\section{Stage 0: Raw Validation Using Bucket-Level Analysis}
\label{sec:stage0}

\subsection{Objective and motivation}

Stage 0 provides an assumption-free test of Merton predictions before any regression analysis. We ask: \emph{Do bonds in similar buckets (rating, maturity, sector) exhibit spread sensitivities consistent with structural theory?}

\textbf{Approach:} Rather than complex issuer-week fixed effects, we use a simple bucket-level pooled regression that aggregates thousands of observations for strong statistical power while still testing the core Merton predictions.

\subsection{Methodology}

\subsubsection{Step 1: Define buckets}

Define buckets $k$ by:
\begin{itemize}
\item \textbf{Rating}: AAA/AA, A, BBB for IG; BB, B, CCC for HY
\item \textbf{Maturity}: 1--2y, 2--3y, 3--5y, 5--7y, 7--10y, 10y+
\item \textbf{Sector}: Bloomberg Class 3
\end{itemize}

For each bucket $k$, compute representative characteristics using bonds in that bucket:
\begin{itemize}
\item $\bar{T}_k$: median maturity
\item $\bar{s}_k$: median spread
\item $\lambda^{\text{Merton}}_k = \lambda_T(\bar{T}_k; 5y, \bar{s}_k) \times \lambda_s(\bar{s}_k; 100)$ using Tables~\ref{tab:merton-lambda-T}--\ref{tab:merton-lambda-s}
\end{itemize}

This creates $3 \times 6 \times 4 = 72$ buckets for IG and 72 for HY (some may be sparse or empty).

\subsubsection{Step 2: Pooled regression by bucket}

For each bucket $k$, estimate using all bond-week observations in that bucket:
\begin{equation}
y_{i,t} = \alpha^{(k)} + \beta^{(k)} \cdot f^{(U)}_{DTS,t} + \varepsilon^{(k)}_{i,t}, \quad i \in \text{bucket } k
\end{equation}

where:
\begin{itemize}
\item $y_{i,t} = \Delta OAS_{i,t} / OAS_{i,t-1}$: percentage spread change for bond $i$ in week $t$
\item $f^{(U)}_{DTS,t} = \Delta OAS^{(U)}_t / OAS^{(U)}_{t-1}$: index-level percentage spread change
\item $\beta^{(k)}$: empirical sensitivity for bonds in bucket $k$
\end{itemize}

\textbf{Estimation}: Pooled OLS within each bucket.

\textbf{Standard errors}: Cluster by week to allow arbitrary correlation across bonds within each week.

\textbf{Sample size}: Each major bucket (e.g., BBB 3--5y Industrials) has 5,000--20,000 bond-week observations over 2010--2024, providing strong statistical power.

\textbf{Advantages}:
\begin{itemize}
\item Simple, transparent methodology
\item Thousands of observations per bucket (strong statistics)
\item Aggregates across many weeks, reducing noise
\item Buckets defined by observable characteristics (rating, maturity, sector)
\item Direct comparison of $\hat{\beta}^{(k)}$ to $\lambda^{\text{Merton}}_k$
\end{itemize}

\subsubsection{Step 3: Compare empirical to theoretical}

For each bucket $k$, we compare:
\begin{itemize}
\item $\hat{\beta}^{(k)}$: empirical DTS sensitivity from regression
\item $\lambda^{\text{Merton}}_k$: theoretical prediction from Merton model
\end{itemize}

\textbf{Test 1 (Level):} Is $\hat{\beta}^{(k)} \approx \lambda^{\text{Merton}}_k$?

Use $t$-test: $H_0: \beta^{(k)} = \lambda^{\text{Merton}}_k$ with clustered standard errors.

\textbf{Test 2 (Cross-maturity pattern):} Across maturity buckets (holding rating and sector fixed), do empirical betas exhibit predicted monotonicity?

\emph{Merton prediction}: $\hat{\beta}^{(1-2y)} > \hat{\beta}^{(3-5y)} > \hat{\beta}^{(5-7y)} > \hat{\beta}^{(10y+)}$

\emph{Test}: Spearman rank correlation between empirical $\hat{\beta}^{(k)}$ and theoretical $\lambda^{\text{Merton}}_k$ across maturity buckets. Expected strong negative correlation in IG (short bonds more sensitive).

\textbf{Test 3 (Regime pattern):} Does the pattern hold strongly in IG but weaken in HY as theory predicts?

Split sample by spread level:
\begin{itemize}
\item IG spreads ($<$300 bps): Expect large cross-maturity dispersion
\item HY spreads (300--1000 bps): Expect moderate dispersion  
\item Distressed ($>$1000 bps): Expect convergence toward $\beta \approx 1$
\end{itemize}

Test whether cross-maturity standard deviation of $\hat{\beta}^{(k)}$ declines as spreads widen (Regime 5 prediction).

\subsection{Statistical tests}

\textbf{Aggregated test across all buckets:}

Pool all buckets and test whether average deviation from Merton is zero:
\begin{equation}
\text{Mean Deviation} = \frac{1}{K} \sum_{k=1}^{K} \left( \hat{\beta}^{(k)} - \lambda^{\text{Merton}}_k \right)
\end{equation}

Test $H_0: \text{Mean Deviation} = 0$ using bootstrap standard errors (resampling weeks).

\textbf{Variation test:}

Does most of the cross-bucket variation in $\hat{\beta}^{(k)}$ come from maturity differences as Merton predicts?

Compute:
\begin{itemize}
\item Total variation: $\text{Var}(\hat{\beta}^{(k)})$ across all buckets
\item Within-rating variation: $\text{Var}(\hat{\beta}^{(k)} | \text{rating})$ holding rating constant
\end{itemize}

If within-rating variation is large and dominated by maturity, this confirms Merton's emphasis on maturity effects.

\subsection{Deliverables for Stage 0}

\begin{itemize}
\item \textbf{Table 0.1:} Bucket-level results for key buckets
\begin{itemize}
\item Rows: Rating $\times$ Maturity combinations (e.g., BBB 1--2y, BBB 3--5y, BBB 7--10y)
\item Columns: $\hat{\beta}^{(k)}$ (empirical), $\lambda^{\text{Merton}}_k$ (theory), Ratio, $t$-stat for $H_0: \beta=\lambda$, Sample size
\item Separate panels for IG and HY
\item Highlight cells where ratio outside [0.8, 1.2]
\end{itemize}

\item \textbf{Table 0.2:} Cross-maturity pattern tests
\begin{itemize}
\item For each rating class, show $\hat{\beta}^{(k)}$ across maturity buckets
\item Spearman correlation with theoretical $\lambda^{\text{Merton}}_k$
\item Test whether monotonicity holds (short $>$ medium $>$ long)
\end{itemize}

\item \textbf{Figure 0.1:} Scatter plot: Empirical $\hat{\beta}^{(k)}$ (y-axis) vs Theoretical $\lambda^{\text{Merton}}_k$ (x-axis) for all buckets
\begin{itemize}
\item 45-degree line for perfect agreement
\item Point size proportional to sample size
\item Color-code by spread level: IG (blue), HY (orange), Distressed (red)
\item Annotate outliers
\end{itemize}

\item \textbf{Figure 0.2:} Cross-maturity patterns by rating
\begin{itemize}
\item Separate panels for AAA/AA, A, BBB, BB, B
\item X-axis: Maturity (1y, 3y, 5y, 7y, 10y)
\item Y-axis: $\hat{\beta}$ (empirical, solid line with points) and $\lambda^{\text{Merton}}$ (theoretical, dashed line)
\item Shows whether empirical pattern matches theoretical prediction
\end{itemize}

\item \textbf{Figure 0.3:} Regime patterns
\begin{itemize}
\item X-axis: Average spread level of bucket
\item Y-axis: Cross-maturity dispersion (std dev of $\hat{\beta}^{(k)}$ across maturity buckets)
\item Shows whether dispersion declines as spreads widen (convergence to Regime 5)
\end{itemize}

\item \textbf{Written Summary (2--3 pages):}
\begin{enumerate}
\item \textbf{Does Merton predict bucket-level sensitivities?} Report mean deviation and distribution of ratios $\hat{\beta}^{(k)} / \lambda^{\text{Merton}}_k$

\item \textbf{Is cross-maturity pattern correct?} Test 2 results: Do short bonds have higher $\beta$ than long bonds as predicted?

\item \textbf{Does pattern differ by spread level?} Test 3 results: Is dispersion larger in IG than HY?

\item \textbf{Where do largest deviations occur?} Which buckets show $\hat{\beta}^{(k)}$ furthest from $\lambda^{\text{Merton}}_k$? Any sector effects?

\item \textbf{Practical implication:} Can we use Merton tables as baseline for Stages A--C, or do we need to estimate $\lambda$ fully empirically?
\end{enumerate}
\end{itemize}

\subsection{Decision point}

Based on Stage 0 results:

\begin{itemize}
\item \textbf{If median ratio $\in [0.8, 1.2]$ and maturity pattern holds}: Merton provides good baseline. Proceed with theory-constrained specifications in Stages A--C. Use $\lambda^{\text{Merton}}$ as starting point, estimate small calibration adjustments.

\item \textbf{If systematic bias} (median ratio $> 1.2$ or $< 0.8$): Merton has right structure but wrong scale. Proceed to calibrated Merton in Stage B (estimate scaling factor).

\item \textbf{If high dispersion but no bias}: Merton captures average but misses heterogeneity. Proceed with both theory-constrained and unrestricted tracks in parallel.

\item \textbf{If wrong patterns} (long bonds more sensitive than short, or no maturity effect): Theory fundamentally fails. Emphasize unrestricted estimation, investigate alternative mechanisms.

\item \textbf{If Merton works in IG but fails in HY}: Regime-differentiated modeling. Use theory-based for IG, fully empirical for HY.
\end{itemize}
\section{Stage A: Establish Cross-Sectional Variation}
\label{sec:stage-a}

\subsection{Objective}

Stage A establishes the empirical fact that DTS betas differ across bonds \emph{before} testing whether Merton explains why. This separation is critical:

\begin{itemize}
\item If no significant variation exists, stop here—standard DTS is adequate
\item If variation exists but Merton doesn't explain it, we know theory fails
\item If variation exists and Merton explains it, theory provides parsimonious structure
\end{itemize}

\subsection{Specification A.1: Bucket-level betas}

For each bucket $k$ (defined by rating $\times$ maturity $\times$ sector), estimate:
\begin{equation}
y_{i,t} = \alpha^{(k)} + \beta^{(k)} f^{(U)}_{DTS,t} + \varepsilon^{(k)}_{i,t}, \quad i \in \text{bucket } k
\end{equation}

\textbf{Bucket definitions}:
\begin{itemize}
\item Rating: AAA/AA, A, BBB for IG; BB, B, CCC for HY
\item Maturity: 1--2y, 2--3y, 3--5y, 5--7y, 7--10y, 10y+
\item Sector: Industrials, Financials, Utilities, Energy (can add more)
\end{itemize}

Creates $3 \times 6 \times 4 = 72$ buckets for IG, $3 \times 6 \times 4 = 72$ for HY (some will be sparse).

\textbf{Estimation}: Pooled OLS within each bucket, cluster standard errors by week.

\textbf{Key outputs}:
\begin{itemize}
\item Table A.1: $\hat{\beta}^{(k)}$ for all buckets with standard errors and $t$-statistics
\item $F$-test for equality across buckets: $H_0: \beta^{(1)} = \beta^{(2)} = \cdots = \beta^{(K)}$
\item \textbf{Critical decision}: If $F$-test fails to reject ($p > 0.10$), no significant variation—stop here, use standard DTS
\end{itemize}

\subsection{Specification A.2: Continuous characteristics}

Instead of discrete buckets, estimate how beta varies with continuous characteristics.

\textbf{Two-step procedure}:

\textbf{Step 1}: For each bond $i$, estimate bond-specific beta using rolling 2-year windows:
\begin{equation}
y_{i,t} = \alpha_i + \beta_i f_{DTS,t} + \varepsilon_{i,t}
\end{equation}

Yields time-series of $\hat{\beta}_{i,\tau}$ for bond $i$ at window midpoint $\tau$.

\textbf{Step 2}: Cross-sectional regression of estimated betas on characteristics:
\begin{equation}
\hat{\beta}_{i,\tau} = \gamma_0 + \gamma_M M_{i,\tau} + \gamma_s s_{i,\tau} + \gamma_{M^2} M_{i,\tau}^2 + \gamma_{Ms} M_{i,\tau} \cdot s_{i,\tau} + u_{i,\tau}
\end{equation}

where $M_{i,\tau}$ is maturity, $s_{i,\tau}$ is spread at window midpoint.

\textbf{Standard errors}: Bootstrap or cluster by bond (account for multiple windows per bond).

\textbf{Key outputs}:
\begin{itemize}
\item Coefficient estimates $\hat{\gamma}_M, \hat{\gamma}_s, \hat{\gamma}_{M^2}, \hat{\gamma}_{Ms}$
\item $R^2$ of cross-sectional fit
\item \textbf{Interpretation}: Do betas vary systematically with maturity and spread?
\end{itemize}

\subsection{Establishing the stylized fact}

\textbf{Deliverable}: Document that DTS betas are not constant. Show:
\begin{enumerate}
\item Statistical significance of variation (F-test from A.1, $R^2$ from A.2)
\item Economic significance: Range of $\hat{\beta}^{(k)}$ across buckets (e.g., min 0.6, max 1.8 implies 3x variation)
\item Pattern: Short-maturity / low-spread bonds have higher betas (preliminary observation, not causal claim)
\end{enumerate}

\textbf{Next stage}: Having established variation exists, Stage B tests whether Merton explains it.

\subsection{Deliverables for Stage A}

\begin{itemize}
\item \textbf{Table A.1:} Bucket-level $\hat{\beta}^{(k)}$ estimates
\begin{itemize}
\item Rows: Maturity buckets
\item Columns: Rating buckets
\item Separate panels for IG and HY (and optionally by sector)
\item Include standard errors, $t$-statistics, sample size per bucket
\end{itemize}

\item \textbf{Table A.2:} Tests of beta equality
\begin{itemize}
\item $F$-test for $H_0:$ all $\beta^{(k)}$ equal, overall and by dimension:
\begin{itemize}
\item Across maturities (holding rating constant)
\item Across ratings (holding maturity constant)
\item Across sectors
\end{itemize}
\item Report $F$-statistic, degrees of freedom, $p$-value
\item \textbf{Decision rule}: If all $p > 0.10$, declare standard DTS adequate
\end{itemize}

\item \textbf{Table A.3:} Continuous characteristic regression (Specification A.2)
\begin{itemize}
\item Coefficients: $\hat{\gamma}_0, \hat{\gamma}_M, \hat{\gamma}_s, \hat{\gamma}_{M^2}, \hat{\gamma}_{Ms}$
\item Standard errors, $t$-statistics, $R^2$
\item Separate for IG and HY
\end{itemize}

\item \textbf{Figure A.1:} Heatmap of $\hat{\beta}^{(k)}$ by maturity (x-axis) $\times$ rating (y-axis). Color intensity represents beta magnitude. Separate panels for IG and HY.

\item \textbf{Figure A.2:} Implied beta surface from Specification A.2: 3D plot or contour plot with maturity on x-axis, spread on y-axis, predicted $\hat{\beta}$ on z-axis/color.

\item \textbf{Diagnostic summary (2 pages):}
\begin{enumerate}
\item Is variation statistically significant? (F-test results)
\item Is variation economically meaningful? (Range and IQR of $\hat{\beta}^{(k)}$)
\item What characteristics drive variation? (Maturity vs spread vs sector)
\item Does IG show more variation than HY? (Regime 2 prediction)
\item \textbf{Recommendation}: Proceed to Stage B to test whether Merton explains patterns, or stop if no variation.
\end{enumerate}
\end{itemize}

\subsection{Decision point}

\begin{itemize}
\item \textbf{If $F$-test $p < 0.01$ and $R^2 > 0.15$ in A.2}: Strong evidence of systematic variation. Proceed to Stage B with high confidence.

\item \textbf{If $F$-test $0.01 < p < 0.10$}: Marginal variation. Proceed to Stage B but may find theory sufficient.

\item \textbf{If $F$-test $p > 0.10$ and $R^2 < 0.05$}: No meaningful variation. \textbf{Stop here—standard DTS is adequate. Report this as primary finding.}
\end{itemize}



% PART 2: Stages B, C, and D
% This continues from Part 1

\section{Stage B: Does Merton Explain the Variation?}
\label{sec:stage-b}

\subsection{Objective}

Having established in Stage A that DTS betas vary across bonds, Stage B tests whether Merton's structural predictions explain this variation. This is the \emph{core empirical test} of the theoretical framework.

\textbf{Critical distinction}: Stage A documented the \emph{fact} of variation. Stage B tests whether \emph{theory explains} it.

\subsection{Specification B.1: Merton as offset (constrained)}

Use Merton-predicted $\lambda^{\text{Merton}}_{i,t}$ as known adjustment factor:
\begin{equation}
y_{i,t} = \alpha + \beta_{\text{Merton}} \cdot [\lambda^{\text{Merton}}_{i,t} \cdot f_{DTS,t}] + \varepsilon_{i,t}
\end{equation}

where $\lambda^{\text{Merton}}_{i,t} = \lambda_T(T_i; 5y, s_{i,t}) \times \lambda_s(s_{i,t}; 100\text{bps})$ using Tables~\ref{tab:merton-lambda-T}--\ref{tab:merton-lambda-s}.

\textbf{Theory prediction}: If Merton is exactly correct, $\beta_{\text{Merton}} = 1$.

\textbf{Test}: Wald test $H_0: \beta = 1$ with clustered standard errors (week $\times$ issuer).

\textbf{Interpretation}:
\begin{itemize}
\item $\hat{\beta}_{\text{Merton}} \in [0.9, 1.1]$: Merton predictions unbiased, theory works
\item $\hat{\beta}_{\text{Merton}} \in [0.8, 1.2]$: Close enough for practical purposes
\item $\hat{\beta}_{\text{Merton}} > 1.2$ or $< 0.8$: Systematic bias, need calibration
\item Low $R^2$ despite $\hat{\beta} \approx 1$: Merton captures mean but misses dispersion
\end{itemize}

\subsection{Specification B.2: Decomposed components}

Test maturity vs credit quality effects separately:
\begin{equation}
y_{i,t} = \alpha + \beta_T [\lambda_T(T_i; 5y, s_{i,t}) \cdot f_{DTS,t}] + \beta_s [\lambda_s(s_{i,t}; 100) \cdot f_{DTS,t}] + \varepsilon_{i,t}
\end{equation}

\textbf{Theory predictions}:
\begin{itemize}
\item $\beta_T \approx 1$: Maturity adjustment works
\item $\beta_s \approx 1$: Credit quality adjustment works
\item If both hold: Merton decomposition empirically valid
\end{itemize}

\textbf{Diagnostic patterns}:
\begin{itemize}
\item $\beta_T \approx 1$ but $\beta_s \neq 1$: Maturity effects correct, quality effects need recalibration
\item $\beta_T \neq 1$ but $\beta_s \approx 1$: Quality effects correct, maturity functional form wrong
\item Both $\neq 1$: Need to reconsider entire Merton structure
\end{itemize}

\subsection{Benchmarking against Stage A}

\textbf{Key comparison}: Does theory-constrained Specification B.1 perform comparably to unrestricted bucket regressions from Stage A?

\textbf{Metrics}:
\begin{enumerate}
\item \textbf{$R^2$ comparison}:
\begin{itemize}
\item $R^2_{\text{buckets}}$ from Stage A (upper bound—fully flexible)
\item $R^2_{\text{Merton}}$ from Specification B.1
\item If $R^2_{\text{Merton}} > 0.9 \times R^2_{\text{buckets}}$: Theory captures 90\%+ of explainable variation
\end{itemize}

\item \textbf{Bucket-level residuals}: For each bucket $k$, compute:
\begin{equation}
\text{Residual}_k = \hat{\beta}^{(k)}_{\text{Stage A}} - \lambda^{\text{Merton}}_k
\end{equation}
If most residuals small (e.g., $<$0.2) and unsystematic, theory adequate.

\item \textbf{RMSE comparison}: Root mean squared error of spread change predictions:
\begin{equation}
\text{RMSE} = \sqrt{\frac{1}{N}\sum_{i,t} (y_{i,t} - \hat{y}_{i,t})^2}
\end{equation}
Compare Merton vs bucket-based predictions.
\end{enumerate}

\subsection{Specification B.3: Unrestricted for comparison}

Estimate fully flexible functional form:
\begin{equation}
\lambda_i = \beta_0 + \beta_M M_i + \beta_{M^2} M_i^2 + \beta_s s_i + \beta_{s^2} s_i^2 + \beta_{Ms} M_i \cdot s_i + \sum_{\text{rating}} \beta_r + \sum_{\text{sector}} \beta_{\text{sec}}
\end{equation}

Then:
\begin{equation}
y_{i,t} = \alpha + [\hat{\lambda}_i \cdot f_{DTS,t}] + \varepsilon_{i,t}
\end{equation}

\textbf{Purpose}: Assess incremental explanatory power beyond theory. If $R^2_{\text{unrestricted}} \gg R^2_{\text{Merton}}$, theory misses important patterns.

\subsection{Theory vs reality table}

\textbf{Critical deliverable}: Direct comparison of empirical betas to Merton predictions from \cite{Wuebben2025}.

\begin{table}[h]
\centering
\caption{Do Empirical Betas Match Merton Predictions?}
\begin{tabular}{lccc}
\toprule
\textbf{Bucket} & $\hat{\beta}^{(k)}$ (Stage A) & $\lambda^{\text{Merton}}_k$ (Theory) & Ratio \\
\midrule
1y, 50bps & 3.45 & 3.62 & 0.95 \\
1y, 100bps & 3.15 & 3.27 & 0.96 \\
1y, 200bps & 2.85 & 2.78 & 1.03 \\
\midrule
3y, 50bps & 1.52 & 1.47 & 1.03 \\
3y, 100bps & 1.38 & 1.42 & 0.97 \\
3y, 200bps & 1.29 & 1.36 & 0.95 \\
\midrule
10y, 50bps & 0.58 & 0.61 & 0.95 \\
10y, 100bps & 0.61 & 0.64 & 0.95 \\
10y, 200bps & 0.65 & 0.67 & 0.97 \\
\bottomrule
\end{tabular}
\end{table}

\textbf{Decision criteria}:
\begin{itemize}
\item \textbf{If 90\%+ of ratios in [0.8, 1.2]}: Merton provides excellent baseline. Use pure Merton or calibrated version for production.

\item \textbf{If systematic bias} (all ratios $> 1.2$ or $< 0.8$): Recalibrate with $\beta_{\text{Merton}}$ from Specification B.1. Theory has right structure, wrong scale.

\item \textbf{If high dispersion but no bias}: Heterogeneity beyond Merton dimensions. Proceed to unrestricted estimation, but Merton still useful as starting point.

\item \textbf{If wrong patterns} (e.g., long bonds have higher sensitivity than short): Theory fundamentally fails. Investigate alternative mechanisms.
\end{itemize}

\subsection{Deliverables for Stage B}

\begin{itemize}
\item \textbf{Table B.1:} Constrained Merton specifications
\begin{itemize}
\item Spec B.1: $\hat{\beta}_{\text{Merton}}$, standard error, Wald test $p$-value for $H_0: \beta=1$
\item Spec B.2: $\hat{\beta}_T$, $\hat{\beta}_s$, standard errors, joint test $p$-value for $H_0: (\beta_T, \beta_s) = (1,1)$
\item Separate panels for IG and HY
\end{itemize}

\item \textbf{Table B.2:} Model comparison
\begin{itemize}
\item Rows: Stage A buckets, Spec B.1 (Merton), Spec B.2 (decomposed), Spec B.3 (unrestricted)
\item Columns: $R^2$, RMSE, AIC, number of parameters
\item $\Delta R^2$ relative to Stage A buckets
\end{itemize}

\item \textbf{Table B.3:} Theory vs Reality (as above)
\begin{itemize}
\item All maturity $\times$ spread bucket combinations
\item Empirical $\hat{\beta}^{(k)}$ from Stage A
\item Theoretical $\lambda^{\text{Merton}}_k$
\item Ratio and absolute deviation
\item Highlight cells where $|\text{Ratio} - 1| > 0.25$
\end{itemize}

\item \textbf{Figure B.1:} Scatter plot: Empirical $\hat{\beta}^{(k)}$ (y-axis) vs Theoretical $\lambda^{\text{Merton}}_k$ (x-axis) for all buckets. 45-degree line for perfect agreement. Color-code by:
\begin{itemize}
\item IG-narrow maturity (circles)
\item IG-wide maturity (squares)
\item HY-narrow (triangles)
\item HY-wide (diamonds)
\item Distressed (stars)
\end{itemize}

\item \textbf{Figure B.2:} Residual analysis: $\hat{\beta}^{(k)} - \lambda^{\text{Merton}}_k$ by:
\begin{itemize}
\item Panel A: By maturity (x-axis: 1y, 3y, 5y, 7y, 10y)
\item Panel B: By spread level (x-axis: 50, 100, 200, 300, 500, 1000 bps)
\item Panel C: By sector (x-axis: Industrials, Financials, Utilities, Energy)
\end{itemize}
Zero line indicates perfect Merton prediction. Look for systematic patterns.

\item \textbf{Figure B.3:} Implied $\lambda$ surface from Spec B.3 (unrestricted) vs Merton prediction:
\begin{itemize}
\item 3D surface plot or side-by-side contour plots
\item X-axis: Maturity (1--10 years)
\item Y-axis: Spread (50--1000 bps)
\item Z-axis/color: Predicted $\lambda$
\item Shows where unrestricted deviates from theory
\end{itemize}

\item \textbf{Diagnostic summary (3--4 pages):}
\begin{enumerate}
\item \textbf{Does Merton work?} (Spec B.1 results: $\hat{\beta}_{\text{Merton}}$ and $R^2$)

\item \textbf{Which component drives fit?} (Spec B.2 results: maturity vs quality)

\item \textbf{Where does theory succeed?} (Table B.3 analysis: which regimes have ratios near 1.0)

\item \textbf{Where does theory fail?} (Residual patterns from Figure B.2)

\item \textbf{Is unrestricted necessary?} (Table B.2 comparison: $\Delta R^2$ and parameter efficiency)

\item \textbf{Practical recommendation}:
\begin{itemize}
\item Use pure Merton tables (simplest)
\item Use calibrated Merton with $\hat{\beta}_{\text{Merton}}$ (simple + data-driven)
\item Need full unrestricted (complex but necessary)
\end{itemize}

\item \textbf{Implications for Stage C}: Does static Merton suffice or need time-variation?
\end{enumerate}
\end{itemize}

\subsection{Decision point}

Based on Stage B results:

\begin{mdframed}[backgroundcolor=yellow!10!white,roundcorner=4pt]
\textbf{Decision Tree for Stage C Entry:}

\textbf{Path 1: Theory works well}

\textit{Condition}: $\hat{\beta}_{\text{Merton}} \in [0.9, 1.1]$ and $R^2_{\text{Merton}} > 0.85 \times R^2_{\text{buckets}}$

\textit{Action}: Proceed to Stage C to test whether static $\lambda^{\text{Merton}}$ suffices or time-variation needed. High confidence in theoretical foundation.

\textbf{Path 2: Theory needs calibration}

\textit{Condition}: $\hat{\beta}_{\text{Merton}}$ outside $[0.9, 1.1]$ but patterns match (high $R^2$, residuals unsystematic)

\textit{Action}: Adopt calibrated Merton: $\lambda^{\text{prod}} = \hat{\beta}_{\text{Merton}} \cdot \lambda^{\text{Merton}}$. Proceed to Stage C to test stability of $\hat{\beta}_{\text{Merton}}$ over time.

\textbf{Path 3: Theory captures structure but misses details}

\textit{Condition}: Moderate $R^2_{\text{Merton}}$ (e.g., 0.6--0.8 $\times$ buckets), some systematic residuals

\textit{Action}: Proceed to Stage C with both theory-guided and unrestricted tracks in parallel. Stage C will reveal if time-variation helps or if static unrestricted better.

\textbf{Path 4: Theory fundamentally fails}

\textit{Condition}: Wrong patterns (e.g., long bonds more sensitive than short), or $R^2_{\text{Merton}} < 0.5 \times R^2_{\text{buckets}}$

\textit{Action}: Skip Stage C (no point testing time-variation of failed model). Proceed directly to Stage D (robustness) to diagnose \emph{why} theory fails (liquidity? tails? specific shocks?). Then Stage E with unrestricted specification only.

\textit{Implication}: Report that structural models don't provide adequate guidance for DTS adjustments in this market. Empirical approach necessary.
\end{mdframed}


\section{Stage C: Does Static Merton Suffice or Do We Need Time-Variation?}
\label{sec:stage-c}

\subsection{Objective and prerequisite}

\textbf{Prerequisite}: Stage B showed that Merton $\lambda(s, T)$ explains cross-sectional variation (Paths 1--3 from Stage B decision tree).

\textbf{Objective}: Test whether the relationship between $\lambda$ and $(s, T)$ is stable over time, or whether macro state variables induce time-variation.

\textbf{Key principle}: Don't add time-variation until you've proven the simple static model fails.

\subsection{Rolling window stability test}

Divide sample into non-overlapping 1-year windows $w \in \{1, 2, \dots, W\}$. For each window:

\begin{equation}
y_{i,t} = \alpha_w + \beta_w \cdot [\lambda^{\text{Merton}}_{i,t} \cdot f_{DTS,t}] + \varepsilon_{i,t}, \quad t \in w
\end{equation}

This yields time series of $\hat{\beta}_w$ with standard errors $\text{se}(\hat{\beta}_w)$.

\textbf{Stability test}: Chow test for structural break:
\begin{equation}
H_0: \beta_1 = \beta_2 = \cdots = \beta_W
\end{equation}

Compute $F$-statistic comparing restricted (single $\beta$) vs unrestricted (separate $\beta_w$) models.

\textbf{Decision rule}:
\begin{itemize}
\item \textbf{If $p > 0.10$}: Static $\lambda$ sufficient. \textbf{Stop Stage C here.} No need for time-varying adjustments. Report that Merton provides stable baseline.

\item \textbf{If $0.01 < p < 0.10$}: Marginal instability. Proceed to investigate drivers but be skeptical of over-parameterization.

\item \textbf{If $p < 0.01$}: Significant time-variation. Proceed to macro driver analysis.
\end{itemize}

\subsection{Visual stability assessment}

\textbf{Time series plot}: $\hat{\beta}_w$ over time with 95\% confidence bands.

\textbf{Interpretation}:
\begin{itemize}
\item Confidence bands overlapping 1.0 throughout: Static Merton works, bands capture sampling variation
\item Confidence bands tight but $\hat{\beta}_w$ drifts (e.g., 0.9 in 2010s, 1.1 in 2020s): Systematic shift, investigate macro drivers
\item Wide swings during crises (2020, 2022) but stable otherwise: Regime-dependent but static in normal times
\end{itemize}

\subsection{Conditional on instability: Macro driver analysis}

\textbf{Only if Chow test rejects}, estimate second-stage regression:

\begin{equation}
\hat{\beta}_w = \delta_0 + \delta_{\text{VIX}} \cdot \overline{\text{VIX}}_w + \delta_{\text{OAS}} \cdot \log(\overline{\text{OAS}}_{index,w}) + \delta_r \cdot \overline{r}_{10y,w} + \eta_w
\end{equation}

where $\overline{X}_w$ denotes window-average of variable $X$.

\textbf{Theory-based predictions} \cite{Wuebben2025}:
\begin{enumerate}
\item $\delta_{\text{VIX}} > 0$: High volatility amplifies sensitivity, especially for short maturities (flight-to-quality concentrates in long end, front end whipsaws more)

\item $\delta_{\text{OAS}} < 0$: Wide spreads reduce dispersion in $\lambda$ (convergence to Regime 5 where all bonds near default, proportionality improves)

\item $\delta_r$: Ambiguous. Higher rates increase discount effect (reduce duration), possibly dampening elasticities. Empirical question.
\end{enumerate}

\textbf{Economic significance threshold}: Only declare time-variation meaningful if macro state changes $\lambda$ by $>$20\% over sample range.

Example: If $\delta_{\text{VIX}} = 0.01$ and VIX ranges from 10 to 40, the effect is $0.01 \times 30 = 0.30$, or 30\% change in $\beta$. This is economically large.

If $\delta_{\text{VIX}} = 0.002$, effect is 6\%—within noise, ignore.

\subsection{Maturity-specific time-variation}

Theory predicts time-variation should differ by maturity: short-maturity IG bonds most affected.

\textbf{Test}: Estimate rolling $\beta_w$ separately for maturity buckets:
\begin{equation}
y_{i,t} = \alpha_{w,m} + \beta_{w,m} \cdot [\lambda^{\text{Merton}}_{i,t} \cdot f_{DTS,t}] + \varepsilon_{i,t}, \quad i \in \text{maturity bucket } m, \, t \in w
\end{equation}

Then regress:
\begin{equation}
\hat{\beta}_{w,m} = \delta_{0,m} + \delta_{\text{VIX},m} \cdot \overline{\text{VIX}}_w + \eta_{w,m}
\end{equation}

\textbf{Prediction}: $\delta_{\text{VIX},1y} > \delta_{\text{VIX},5y} > \delta_{\text{VIX},10y}$ (short bonds more regime-dependent).

If this pattern holds, supports theory-based intuition about crisis dynamics.

\subsection{Practical implication assessment}

\textbf{Question}: Even if time-variation is statistically significant, does it matter for portfolio management?

\textbf{Scenario analysis}: Compare static vs time-varying $\lambda$ for:
\begin{enumerate}
\item \textbf{Risk model accuracy}: Does time-varying $\lambda$ reduce tracking error in out-of-sample hedging?

\item \textbf{Crisis performance}: During 2020 COVID shock, did static $\lambda$ severely misprice front-end IG?

\item \textbf{Operational complexity}: Time-varying $\lambda$ requires daily macro state inputs and recalibration. Worth the cost?
\end{enumerate}

\textbf{Recommendation framework}:
\begin{itemize}
\item If time-variation changes risk estimates by $<$10\% except during rare crises: Use static $\lambda$, add crisis overlays manually

\item If time-variation changes risk estimates by $>$20\% routinely: Implement time-varying $\lambda$ with macro state

\item If time-variation important only for specific buckets (e.g., 1--2y IG): Use static for most bonds, time-varying for front end only
\end{itemize}

\subsection{Deliverables for Stage C}

\begin{itemize}
\item \textbf{Table C.1:} Rolling window stability test
\begin{itemize}
\item Rows: Time windows (2010--2011, 2011--2012, ..., 2023--2024)
\item Columns: $\hat{\beta}_w$, standard error, 95\% CI, sample size
\item Separate panels for IG and HY
\item Chow test: $F$-statistic, $p$-value
\end{itemize}

\item \textbf{Table C.2:} Macro driver regression (conditional on instability)
\begin{itemize}
\item Coefficients: $\hat{\delta}_{\text{VIX}}$, $\hat{\delta}_{\text{OAS}}$, $\hat{\delta}_r$
\item Standard errors, $t$-statistics, $R^2$
\item Economic significance: Effect of 1 SD change in each macro variable on $\beta$
\item Test predicted signs: $\delta_{\text{VIX}} > 0$, $\delta_{\text{OAS}} < 0$
\end{itemize}

\item \textbf{Table C.3:} Maturity-specific time-variation
\begin{itemize}
\item Rows: Maturity buckets (1--2y, 3--5y, 7--10y)
\item Columns: $\hat{\delta}_{\text{VIX},m}$, standard error, $t$-statistic
\item Test: Is $\delta_{\text{VIX},1y}$ significantly larger than $\delta_{\text{VIX},10y}$?
\end{itemize}

\item \textbf{Figure C.1:} Time series of $\hat{\beta}_w$ for IG and HY
\begin{itemize}
\item X-axis: Year (2010--2024)
\item Y-axis: $\hat{\beta}_w$
\item Point estimates with 95\% confidence bands
\item Horizontal line at $\beta = 1$ (theory prediction)
\item Shade crisis periods (2020 COVID, 2022 rate shock)
\item Interpretation: Does $\hat{\beta}_w$ spike during crises?
\end{itemize}

\item \textbf{Figure C.2:} $\hat{\beta}_w$ vs macro state variables
\begin{itemize}
\item Panel A: $\hat{\beta}_w$ (y-axis) vs $\overline{\text{VIX}}_w$ (x-axis)
\item Panel B: $\hat{\beta}_w$ (y-axis) vs $\log(\overline{\text{OAS}}_w)$ (x-axis)
\item Scatter with OLS fit line
\item Color-code by time period (pre-2020, COVID, post-COVID)
\item Shows whether macro variables predict time-variation
\end{itemize}

\item \textbf{Figure C.3:} Implied $\lambda_{i,t}$ for representative bonds over time
\begin{itemize}
\item Three lines: 1-year BBB, 5-year BBB, 10-year BBB (all industrial)
\item Static $\lambda$ (dashed) vs time-varying $\lambda_t$ (solid)
\item Shows when and how much time-variation matters
\end{itemize}

\item \textbf{Figure C.4:} Scenario analysis—crisis vs normal
\begin{itemize}
\item Histogram of spread changes during normal periods (VIX $<$ 20)
\item Histogram of spread changes during stress (VIX $>$ 30)
\item Overlay: Static Merton prediction, time-varying prediction
\item Shows whether static model fails systematically in crises
\end{itemize}

\item \textbf{Summary and recommendation (3--4 pages):}
\begin{enumerate}
\item \textbf{Is relationship stable?} (Chow test results)

\item \textbf{If unstable, what drives it?} (Macro driver analysis)

\item \textbf{Is instability economically meaningful?} (Effect size in % terms)

\item \textbf{Does theory-based intuition hold?} (VIX amplifies front-end, OAS compresses dispersion)

\item \textbf{Practical recommendation}:
\begin{itemize}
\item Use static $\lambda$: Adequate for normal markets, simple implementation
\item Use time-varying $\lambda$: Necessary for crisis periods, worth complexity
\item Hybrid: Static baseline with crisis adjustments (VIX $>$ 30)
\end{itemize}

\item \textbf{Implication for production}: If static suffices, Stage E will select among pure/calibrated Merton. If time-varying needed, add macro state to production spec.
\end{enumerate}
\end{itemize}

\subsection{Decision point}

\begin{itemize}
\item \textbf{If Chow test $p > 0.10$}: Static $\lambda$ sufficient. Proceed to Stage D (robustness) with confidence in stable baseline.

\item \textbf{If Chow test $p < 0.10$ but economic effects $<$20\%}: Marginal time-variation. Treat as robustness consideration, not core production feature.

\item \textbf{If Chow test $p < 0.01$ and effects $>$20\% in crises}: Time-varying $\lambda$ necessary. Incorporate macro state in Stage E production specification.
\end{itemize}


\section{Stage D: Robustness and Extensions}
\label{sec:stage-d}

\subsection{Objective and positioning}

\textbf{Prerequisite}: Stages A--C established whether/how Merton predictions hold for \emph{mean} spread changes in standard conditions.

\textbf{Objective}: Test robustness across:
\begin{enumerate}
\item Tail events (quantile regression)
\item Shock types (systematic vs idiosyncratic decomposition)
\item Spread components (default vs liquidity)
\end{enumerate}

\textbf{Key framing}: These are \textbf{secondary} tests. If Stages A--C show Merton fails, Stage D helps diagnose \emph{why}. If Stages A--C validate Merton, Stage D confirms it's not just a mean effect.

\subsection{D.1: Tail behavior (quantile regression)}

\subsubsection{Motivation}

Merton model assumes normal shocks (geometric Brownian motion for firm value). If tails differ from mean, this suggests:
\begin{itemize}
\item Jump-to-default risk not captured by continuous diffusion
\item Liquidity evaporation in stress (left tail)
\item Asymmetric investor behavior (panic selling vs gradual buying)
\end{itemize}

\subsubsection{Specification}

For quantiles $\tau \in \{0.05, 0.10, 0.25, 0.50, 0.75, 0.90, 0.95\}$, estimate:

\begin{equation}
Q_{\tau}(y_{i,t} \mid f_{DTS,t}, \lambda^{\text{Merton}}_i) = \alpha_{\tau} + \beta_{\tau} \cdot [\lambda^{\text{Merton}}_i \cdot f_{DTS,t}]
\end{equation}

This models the $\tau$-th conditional quantile of spread changes, allowing elasticity to differ across the distribution.

\textbf{Merton prediction} \cite{Wuebben2025}: $\beta^{(G)} \approx \beta^{(S)} \approx \beta^{(I)} \approx 1$ (all shocks respect structural elasticities).

\textbf{Diagnostics}:
\begin{itemize}
\item If $\beta_{0.05} \gg \beta_{0.95}$: Left tail (spread widening) has amplified sensitivity. Consistent with jump-to-default or liquidity spirals.

\item If $\beta_{0.95} > \beta_{0.05}$: Right tail (spread tightening) more sensitive. Suggests momentum/technical buying in rallies.

\item If $\beta_{\tau}$ U-shaped (high at both tails): Both extreme moves behave differently than moderate moves—non-linearity in spread dynamics.
\end{itemize}

\subsubsection{Practical implications}

\textbf{Risk management}:
\begin{itemize}
\item Use $\beta_{0.05}$ for Value-at-Risk (VaR) and Expected Shortfall (ES) calculations
\item Use $\beta_{0.50}$ for expected return / attribution models
\item If $\beta_{0.05} = 1.5 \times \beta_{0.50}$: Tail risk 50\% larger than mean-based models predict
\end{itemize}

\textbf{Stress testing}:
\begin{itemize}
\item Standard Merton $\lambda$ may underestimate losses in left-tail scenarios
\item Adjust: $DTS^{*,\text{stress}}_{i,t} = \beta_{0.05} \cdot \lambda^{\text{Merton}}_i \cdot DTS_{i,t}$
\end{itemize}

\subsubsection{Deliverables for D.1}

\begin{itemize}
\item \textbf{Table D.1:} Quantile-specific $\beta_{\tau}$ estimates
\begin{itemize}
\item Rows: $\tau \in \{0.05, 0.10, 0.25, 0.50, 0.75, 0.90, 0.95\}$
\item Columns: $\hat{\beta}_{\tau}$, standard error, 95\% CI
\item Separate panels for IG and HY
\item Test: $H_0: \beta_{0.05} = \beta_{0.50}$ and $H_0: \beta_{0.95} = \beta_{0.50}$
\end{itemize}

\item \textbf{Figure D.1:} Plot of $\hat{\beta}_{\tau}$ across $\tau \in [0.05, 0.95]$
\begin{itemize}
\item X-axis: Quantile $\tau$
\item Y-axis: $\hat{\beta}_{\tau}$
\item Horizontal line at $\beta = 1$ (Merton prediction)
\item Confidence bands (bootstrap)
\item Interpretation: Flat line = Merton works across distribution. Upward/downward slope = asymmetry.
\end{itemize}

\item \textbf{Table D.2:} Tail amplification factors by bucket
\begin{itemize}
\item Representative buckets: 1y BBB, 5y BBB, 10y BBB
\item Columns: $\beta_{0.50}$, $\beta_{0.05}$, Ratio $\beta_{0.05}/\beta_{0.50}$
\item Shows which bonds have largest tail amplification
\end{itemize}

\item \textbf{Interpretation note}: Are tail deviations concentrated in specific regimes (e.g., front-end IG)? If so, suggests these bonds have additional jump risk beyond Merton's continuous framework.
\end{itemize}

\subsection{D.2: Shock decomposition}

\subsubsection{Motivation}

Merton model treats all firm value shocks identically—whether macro, sector, or idiosyncratic, the elasticity $\lambda_i$ should be the same because all operate through the firm's asset value.

Empirically test: Do different shock types exhibit different elasticities? If so, suggests mechanisms beyond firm fundamentals (e.g., liquidity contagion for sector shocks, information asymmetry for idiosyncratic shocks).

\subsubsection{Factor construction}

Decompose bond $i$'s spread change into orthogonal components:

\begin{align}
y_{i,t} &= \underbrace{f^{(G)}_{DTS,t}}_{\text{Global factor}} + \underbrace{f^{(S)}_{DTS,s(i),t}}_{\text{Sector factor}} + \underbrace{f^{(I)}_{DTS,i,t}}_{\text{Issuer-specific}} + \varepsilon_{i,t}
\end{align}

\textbf{Estimation procedure}:

\textbf{Step 1}: Global factor
\begin{equation}
f^{(G)}_{DTS,t} = \frac{\Delta OAS^{(U)}_t}{OAS^{(U)}_{t-1}}
\end{equation}

\textbf{Step 2}: Sector factors (orthogonalized to global)
\begin{equation}
f^{(S)}_{DTS,s,t} = \frac{\Delta OAS^{(U)}_{s,t}}{OAS^{(U)}_{s,t-1}} - f^{(G)}_{DTS,t}
\end{equation}

\textbf{Step 3}: Issuer-specific (residual)
\begin{equation}
f^{(I)}_{DTS,i,t} = y_{i,t} - f^{(G)}_{DTS,t} - f^{(S)}_{DTS,s(i),t}
\end{equation}

\subsubsection{Multi-factor regression with Merton baseline}

Estimate:
\begin{equation}
y_{i,t} = \lambda^{(G)}_i \cdot f^{(G)}_{DTS,t} + \lambda^{(S)}_i \cdot f^{(S)}_{DTS,s(i),t} + \lambda^{(I)}_i \cdot f^{(I)}_{DTS,i,t} + \varepsilon_{i,t}
\end{equation}

\textbf{Constrained specification}: Impose $\lambda^{(G)}_i = \lambda^{(S)}_i = \lambda^{(I)}_i = \lambda^{\text{Merton}}_i$, estimate single coefficient:
\begin{equation}
y_{i,t} = \lambda^{\text{Merton}}_i \cdot [f^{(G)}_t + f^{(S)}_{s(i),t} + f^{(I)}_{i,t}] + \varepsilon_{i,t}
\end{equation}

\textbf{Unconstrained specification}: Allow separate coefficients:
\begin{equation}
y_{i,t} = \beta^{(G)} [\lambda^{\text{Merton}}_i \cdot f^{(G)}_t] + \beta^{(S)} [\lambda^{\text{Merton}}_i \cdot f^{(S)}_{s(i),t}] + \beta^{(I)} [\lambda^{\text{Merton}}_i \cdot f^{(I)}_{i,t}] + \varepsilon_{i,t}
\end{equation}

\textbf{Merton prediction}: $\beta^{(G)} \approx \beta^{(S)} \approx \beta^{(I)} \approx 1$ (all shocks respect structural elasticities).

\subsubsection{Diagnostic patterns}

\begin{itemize}
\item \textbf{If $\beta^{(G)} \approx \beta^{(S)} \approx \beta^{(I)} \approx 1$}: Merton applies uniformly—all shocks operate through firm value.

\item \textbf{If $\beta^{(S)} > \beta^{(G)}$}: Sector shocks have amplified effects. Suggests contagion, correlation trading, or common liquidity factors beyond fundamentals.

\item \textbf{If $\beta^{(I)} \gg \beta^{(G)}$}: Idiosyncratic news has exaggerated spread impact. Consistent with information asymmetry, adverse selection in trading.

\item \textbf{If $\beta^{(G)} < 1$ but $\beta^{(I)} > 1$}: Bonds under-react to macro (diversified portfolios stabilize) but over-react to issuer-specific (concentrated positions, forced selling).
\end{itemize}

\subsubsection{Deliverables for D.2}

\begin{itemize}
\item \textbf{Table D.3:} Variance decomposition
\begin{itemize}
\item Rows: Rating $\times$ maturity buckets
\item Columns: \% variance from Global, Sector, Issuer-specific, Residual
\item Shows relative importance of each factor type
\item Separate for IG and HY (expect IG more global-driven, HY more issuer-specific)
\end{itemize}

\item \textbf{Table D.4:} Shock-specific elasticities
\begin{itemize}
\item Rows: $\beta^{(G)}$, $\beta^{(S)}$, $\beta^{(I)}$
\item Columns: Estimate, standard error, 95\% CI
\item Test: $H_0: \beta^{(G)} = \beta^{(S)} = \beta^{(I)} = 1$ (joint test)
\item Test: Pairwise $H_0: \beta^{(G)} = \beta^{(S)}$, etc.
\end{itemize}

\item \textbf{Figure D.2:} Bar chart of $\hat{\beta}$ by factor type
\begin{itemize}
\item Three bars: Global, Sector, Issuer-specific
\item Error bars for 95\% CI
\item Horizontal line at $\beta = 1$
\item Separate panels for IG and HY
\end{itemize}

\item \textbf{Interpretation}: If all three $\approx$ 1, Merton universally applicable. If sector/issuer deviate, need factor-specific adjustments in production models.
\end{itemize}

\subsection{D.3: Liquidity adjustment}

\subsubsection{Motivation}

OAS includes both default risk and liquidity premium. Merton model predicts elasticity for \emph{default component only}. If liquidity shocks don't respect structural elasticities, need to decompose.

\subsubsection{Liquidity-adjusted spread construction}

\textbf{Step 1}: Estimate liquidity component cross-sectionally each period:
\begin{equation}
s^{\text{liq}}_{i,t} = \phi_0 + \phi_1 \text{BidAsk}_{i,t} + \phi_2 \log(\text{Size}_i) + \phi_3 \log(\text{Turnover}_{i,t}) + \phi_4 \text{Age}_{i,t} + \eta_{i,t}
\end{equation}

Standard approach: Cross-sectional regression within each rating $\times$ maturity cell to avoid confounding credit quality with liquidity.

\textbf{Step 2}: Define default component as residual:
\begin{equation}
OAS^{\text{def}}_{i,t} = OAS_{i,t} - \widehat{OAS}^{\text{liq}}_{i,t}
\end{equation}

\textbf{Step 3}: Compute default-based spread changes:
\begin{equation}
y^{\text{def}}_{i,t} = \frac{\Delta OAS^{\text{def}}_{i,t}}{OAS^{\text{def}}_{i,t-1}}
\end{equation}

\subsubsection{Re-estimate Merton fit on default component}

Run Stage B regression using $y^{\text{def}}_{i,t}$ as dependent variable:
\begin{equation}
y^{\text{def}}_{i,t} = \alpha + \beta_{\text{def}} \cdot [\lambda^{\text{Merton}}_i \cdot f^{\text{def}}_{DTS,t}] + \varepsilon_{i,t}
\end{equation}

where $f^{\text{def}}_{DTS,t}$ is the index-level default spread factor (after removing liquidity).

\textbf{Theory prediction}: $\beta_{\text{def}} \approx 1$ (Merton works for default component) and $R^2_{\text{def}} > R^2_{\text{total}}$ (less noise without liquidity shocks).

\subsubsection{Diagnostic patterns}

\begin{itemize}
\item \textbf{If $\beta_{\text{def}} \approx 1$ and $\beta_{\text{total}} \approx 1$}: Liquidity adjustment unnecessary—Merton works on total OAS.

\item \textbf{If $\beta_{\text{def}} \approx 1$ but $\beta_{\text{total}} < 1$}: Liquidity shocks dampen spread movements. Total OAS underestimates default sensitivity.

\item \textbf{If $\beta_{\text{def}} < \beta_{\text{total}}$}: Liquidity shocks amplify spread movements beyond Merton. Need separate liquidity beta.

\item \textbf{If improvement concentrated in illiquid bonds}: Signals that liquidity noise matters for small size, low turnover bonds but not for liquid benchmarks.
\end{itemize}

\subsubsection{Practical decision rule}

\begin{mdframed}[backgroundcolor=blue!5!white,roundcorner=4pt]
\textbf{When to Use Liquidity Adjustment:}

\textbf{Liquid IG (BidAsk $<$ 50bps, Size $>$ \$1B):} 
\begin{itemize}
\item Use total OAS, liquidity adjustment negligible ($<$5bps typical)
\item Liquidity $R^2$ low, not worth complexity
\end{itemize}

\textbf{Illiquid IG (BidAsk $>$ 100bps, Size $<$ \$500M):}
\begin{itemize}
\item Liquidity premium 20--50bps, material fraction of spread
\item Consider decomposition for risk models
\end{itemize}

\textbf{HY (BidAsk typically $>$ 200bps):}
\begin{itemize}
\item Decompose into $OAS^{\text{def}}$ and $OAS^{\text{liq}}$
\item Use $\lambda^{\text{def}}$ from Merton for default component
\item Add separate $\lambda^{\text{liq}}$ empirically estimated (not theory-based)
\end{itemize}
\end{mdframed}

\subsubsection{Deliverables for D.3}

\begin{itemize}
\item \textbf{Table D.5:} Liquidity model estimates
\begin{itemize}
\item Cross-sectional regression: $\hat{\phi}_1$ (bid-ask), $\hat{\phi}_2$ (size), $\hat{\phi}_3$ (turnover), $\hat{\phi}_4$ (age)
\item $R^2$ by rating $\times$ maturity bucket
\item Typical liquidity premium: Mean $\widehat{OAS}^{\text{liq}}$ by bucket
\end{itemize}

\item \textbf{Table D.6:} Merton fit comparison
\begin{itemize}
\item Rows: Total OAS, Default component only
\item Columns: $\hat{\beta}$, $R^2$, RMSE
\item Test: Is $\beta_{\text{def}}$ closer to 1 than $\beta_{\text{total}}$?
\item Improvement: $\Delta R^2 = R^2_{\text{def}} - R^2_{\text{total}}$
\end{itemize}

\item \textbf{Table D.7:} Improvement by liquidity regime
\begin{itemize}
\item Split sample by liquidity quartiles (bid-ask or turnover)
\item For each quartile: $\beta_{\text{total}}$, $\beta_{\text{def}}$, $\Delta R^2$
\item Shows whether adjustment helps mainly for illiquid bonds
\end{itemize}

\item \textbf{Figure D.3:} Scatter plot: $\beta_{\text{def}} - \beta_{\text{total}}$ (y-axis) vs average bid-ask (x-axis)
\begin{itemize}
\item Each point: a rating $\times$ maturity bucket
\item Interpretation: Positive slope = illiquid bonds benefit more from decomposition
\end{itemize}

\item \textbf{Interpretation and recommendation}:
\begin{itemize}
\item For which bond types is liquidity adjustment material?
\item Does Merton work better on default component than total OAS?
\item Production implication: Use total OAS for liquid IG, decompose for HY?
\end{itemize}
\end{itemize}

\subsection{Summary of Stage D}

Stage D is \textbf{diagnostic}, not \textbf{decisional}. It answers:

\begin{enumerate}
\item \textbf{Where does Merton fail?} (Tails? Specific shock types? Liquidity-contaminated spreads?)

\item \textbf{How large are the failures?} (20\% effect in tails? 50\%? Negligible?)

\item \textbf{Are failures systematic?} (Always front-end IG? Always sector shocks?)
\end{enumerate}

\textbf{Use Stage D results in Stage E} to decide:
\begin{itemize}
\item If tail effects large: Add quantile-specific $\lambda$ for VaR/ES
\item If sector shocks deviate: Consider sector adjustments beyond Merton
\item If liquidity matters: Decompose OAS for HY, keep total for IG
\end{itemize}

But don't let Stage D derail the main program. If Stages A--C showed Merton works well on average, Stage D refinements are enhancements, not requirements.

% PART 3: Stage E (Production Specification), Summary, Conclusion
% This continues from Part 2

\section{Stage E: Production Specification Selection}
\label{sec:stage-e}

\subsection{Objective}

Stage E selects the parsimonious production model that balances theoretical coherence, empirical fit, and implementation cost. 

\textbf{Key principle}: Use \emph{hierarchical testing} guided by theory. Stop at the simplest adequate model. Don't over-engineer.

\textbf{Philosophical stance}: Theory provides a strong prior. Only deviate when data strongly reject it. The burden of proof is on the more complex model.

\subsection{Decision framework: Hierarchical testing}

\begin{mdframed}[backgroundcolor=yellow!10!white,roundcorner=4pt]
\textbf{Level 1: Is standard DTS adequate?}

\textbf{Test}: Stage A, $F$-test for $H_0: \beta^{(1)} = \beta^{(2)} = \cdots = \beta^{(K)} = 1$

\textbf{Decision rule}:
\begin{itemize}
\item If fail to reject ($p > 0.10$): \textbf{Use standard DTS}. No adjustments needed. Done.
\item If reject: Proceed to Level 2
\end{itemize}

\textbf{Production spec}: $y_{i,t} \approx f_{DTS,t}$

\textbf{Parameters}: 0

\textbf{Implementation}: Trivial (already in all systems)

\vspace{0.3cm}

\textbf{Level 2: Does pure Merton suffice?}

\textbf{Test}: Stage B Specification B.1, $H_0: \beta_{\text{Merton}} = 1$

\textbf{Decision rule}:
\begin{itemize}
\item If $\hat{\beta}_{\text{Merton}} \in [0.9, 1.1]$ \textbf{and} $R^2_{\text{Merton}} > 0.9 \times R^2_{\text{buckets}}$:

\textbf{Use Pure Merton} (Tables~\ref{tab:merton-lambda-T}--\ref{tab:merton-lambda-s} from \cite{Wuebben2025}). Done.

\item If systematic bias ($\hat{\beta}_{\text{Merton}}$ outside $[0.9, 1.1]$): Proceed to Level 3

\item If poor fit ($R^2_{\text{Merton}} < 0.7 \times R^2_{\text{buckets}}$): Proceed to Level 4
\end{itemize}

\textbf{Production spec}: 
\begin{equation}
\lambda^{\text{prod}}_i = \lambda_T(T_i; 5y, s_i) \times \lambda_s(s_i; 100)
\end{equation}

\textbf{Parameters}: 0 (lookup tables from \cite{Wuebben2025})

\textbf{Implementation}: Read current $s_i, T_i$, look up $\lambda$ from tables or use closed-form $(s/100)^{-0.25}$

\vspace{0.3cm}

\textbf{Level 3: Calibrated Merton}

\textbf{Condition}: Merton has right functional form but wrong scale

\textbf{Production spec}: 
\begin{equation}
\lambda^{\text{prod}}_i = c_0 \cdot \lambda_T(T_i; 5y, s_i) \times \lambda_s(s_i; 100)^{c_s}
\end{equation}

where:
\begin{itemize}
\item $c_0$: overall scaling factor (estimated from Stage B: $\hat{c}_0 = \hat{\beta}_{\text{Merton}}$)
\item $c_s$: adjust power-law exponent (theory suggests $-0.25$, data may differ)
\end{itemize}

\textbf{Estimation}: 
\begin{equation}
y_{i,t} = \alpha + \beta_0 [\lambda_T(T_i) \cdot f_{DTS,t}] + \beta_s [\lambda_s(s_i)^c \cdot f_{DTS,t}] + \varepsilon_{i,t}
\end{equation}

Grid search over $c \in [-0.5, 0]$ to maximize $R^2$, then calibrate $\beta_0, \beta_s$.

\textbf{Decision rule}:
\begin{itemize}
\item If $\hat{c}_0 \in [0.8, 1.2]$ and $\hat{c}_s \in [-0.35, -0.15]$: Theory approximately correct. Use calibrated version.

\textbf{Production spec} = \textbf{Calibrated Merton} with 2 parameters.

\item If outside these ranges: Proceed to Level 4
\end{itemize}

\textbf{Parameters}: 2 ($c_0, c_s$)

\textbf{Implementation}: Requires one-time calibration using sample data, then fixed multipliers

\vspace{0.3cm}

\textbf{Level 4: Full empirical}

\textbf{Condition}: Calibrated Merton inadequate, need unrestricted functional form

\textbf{Production spec}:
\begin{equation}
\lambda^{\text{prod}}_i = \exp(\beta_0 + \beta_M \log M_i + \beta_s \log s_i + \beta_{M^2} (\log M_i)^2 + \beta_{Ms} \log M_i \cdot \log s_i + \sum_{\text{sector}} \beta_{\text{sec}})
\end{equation}

Estimate via Stage B Specification B.3 (unrestricted).

\textbf{Decision rule}:
\begin{itemize}
\item If $R^2_{\text{unrestricted}} - R^2_{\text{calibrated}} > 0.05$: Empirical spec justified

\textbf{Production spec} = \textbf{Empirical} with 8--12 parameters

\item Otherwise: Stay at Level 3 (principle of parsimony)
\end{itemize}

\textbf{Parameters}: 8--12 (maturity terms, spread terms, interactions, sector dummies)

\textbf{Implementation}: Requires regression estimation, periodic recalibration (annually)

\vspace{0.3cm}

\textbf{Level 5: Time-varying (optional)}

\textbf{Condition}: Stage C showed significant instability ($p < 0.01$ in Chow test) \textbf{and} macro state changes $\lambda$ by $>$30\% during crises

\textbf{Production spec}:
\begin{equation}
\lambda^{\text{prod}}_{i,t} = \lambda^{\text{base}}_i \times \exp(\gamma_{\text{VIX}} \cdot \text{VIX}_t + \gamma_{\text{OAS}} \cdot \log(\text{OAS}_{index,t}))
\end{equation}

where $\lambda^{\text{base}}_i$ comes from Level 2, 3, or 4.

\textbf{Decision rule}:
\begin{itemize}
\item If improvement in crisis periods substantial (RMSE reduction $>$20\%) \textbf{and} operational complexity acceptable:

\textbf{Production spec} = \textbf{Time-varying} with base + 2 macro parameters

\item Otherwise: Use static $\lambda^{\text{base}}_i$ with manual crisis overlays
\end{itemize}

\textbf{Parameters}: Base parameters + 2 (macro state coefficients)

\textbf{Implementation}: Requires daily macro data feeds, dynamic recalculation
\end{mdframed}

\subsection{Cost-benefit analysis}

\begin{table}[h]
\centering
\caption{Model Comparison: Parsimony vs Performance}
\begin{tabular}{lcccp{4cm}}
\toprule
\textbf{Specification} & \textbf{Params} & \textbf{$R^2$} & \textbf{Impl.} & \textbf{Use When} \\
\midrule
Standard DTS & 0 & $R^2_0$ & Trivial & No cross-sectional variation \\
\midrule
Pure Merton & 0 & $R^2_M$ & Simple & Theory unbiased, $\beta \in [0.9,1.1]$ \\
\midrule
Calibrated Merton & 2 & $R^2_M + \delta_1$ & Moderate & Theory right structure, needs scaling \\
\midrule
Empirical & 10 & $R^2_M + \delta_2$ & Complex & Theory inadequate, $\Delta R^2 > 0.05$ \\
\midrule
Time-varying & 12 & $R^2_M + \delta_3$ & Very complex & Significant instability, large crisis effects \\
\bottomrule
\end{tabular}
\end{table}

\textbf{Expected incremental $R^2$ gains}:
\begin{itemize}
\item Pure Merton vs Standard DTS: +5--15\% (if Regime 2 prevalent)
\item Calibrated vs Pure Merton: +1--3\% (scaling correction)
\item Empirical vs Calibrated: +2--5\% (if sectors or interactions matter)
\item Time-varying vs Static: +1--2\% overall, +10--20\% in crises
\end{itemize}

\textbf{Implementation cost ranking}:
\begin{enumerate}
\item Standard DTS: Zero (already implemented)
\item Pure Merton: Low (lookup tables, one-time coding)
\item Calibrated Merton: Low-moderate (one-time calibration, quarterly review)
\item Empirical: Moderate-high (annual recalibration, more inputs)
\item Time-varying: High (daily macro feeds, operational burden)
\end{enumerate}

\subsection{Out-of-sample validation}

For each candidate specification at its decision level, conduct rolling-window out-of-sample test.

\subsubsection{Methodology}

\textbf{Window structure}:
\begin{itemize}
\item Training window: 3 years (rolling)
\item Test window: 1 year (out-of-sample)
\item Roll forward by 1 year, repeat
\end{itemize}

\textbf{Procedure for each window}:
\begin{enumerate}
\item Estimate specification parameters using training data
\item Generate predictions $\hat{y}^{\text{model}}_{i,t} = \hat{\lambda}_{i,t-1} \cdot f_{DTS,t}$ for test window
\item Compute performance metrics
\item Roll window forward
\end{enumerate}

\subsubsection{Performance metrics}

\textbf{Metric 1: Forecast accuracy}
\begin{equation}
\text{RMSE}_{\text{OOS}} = \sqrt{\frac{1}{N_{\text{test}}} \sum_{i,t \in \text{test}} (y_{i,t} - \hat{y}_{i,t})^2}
\end{equation}

Compare across specifications. Lower is better.

\textbf{Metric 2: Out-of-sample $R^2$}
\begin{equation}
R^2_{\text{OOS}} = 1 - \frac{\sum (y_{i,t} - \hat{y}_{i,t})^2}{\sum (y_{i,t} - \bar{y})^2}
\end{equation}

Can be negative if model performs worse than mean. Compare to in-sample $R^2$ to assess overfitting.

\textbf{Metric 3: Hedge effectiveness}

Construct hedges based on model predictions:
\begin{itemize}
\item Portfolio: Long 1--2y BBB bonds
\item Hedge: Short $h_t = \lambda^{\text{model}}_{1y} / \lambda^{\text{model}}_{5y}$ units of 5y BBB bonds
\end{itemize}

Compute tracking error of hedged portfolio. Lower tracking error = better model.

\textbf{Metric 4: Sharpe ratio of mispricing signal}

Define mispricing:
\begin{equation}
\text{Mispricing}_{i,t} = y_{i,t} - \hat{y}^{\text{model}}_{i,t}
\end{equation}

If bond under-reacted last period (positive mispricing), predict mean-reversion. Trade on signal:
\begin{equation}
\text{Position}_{i,t+1} = -\text{sign}(\text{Mispricing}_{i,t})
\end{equation}

Better model $\Rightarrow$ stronger mean-reversion $\Rightarrow$ higher Sharpe on position.

\subsubsection{Regime-specific performance}

Evaluate each specification separately during:
\begin{enumerate}
\item \textbf{Normal periods}: VIX $<$ 20
\item \textbf{Stress periods}: VIX $\in [20, 30]$
\item \textbf{Crisis periods}: VIX $>$ 30
\end{enumerate}

\textbf{Key question}: Does time-varying specification outperform static mainly in crises? If so, operational complexity may not be worth 98\% of the time.

\subsection{Recommended approach}

\begin{mdframed}[backgroundcolor=green!5!white,roundcorner=4pt]
\textbf{Decision Protocol:}

\textbf{Step 1}: Execute hierarchical testing (Levels 1--5)

\textbf{Step 2}: At the level where you stop (e.g., Level 2 = Pure Merton), conduct out-of-sample validation against:
\begin{itemize}
\item Previous level (e.g., Standard DTS)
\item Next level (e.g., Calibrated Merton)
\end{itemize}

\textbf{Step 3}: If next level improves out-of-sample RMSE by $<$5\%, stick with current level (parsimony wins)

\textbf{Step 4}: If next level improves out-of-sample RMSE by $>$10\%, adopt it (complexity justified)

\textbf{Step 5}: Document production specification with:
\begin{itemize}
\item Parameter values (if any)
\item Implementation pseudo-code
\item Recalibration frequency
\item Performance benchmarks
\end{itemize}

\textbf{Guiding principle}: Occam's Razor—prefer simplest model with adequate fit. A 2-parameter model with $R^2=0.75$ beats a 20-parameter model with $R^2=0.78$.
\end{mdframed}

\subsection{Deliverables for Stage E}

\begin{itemize}
\item \textbf{Table E.1:} Hierarchical test results
\begin{itemize}
\item Rows: Levels 1--5
\item Columns: Test statistic, $p$-value, Decision (PASS / Proceed to next level)
\item Mark the stopping level
\end{itemize}

\item \textbf{Table E.2:} Model comparison (all candidate specs)
\begin{itemize}
\item Rows: Standard DTS, Pure Merton, Calibrated Merton, Empirical, Time-varying
\item Columns: Parameters, In-sample $R^2$, OOS $R^2$, OOS RMSE, Hedge tracking error, Sharpe ratio
\item Highlight recommended specification
\end{itemize}

\item \textbf{Table E.3:} Performance by regime
\begin{itemize}
\item Panel A: Normal periods (VIX $<$ 20)
\item Panel B: Stress periods (VIX 20--30)
\item Panel C: Crisis periods (VIX $>$ 30)
\item For each panel: OOS $R^2$ and RMSE for each specification
\item Shows where complexity pays off
\end{itemize}

\item \textbf{Table E.4:} Recommended production specification
\begin{itemize}
\item Specification name and level
\item Parameter estimates with standard errors
\item Implementation formula
\item Recalibration frequency
\item Expected $R^2$ improvement over baseline
\item Complexity rating (Low / Moderate / High)
\end{itemize}

\item \textbf{Figure E.1:} Out-of-sample $R^2$ over rolling windows
\begin{itemize}
\item X-axis: Test window start date
\item Y-axis: OOS $R^2$
\item Multiple lines: Standard DTS, Pure Merton, Calibrated, Empirical, Time-varying
\item Shade crisis periods
\item Shows stability and relative performance over time
\end{itemize}

\item \textbf{Figure E.2:} Forecast error distribution
\begin{itemize}
\item Histogram of $(y_{i,t} - \hat{y}_{i,t})$ for recommended spec
\item Overlay: Normal distribution with same mean/variance
\item Q-Q plot in corner panel
\item Check for unmodeled fat tails or asymmetry
\end{itemize}

\item \textbf{Figure E.3:} Scatter: Predicted vs actual spread changes
\begin{itemize}
\item X-axis: $\hat{y}_{i,t}$ (model prediction)
\item Y-axis: $y_{i,t}$ (actual)
\item 45-degree line
\item Color-code by regime (IG-narrow, IG-wide, HY, distressed)
\item Shows where model works best/worst
\end{itemize}

\item \textbf{Implementation blueprint (5--7 pages):}

\begin{enumerate}
\item \textbf{Algorithmic steps}:
\begin{itemize}
\item Input data required (bond OAS, maturity, rating, sector; macro state if time-varying)
\item Step-by-step calculation of $\lambda_i$ or $\lambda_{i,t}$
\item Computation of adjusted DTS: $DTS^*_{i,t} = \lambda_i \cdot DTS_{i,t}$
\end{itemize}

\item \textbf{Pseudo-code / function definitions}:

\begin{verbatim}
function compute_lambda(OAS, Maturity, Sector):
    # Pure Merton example
    lambda_T = lookup_maturity_adjustment(Maturity, OAS)
    lambda_s = (OAS / 100) ^ (-0.25)
    lambda = lambda_T * lambda_s
    return lambda

function lookup_maturity_adjustment(T, s):
    # Bilinear interpolation on Table 3.1
    # Returns lambda_T(T; 5y, s)
    ...
\end{verbatim}

\item \textbf{Lookup tables}: Full tables if using Pure Merton (or reference to theory paper)

\item \textbf{Recalibration protocol}:
\begin{itemize}
\item Frequency: Quarterly review for Pure Merton (check if spreads/maturities still in table range), Annual for Calibrated/Empirical
\item Procedure: Re-estimate using trailing 3-year window, compare to current parameters
\item Trigger for update: If new estimates differ by $>$20\%, adopt new parameters
\item Documentation: Maintain version history of parameter changes
\end{itemize}

\item \textbf{Edge case handling}:
\begin{itemize}
\item Very short maturity ($<$ 6 months): Use 6-month $\lambda$ as floor, don't extrapolate
\item Distressed spreads ($>$ 2000 bps): Cap $\lambda$ at 1.2 (proportionality improves in extreme stress)
\item Missing sector: Use aggregate cross-sector average
\item New issues ($<$ 3 months old): Apply new issue adjustment $\gamma_{\text{new}}$ if Stage B found significant effect
\end{itemize}

\item \textbf{Integration with existing systems}:
\begin{itemize}
\item Risk models: Replace $DTS_{i,t}$ with $DTS^*_{i,t} = \lambda_i \cdot DTS_{i,t}$ in portfolio risk calculations
\item Attribution: Decompose factor return into $f_{DTS,t}$ and cross-sectional $\lambda_i$ effect
\item Relative value: Flag bonds with large deviations from $\lambda$-adjusted fair value
\end{itemize}

\item \textbf{Performance monitoring}:
\begin{itemize}
\item Track out-of-sample $R^2$ monthly
\item Alert if $R^2$ drops below threshold (e.g., $<$ 50\% of historical average)
\item Compare to benchmark (Standard DTS) quarterly
\end{itemize}
\end{enumerate}

\item \textbf{Comparative performance analysis (3--4 pages)}:

\begin{enumerate}
\item \textbf{Executive summary}: How much does recommended spec improve over baseline DTS?
\begin{itemize}
\item RMSE reduction: e.g., 25\% lower forecast errors
\item $R^2$ improvement: e.g., from 0.55 to 0.72
\item Hedge tracking error: e.g., 30 bps lower annualized
\end{itemize}

\item \textbf{Which regimes see largest gains?}
\begin{itemize}
\item IG with wide maturity range: 40\% RMSE reduction (Regime 2 targeted)
\item IG with narrow maturity: 10\% reduction (smaller benefit)
\item HY: 15\% reduction (smaller maturity effects but credit quality variation)
\end{itemize}

\item \textbf{Economic value}: Translate statistical improvements to portfolio management gains
\begin{itemize}
\item \textbf{Example 1: Hedging efficiency}

``Portfolio manager long \$100M of 1-year BBB industrials, wants to hedge with 5-year BBB. 

Standard DTS: Hedge ratio = 1.0, residual tracking error = 120 bps/year

Merton-adjusted: Hedge ratio = 3.2, residual tracking error = 40 bps/year

Value: 80 bps lower tracking error, equivalent to \$800K lower unexpected P\&L volatility''

\item \textbf{Example 2: Relative value signals}

``Identifying rich/cheap bonds within issuer capital structure. 

Standard DTS: Assumes all bonds move proportionally, misses 300--500\% cross-maturity differences.

Merton-adjusted: Properly scales for maturity, identifies mispricings averaging 15 bps.

Trade: Long cheap (e.g. 10y), short rich (e.g. 1y), earn 30 bps as convergence occurs over 6 months.''

\item \textbf{Example 3: Portfolio construction}

``Constructing credit barbell (short + long maturity, avoid intermediate).

Standard DTS: Treats all durations equally, over-concentrates DTS risk in short end.

Merton-adjusted: Recognizes front-end 3--4$\times$ more sensitive, rebalances to equalize risk contribution.

Result: More stable returns, 25\% reduction in unexpected drawdowns during spread volatility spikes.''
\end{itemize}

\item \textbf{Limitations and caveats}:
\begin{itemize}
\item Out-of-sample performance may degrade if regime shifts outside historical experience
\item Liquidity crises (2008-style) may break Merton assumptions
\item Model requires clean data (OAS, maturity, liquidity proxies)—quality control critical
\item Parameter drift possible—requires monitoring and periodic recalibration
\end{itemize}

\item \textbf{Sensitivity to implementation choices}:
\begin{itemize}
\item Bucket definitions: Finer buckets improve fit but reduce sample size per bucket
\item Frequency: Daily more noisy, monthly smoother but lags; weekly optimal balance
\item Clustering: Week clustering adequate, bond clustering more conservative but similar results
\end{itemize}
\end{enumerate}
\end{itemize}

\section{Summary of Research Tasks}
\label{sec:summary-tasks}

This section provides an actionable checklist for executing the research program.

\begin{mdframed}[backgroundcolor=blue!3!white,roundcorner=4pt,innerleftmargin=10pt]
\textbf{Phase 0: Data and Universe Construction}

\begin{enumerate}[label=\arabic*., leftmargin=*]
\item Extract full histories of bond-level OAS, OASD, ratings, maturities, sectors, liquidity proxies for all constituents of Bloomberg Barclays U.S.\ Corporate IG and HY indices (2010--2024)

\item Construct weekly observation grid (Friday close or last trading day)

\item Apply bond-level filters: seniority, maturity $\geq$ 1y, liquidity $\geq$ 5 trades/month, price 30--150\% par

\item Construct issuer identifiers using Ultimate Parent + Seniority matching

\item Compute $y_{i,t} = \Delta OAS_{i,t} / OAS_{i,t-1}$ and $f^{(U)}_{DTS,t} = \Delta OAS^{(U)}_t / OAS^{(U)}_{t-1}$

\item Document sample size: bonds per week, issuers with 2+ bonds, total issuer-weeks

\item \textbf{Deliverable}: Sample construction memo (3--5 pages) + summary statistics table
\end{enumerate}

\medskip
\textbf{Phase 0.5: Structural Model Setup}

\begin{enumerate}[label=\arabic*., leftmargin=*, resume]
\item Implement Merton model functions: $\lambda_T(T; s)$ and $\lambda_s(s)$ using formulas from theory paper

\item Generate lookup tables (Tables~\ref{tab:merton-lambda-T}--\ref{tab:merton-lambda-s}) if not already available

\item For each bond-date, compute $\lambda^{\text{Merton}}_{i,t} = \lambda_T(T_i; 5y, s_i) \times \lambda_s(s_i; 100)$

\item Validate: Spot-check 10--20 bonds to ensure $\lambda^{\text{Merton}}$ values sensible (e.g., 1y IG $\lambda \approx 3$, 10y IG $\lambda \approx 0.6$)

\item \textbf{Deliverable}: Merton calculation functions (code) + validation memo
\end{enumerate}

\medskip
\textbf{Phase 1: Stage 0—Raw Validation}

\begin{enumerate}[label=\arabic*., leftmargin=*, start=13]
\item Identify all issuer-weeks with $\geq$ 2 bonds outstanding (same Ultimate Parent + Seniority)

\item For each issuer-week, estimate within-issuer regression:
\begin{equation*}
\frac{\Delta s_{ij,t}}{s_{ij,t-1}} = \alpha_{i,t} + \beta_{i,t} \cdot \lambda^{\text{Merton}}_{ij,t} + \epsilon_{ij,t}
\end{equation*}

\item Pool across issuer-weeks using weighted least squares (weight by inverse residual variance)

\item Estimate stacked specification with issuer-week fixed effects

\item Estimate maturity-bucket specification within issuer-weeks

\item Statistical tests: Wald test $H_0: \beta=1$, Spearman correlation (maturity vs $\lambda$), regime pattern test

\item \textbf{Deliverables}: Tables 0.1--0.2, Figures 0.1--0.2, written summary (2--3 pages)

\item \textbf{Decision point}: Does Merton provide adequate baseline? (If yes, emphasize theory-constrained track; if no, emphasize unrestricted)
\end{enumerate}

\medskip
\textbf{Phase 2: Stage A—Establish Variation}

\begin{enumerate}[label=\arabic*., leftmargin=*, start=21]
\item Define buckets: Rating (AAA/AA, A, BBB for IG; BB, B, CCC for HY) $\times$ Maturity (1--2y, 2--3y, 3--5y, 5--7y, 7--10y, 10y+) $\times$ Sector

\item Estimate bucket-level regressions (Spec A.1): $y_{i,t} = \alpha^{(k)} + \beta^{(k)} f_{DTS,t} + \varepsilon^{(k)}_{i,t}$

\item $F$-test for equality: $H_0: \beta^{(1)} = \cdots = \beta^{(K)}$

\item Two-step estimation (Spec A.2): Rolling window bond-specific betas, then cross-sectional regression on $(M_i, s_i, M_i^2, M_i \cdot s_i)$

\item \textbf{Deliverables}: Tables A.1--A.3, Figures A.1--A.2, diagnostic summary (2 pages)

\item \textbf{Decision point}: Is variation significant? (If $F$-test $p > 0.10$, stop—standard DTS adequate)
\end{enumerate}

\medskip
\textbf{Phase 3: Stage B—Test Theory}

\begin{enumerate}[label=\arabic*., leftmargin=*, start=27]
\item Estimate constrained Merton (Spec B.1): $y_{i,t} = \alpha + \beta_{\text{Merton}} \cdot [\lambda^{\text{Merton}}_{i,t} \cdot f_{DTS,t}] + \varepsilon_{i,t}$

\item Test: Wald $H_0: \beta_{\text{Merton}} = 1$

\item Estimate decomposed Merton (Spec B.2): Separate $\beta_T$ and $\beta_s$

\item Estimate unrestricted (Spec B.3): Flexible $\lambda(M_i, s_i, \text{sector}_i)$

\item Construct Theory vs Reality table: Empirical $\hat{\beta}^{(k)}$ vs Theoretical $\lambda^{\text{Merton}}_k$ for all buckets

\item Compare $R^2$: Does $R^2_{\text{Merton}} > 0.9 \times R^2_{\text{buckets}}$?

\item \textbf{Deliverables}: Tables B.1--B.3, Figures B.1--B.3, diagnostic summary (3--4 pages)

\item \textbf{Decision point}: Pure Merton / Calibrated / Unrestricted? Follow decision tree from Section~\ref{sec:stage-b}
\end{enumerate}

\medskip
\textbf{Phase 4: Stage C—Test Stability}

\begin{enumerate}[label=\arabic*., leftmargin=*, start=35]
\item Divide sample into non-overlapping 1-year windows

\item For each window, estimate: $y_{i,t} = \alpha_w + \beta_w \cdot [\lambda^{\text{Merton}}_{i,t} \cdot f_{DTS,t}] + \varepsilon_{i,t}$

\item Chow test: $H_0: \beta_1 = \beta_2 = \cdots = \beta_W$

\item \textbf{If stable} ($p > 0.10$): Stop Stage C, use static $\lambda$

\item \textbf{If unstable}: Estimate macro driver regression: $\hat{\beta}_w = \delta_0 + \delta_{\text{VIX}} \overline{\text{VIX}}_w + \delta_{\text{OAS}} \log(\overline{\text{OAS}}_w) + \eta_w$

\item Assess economic significance: Does macro state change $\lambda$ by $>$20\%?

\item Maturity-specific time-variation: Estimate $\beta_{w,m}$ separately by maturity bucket

\item \textbf{Deliverables}: Tables C.1--C.3, Figures C.1--C.4, summary (3--4 pages)

\item \textbf{Decision point}: Static sufficient / Time-varying necessary? Follow thresholds from Section~\ref{sec:stage-c}
\end{enumerate}

\medskip
\textbf{Phase 5: Stage D—Robustness}

\begin{enumerate}[label=\arabic*., leftmargin=*, start=44]
\item Quantile regression (D.1): Estimate $Q_{\tau}(y_{i,t}) = \alpha_{\tau} + \beta_{\tau} \cdot [\lambda^{\text{Merton}}_i \cdot f_{DTS,t}]$ for $\tau \in \{0.05, 0.50, 0.95\}$

\item Test: Is $\beta_{0.05}$ significantly different from $\beta_{0.50}$?

\item Shock decomposition (D.2): Construct orthogonalized factors $f^{(G)}_t, f^{(S)}_{s,t}, f^{(I)}_{i,t}$

\item Estimate: $y_{i,t} = \beta^{(G)} [\lambda^{\text{Merton}}_i f^{(G)}_t] + \beta^{(S)} [\lambda^{\text{Merton}}_i f^{(S)}_{s,t}] + \beta^{(I)} [\lambda^{\text{Merton}}_i f^{(I)}_{i,t}] + \varepsilon_{i,t}$

\item Test: $H_0: \beta^{(G)} = \beta^{(S)} = \beta^{(I)}$

\item Liquidity adjustment (D.3): Estimate $s^{\text{liq}}_{i,t}$ cross-sectionally, define $OAS^{\text{def}}_{i,t} = OAS_{i,t} - \widehat{OAS}^{\text{liq}}_{i,t}$

\item Re-estimate Stage B using $y^{\text{def}}_{i,t}$, compare $\beta_{\text{def}}$ to $\beta_{\text{total}}$

\item \textbf{Deliverables}: Tables D.1--D.7, Figures D.1--D.3, interpretation notes

\item \textbf{Use in Stage E}: Identify where/why Merton fails, inform production spec refinements
\end{enumerate}

\medskip
\textbf{Phase 6: Stage E—Production Specification}

\begin{enumerate}[label=\arabic*., leftmargin=*, start=53]
\item Execute hierarchical testing: Levels 1--5 from Section~\ref{sec:stage-e}

\item At each level, conduct stopping criterion test

\item Identify stopping level (e.g., Level 2 = Pure Merton)

\item Rolling-window out-of-sample validation: 3-year training, 1-year test, roll forward

\item Compute performance metrics: OOS $R^2$, RMSE, hedge tracking error, Sharpe ratio of mispricing signal

\item Compare recommended spec to previous and next levels

\item If next level improves RMSE by $<$5\%, stick with current (parsimony); if $>$10\%, adopt next level

\item Document production specification: Formula, parameters, implementation, recalibration protocol, edge cases

\item Create implementation blueprint: Pseudo-code, lookup tables, system integration guide

\item Comparative performance analysis: Economic value examples (hedging, relative value, portfolio construction)

\item \textbf{Deliverables}: Tables E.1--E.4, Figures E.1--E.3, implementation blueprint (5--7 pages), comparative analysis (3--4 pages), final recommendation template
\end{enumerate}

\medskip
\textbf{Phase 7: Final Report and Presentation}

\begin{enumerate}[label=\arabic*., leftmargin=*, start=64]
\item Compile all stage deliverables into comprehensive report (50--80 pages)

\item Executive summary (3--5 pages): Key findings, recommended spec, expected improvements

\item Technical appendix: Detailed methodology, robustness checks, sensitivity analyses

\item Create presentation deck for stakeholders (20--30 slides):
\begin{itemize}
\item Motivation: Why DTS adjustments needed
\item Theory: Merton predictions in plain language
\item Results: What we found (stage by stage)
\item Recommendation: Production spec with performance comparison
\item Implementation: Roadmap and timeline
\end{itemize}

\item Supplementary materials: Code repository, data dictionary, replication instructions
\end{enumerate}
\end{mdframed}



\section{Conclusion}

This research program provides a comprehensive framework for enhancing the Duration-Times-Spread (DTS) model through theory-guided empirical estimation. The key innovations relative to standard approaches include:

\subsection{Methodological contributions}

\begin{enumerate}
\item \textbf{Sequential testing with explicit decision points}: Rather than running all analyses in parallel, we proceed hierarchically with clear stopping rules. This prevents over-engineering and respects the principle that simpler models are preferable when adequate.

\item \textbf{Theory as baseline, not constraint}: Merton predictions provide strong priors that sharpen empirical tests and reduce parameter space, but we systematically evaluate when theory fails and data require more flexibility.

\item \textbf{Separation of existence and explanation}: Stage A establishes \emph{that} DTS betas vary before Stage B tests \emph{whether} theory explains it. This clarifies where standard DTS fails versus where our adjustments add value.

\item \textbf{Full-sample within-issuer methodology}: Stage 0's use of all issuer-weeks with within-issuer controls avoids the sample size, representativeness, and circular reasoning problems of event-week approaches.

\item \textbf{Hierarchical model selection}: The Level 1--5 framework guided by theory replaces atheoretical horse-racing. We stop at the simplest adequate model rather than maximizing in-sample fit.
\end{enumerate}

\subsection{Practical contributions}

\begin{enumerate}
\item \textbf{Implementable specifications}: Every stage produces actionable outputs—lookup tables, regression coefficients, implementation pseudo-code—not just statistical tests.

\item \textbf{Economic significance thresholds}: We don't just test statistical significance but also whether effects are large enough to matter for portfolio management (e.g., 20\% threshold for time-variation).

\item \textbf{Production readiness}: Stage E delivers a complete implementation blueprint with recalibration protocols, edge case handling, and performance monitoring.

\item \textbf{Regime awareness}: Rather than assuming one-size-fits-all, we explicitly acknowledge that DTS adjustments are most critical in Regime 2 (IG with wide maturity range) and less important elsewhere.
\end{enumerate}

\subsection{Academic contributions}

\begin{enumerate}
\item \textbf{First comprehensive empirical test}: This program provides the first market-wide test of structural model spread dynamics using modern bond data (2010--2024) with proper controls.

\item \textbf{Theory-data dialogue}: Rather than pure theory (no empirics) or pure empirics (ignoring theory), we create a dialogue where theory guides specification and data discipline theory.

\item \textbf{Quantification of regime effects}: We move beyond qualitative statements about when proportional spread movements fail to precise estimates of elasticity ratios by spread level and maturity.

\item \textbf{Decomposition of failures}: Stage D systematically diagnoses \emph{why} theory fails (tails? liquidity? specific shocks?) rather than just documenting that it does.
\end{enumerate}

\subsection{Expected outcomes}

Based on prior research and the theoretical framework, we anticipate:

\begin{enumerate}
\item \textbf{Stage 0 will validate Merton for IG cross-maturity effects}: The structural model's prediction \cite{Wuebben2025} that 1-year IG bonds have 3--4$\times$ higher sensitivity than 10-year bonds should be confirmed by within-issuer tests.

\item \textbf{Stage A will show significant variation}: $F$-tests will reject beta equality with $p < 0.01$, motivating the need for adjustments.

\item \textbf{Stage B will show Merton explains most IG variation}: For investment-grade with wide maturity dispersion, $R^2_{\text{Merton}} > 0.85 \times R^2_{\text{buckets}}$ likely. The production specification will be Pure or Calibrated Merton (Level 2--3).

\item \textbf{Stage C will find static $\l$ sufficient}: Time-variation likely marginal except in rare crises. Recommendation will be static baseline with optional crisis overlays.

\item \textbf{Stage D will identify HY liquidity effects}: Merton may perform better on default component than total OAS for high-yield bonds, suggesting liquidity adjustment valuable for illiquid segments.

\item \textbf{Stage E will recommend Calibrated Merton}: Most likely outcome is Level 3 with $c_0 \approx 0.9$--$1.1$ and $c_s \approx -0.25$, providing 20--30\% RMSE improvement over Standard DTS with minimal implementation complexity.
\end{enumerate}

\subsection{Broader implications}

This research program has implications beyond corporate credit DTS:

\begin{enumerate}
\item \textbf{Methodology template}: The hierarchical testing framework with theory priors applies to other markets where structural models exist (sovereign credit, mortgages, structured products).

\item \textbf{Theory-practice integration}: Demonstrates how academic models can inform practitioner tools when subjected to rigorous empirical validation rather than blind application.

\item \textbf{Risk model enhancement}: Improved DTS specifications translate directly to better portfolio risk estimates, hedge ratios, and stress test scenarios.

\item \textbf{Alpha generation}: Properly scaled relative-value signals (rich/cheap within issuer capital structures) can generate trading alpha when market prices deviate from theory-adjusted fair values.
\end{enumerate}

\subsection{Limitations and extensions}

Several important limitations should be acknowledged:

\begin{enumerate}
\item \textbf{Merton is approximate}: Even if empirically validated, the structural model rests on simplifying assumptions (log-normal firm value, constant parameters, no jumps). Stage D robustness checks partially address this.

\item \textbf{Parameter instability}: Coefficients estimated on 2010--2024 data may not hold if credit cycle or market structure changes. Ongoing monitoring essential.

\item \textbf{Data quality dependence}: Results critically depend on accurate OAS, maturity, and liquidity data. Garbage in, garbage out.

\item \textbf{Sample period limitations}: 2010--2024 includes only one true crisis (COVID 2020). Performance in 2008-style liquidity breakdown uncertain.
\end{enumerate}

\textbf{Future extensions} could include:

\begin{enumerate}
\item \textbf{International markets}: Apply framework to EUR, GBP, emerging market credit. Test whether Merton predictions universal or US-specific.

\item \textbf{Sector specialization}: Develop sector-specific $\lambda$ for Financials (regulatory capital effects), Utilities (rate sensitivity), Energy (commodity correlation).

\item \textbf{Integration with equity signals}: Combine with equity volatility, CDS spreads, structural model equity-based default probabilities.

\item \textbf{Machine learning augmentation}: Use ML to model $\lambda$ as non-linear function of characteristics, but benchmark against Merton baseline to ensure interpretability.

\item \textbf{Transaction costs and implementation}: Assess whether improved DTS adjustments survive real-world trading costs, market impact, and operational constraints.
\end{enumerate}

\subsection{Final perspective}

The Duration-Times-Spread framework is elegantly simple and widely adopted, but its assumption of proportional spread movements breaks down systematically in investment-grade markets with maturity dispersion. This research program demonstrates that structural credit theory \cite{Wuebben2025}, particularly the Merton model, provides the key to understanding \emph{when} and \emph{why} proportionality fails.

By integrating theoretical priors with rigorous empirical testing, we deliver not just statistical improvements but economically meaningful enhancements that respect both the wisdom of markets and the insights of financial theory. The hierarchical testing framework ensures we adopt only the complexity that data justify, resulting in production specifications that are simultaneously theoretically grounded, empirically validated, and practically implementable.

The ultimate deliverable is not merely a set of regression coefficients but a complete framework for understanding credit spread dynamics—one that portfolio managers can use with confidence, risk managers can rely on for accurate measurements, and researchers can build upon for future innovations.

\bibliographystyle{apalike}
\begin{thebibliography}{Collin-Dufresne et~al., 2001}
\bibitem[Merton, 1974]{Merton1974} Merton, R.~C. (1974). On the pricing of corporate debt: The risk structure of interest rates. \emph{Journal of Finance}, 29(2), 449--470.

\bibitem[Black \& Cox, 1976]{Black1976} Black, F., \& Cox, J.~C. (1976). Valuing corporate securities: Some effects of bond indenture provisions. \emph{Journal of Finance}, 31(2), 351--367.

\bibitem[Leland, 1994]{Leland1994} Leland, H.~E. (1994). Corporate debt value, bond covenants, and optimal capital structure. \emph{Journal of Finance}, 49(4), 1213--1252.

\bibitem[Ben Dor et~al., 2010]{BenDor2010} Ben Dor, A., Dynkin, L., Hyman, J., Houweling, P., Van Leeuwen, E., \& Penninga, O. (2010). DTS (Duration Times Spread). \emph{Barclays Capital Quantitative Portfolio Strategy Research}.

\bibitem[Wooldridge, 2010]{Wooldridge2010} Wooldridge, J. (2010). \emph{Econometric Analysis of Cross Section and Panel Data}. MIT Press.

\bibitem[Koenker, 2005]{Koenker2005} Koenker, R. (2005). \emph{Quantile Regression}. Cambridge University Press.

\bibitem[Campbell \& Taksler, 2003]{Campbell2003} Campbell, J.~Y., \& Taksler, G.~B. (2003). Equity volatility and corporate bond yields. \emph{Journal of Finance}, 58(6), 2321--2350.

\bibitem[Chen et~al., 2007]{Chen2007} Chen, L., Lesmond, D.~A., \& Wei, J. (2007). Corporate yield spreads and bond liquidity. \emph{Journal of Finance}, 62(1), 119--149.

\bibitem[Collin-Dufresne et~al., 2001]{CollinDufresne2001} Collin-Dufresne, P., Goldstein, R.~S., \& Martin, J.~S. (2001). The determinants of credit spread changes. \emph{Journal of Finance}, 56(6), 2177--2207.

\bibitem[Wuebben, 2025]{Wuebben2025} Wuebben, B.~J. (2025). When do credit spreads move proportionally? A structural model analysis of the Merton framework. \emph{Working Paper}, AllianceBernstein.
\end{thebibliography}
\end{document}
